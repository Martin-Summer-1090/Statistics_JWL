%% Generated by Sphinx.
\def\sphinxdocclass{jupyterBook}
\documentclass[letterpaper,10pt,english]{jupyterBook}
\ifdefined\pdfpxdimen
   \let\sphinxpxdimen\pdfpxdimen\else\newdimen\sphinxpxdimen
\fi \sphinxpxdimen=.75bp\relax
\ifdefined\pdfimageresolution
    \pdfimageresolution= \numexpr \dimexpr1in\relax/\sphinxpxdimen\relax
\fi
%% let collapsible pdf bookmarks panel have high depth per default
\PassOptionsToPackage{bookmarksdepth=5}{hyperref}
%% turn off hyperref patch of \index as sphinx.xdy xindy module takes care of
%% suitable \hyperpage mark-up, working around hyperref-xindy incompatibility
\PassOptionsToPackage{hyperindex=false}{hyperref}
%% memoir class requires extra handling
\makeatletter\@ifclassloaded{memoir}
{\ifdefined\memhyperindexfalse\memhyperindexfalse\fi}{}\makeatother

\PassOptionsToPackage{warn}{textcomp}

\catcode`^^^^00a0\active\protected\def^^^^00a0{\leavevmode\nobreak\ }
\usepackage{cmap}
\usepackage{fontspec}
\defaultfontfeatures[\rmfamily,\sffamily,\ttfamily]{}
\usepackage{amsmath,amssymb,amstext}
\usepackage{polyglossia}
\setmainlanguage{english}



\setmainfont{FreeSerif}[
  Extension      = .otf,
  UprightFont    = *,
  ItalicFont     = *Italic,
  BoldFont       = *Bold,
  BoldItalicFont = *BoldItalic
]
\setsansfont{FreeSans}[
  Extension      = .otf,
  UprightFont    = *,
  ItalicFont     = *Oblique,
  BoldFont       = *Bold,
  BoldItalicFont = *BoldOblique,
]
\setmonofont{FreeMono}[
  Extension      = .otf,
  UprightFont    = *,
  ItalicFont     = *Oblique,
  BoldFont       = *Bold,
  BoldItalicFont = *BoldOblique,
]



\usepackage[Bjarne]{fncychap}
\usepackage[,numfigreset=1,mathnumfig]{sphinx}

\fvset{fontsize=\small}
\usepackage{geometry}


% Include hyperref last.
\usepackage{hyperref}
% Fix anchor placement for figures with captions.
\usepackage{hypcap}% it must be loaded after hyperref.
% Set up styles of URL: it should be placed after hyperref.
\urlstyle{same}


\usepackage{sphinxmessages}



        % Start of preamble defined in sphinx-jupyterbook-latex %
         \usepackage[Latin,Greek]{ucharclasses}
        \usepackage{unicode-math}
        % fixing title of the toc
        \addto\captionsenglish{\renewcommand{\contentsname}{Contents}}
        \hypersetup{
            pdfencoding=auto,
            psdextra
        }
        % End of preamble defined in sphinx-jupyterbook-latex %
        

\title{Solution manual for excercises in Statistics the art and science of learning from data}
\date{Jan 07, 2024}
\release{}
\author{Martin Summer}
\newcommand{\sphinxlogo}{\vbox{}}
\renewcommand{\releasename}{}
\makeindex
\begin{document}

\pagestyle{empty}
\sphinxmaketitle
\pagestyle{plain}
\sphinxtableofcontents
\pagestyle{normal}
\phantomsection\label{\detokenize{intro::doc}}


\sphinxAtStartPar
This is a manual with worked solutions for all the exercises and
project steps in the course “Statistics: The art and science of learning
from data”. The manual is for the use of on\sphinxhyphen{}site learning facilitators to
support students in their efforts to work on exercises and assignments.

\sphinxAtStartPar
Note that there are usually several ways to solve a problem, in particular
how to implement certain tasks in R. So the solutions worked out here
should be seen as a suggested solution.

\sphinxAtStartPar
I have taken great efforts to assure that all exercises and problems can
be solved and when numerical answers are asked for that they are correct.
Should you despite of all this detect a mistake, please report the
issue to \sphinxhref{mailto:martin.summer@chello.at}{martin.summer@chello.at}
\begin{itemize}
\item {} 
\sphinxAtStartPar
{\hyperref[\detokenize{exercises_unit_1::doc}]{\sphinxcrossref{Exercises: Unit 1, Categorical data and proportions}}}

\item {} 
\sphinxAtStartPar
{\hyperref[\detokenize{exercises_unit_2::doc}]{\sphinxcrossref{Exercises: Unit 2, Summarizing and communication lot’s of data}}}

\item {} 
\sphinxAtStartPar
{\hyperref[\detokenize{exercises_unit_3::doc}]{\sphinxcrossref{Exercises: Unit 3, Generalizing observations from data and knowing what causes what}}}

\item {} 
\sphinxAtStartPar
{\hyperref[\detokenize{exercises_unit_4::doc}]{\sphinxcrossref{Exercises: Unit 4 Making predictions using regression}}}

\item {} 
\sphinxAtStartPar
{\hyperref[\detokenize{exercises_unit_5::doc}]{\sphinxcrossref{Exercises: Unit 5, How sure can we be about what is going on: Estimates and Intervals}}}

\item {} 
\sphinxAtStartPar
{\hyperref[\detokenize{exercises_unit_6::doc}]{\sphinxcrossref{Exercises}}}

\item {} 
\sphinxAtStartPar
{\hyperref[\detokenize{exercises_unit_7::doc}]{\sphinxcrossref{Exercises}}}

\item {} 
\sphinxAtStartPar
{\hyperref[\detokenize{exercises_unit_8::doc}]{\sphinxcrossref{Exercises}}}

\end{itemize}

\sphinxstepscope


\chapter{Exercises: Unit 1, Categorical data and proportions}
\label{\detokenize{exercises_unit_1:exercises-unit-1-categorical-data-and-proportions}}\label{\detokenize{exercises_unit_1::doc}}

\section{Pen and paper exercises}
\label{\detokenize{exercises_unit_1:pen-and-paper-exercises}}

\subsection{Exercise 1: Introduce yourself}
\label{\detokenize{exercises_unit_1:exercise-1-introduce-yourself}}
\sphinxAtStartPar
My name is Martin Summer. I live in Vienna, the capital of Austria in Europe with my wife Gabriele and my two sons Paul, 17 and Peter, 15. I am an economist by training and profession and I am currently heading a research department at Oesterreichische Nationalbank, which is the central bank of Austria and together with the European Central Bank part of the Europan system of central banks. I came to statistics and data analysis through my work, where I use these tools to do research, advise policy and more generally trying to understand what is going on in the world more broadly.

\sphinxAtStartPar
I have admired Jesuit Worldwide Learning as a project and institution for many years. This at some stage led to a cooperation when JWL asked and entrusted me to develop a statistics course for their program. The course you are taking now is the first result in this attempt and I hope you will enjy it and find it interesting and useful. I love teaching and I love statistics and this is a key reason why I decided to take on this honorable assignemt. Of course I also want to support you in your desire to learn by giving the best I can from my experience. I also want to support my Jesuit friends who founded and maintain this wonderful project, which I truely admire and love.

\sphinxAtStartPar
Statistics impacts my life in many ways: As a source of wonder and a well of knowledge about the world, as an amazing field of insight and inspiration, to which many of the greatest minds in human history, both theoretically and practically have contributed. It is just wonderful to partcipate in this human project of knowledge. Statistics also impacts my professional life, where it is one tool of the trade in economic analysis and policy advice. At the moment I am together with colleagues at OeNB are engaged in a research project where we uses statistical methods to find out whether people in Austria would be willing to adopt a digital form of the Euro, our money which we currently just as banknotes or through accounts at commercial banks. Isn’t it almost like magic that statistics allows us to glimpse not only into an unknown future but also into potential behavior of people in using a payment method \sphinxhyphen{} the digital Euro \sphinxhyphen{} that does not yet exist at all but as an idea?


\subsection{Exercise 2: Computing the percentage decrease in infant mortality rates of 8 European countries between 1860 and 2020}
\label{\detokenize{exercises_unit_1:exercise-2-computing-the-percentage-decrease-in-infant-mortality-rates-of-8-european-countries-between-1860-and-2020}}
\sphinxAtStartPar
Here are the numbers we discussed in the text in a table. I named the variables for the infant mortality rate in this exercise MR\sphinxhyphen{}1860 for  the infant mortality rate in 1860 and MR\sphinxhyphen{}2020 for the corresponding mortality rate in 2020.


\begin{savenotes}\sphinxattablestart
\centering
\begin{tabulary}{\linewidth}[t]{|T|T|T|T|}
\hline
\sphinxstyletheadfamily 
\sphinxAtStartPar
Country
&\sphinxstyletheadfamily 
\sphinxAtStartPar
Continent
&\sphinxstyletheadfamily 
\sphinxAtStartPar
MR\sphinxhyphen{}1860
&\sphinxstyletheadfamily 
\sphinxAtStartPar
MR\sphinxhyphen{}2020
\\
\hline
\sphinxAtStartPar
Austria
&
\sphinxAtStartPar
Europe
&
\sphinxAtStartPar
0.237
&
\sphinxAtStartPar
0.0030
\\
\hline
\sphinxAtStartPar
Belgium
&
\sphinxAtStartPar
Europe
&
\sphinxAtStartPar
0.139
&
\sphinxAtStartPar
0.0034
\\
\hline
\sphinxAtStartPar
Denmark
&
\sphinxAtStartPar
Europe
&
\sphinxAtStartPar
0.136
&
\sphinxAtStartPar
0.0031
\\
\hline
\sphinxAtStartPar
France
&
\sphinxAtStartPar
Europe
&
\sphinxAtStartPar
0.150
&
\sphinxAtStartPar
0.0034
\\
\hline
\sphinxAtStartPar
Germany
&
\sphinxAtStartPar
Europe
&
\sphinxAtStartPar
0.260
&
\sphinxAtStartPar
0.0031
\\
\hline
\sphinxAtStartPar
Norway
&
\sphinxAtStartPar
Europe
&
\sphinxAtStartPar
0.102
&
\sphinxAtStartPar
0.0018
\\
\hline
\sphinxAtStartPar
Spain
&
\sphinxAtStartPar
Europe
&
\sphinxAtStartPar
0.174
&
\sphinxAtStartPar
0.0027
\\
\hline
\sphinxAtStartPar
Sweden
&
\sphinxAtStartPar
Europe
&
\sphinxAtStartPar
0.124
&
\sphinxAtStartPar
0.0021
\\
\hline
\end{tabulary}
\par
\sphinxattableend\end{savenotes}

\sphinxAtStartPar
In case you wonder how I did this table in the Jupyter notebook, look at the cell input and study the code. For
textcells in Jupyter you can use a markup language called Markdown. Mardown will allow you to format text by using certain markup symbols that the Jupyter notebook can interpret and process. Here is a link to a cheat sheet summarizing the most important markup instructions: \sphinxurl{https://www.markdownguide.org/cheat-sheet/}

\sphinxAtStartPar
I encourage you as facilitators to familiarize yourself with markup and teach it to the students you work with when working with the Jupyter Notebooks.

\sphinxAtStartPar
The infant mortality rate in Austria was \(0.237\) in 1860 and was \(0.0030\) 160 years later in 2020. The percentage decrease over this period is computed as:
\textbackslash{}begin\{equation*\}
\textbackslash{}frac\{0.237 \sphinxhyphen{} 0.003\}\{0.237\}
\textbackslash{}end\{equation*\}

\sphinxAtStartPar
you can compute this by hand or by calculator or by R. Let me use R for the sake of the example

\begin{sphinxuseclass}{cell}\begin{sphinxVerbatimInput}

\begin{sphinxuseclass}{cell_input}
\begin{sphinxVerbatim}[commandchars=\\\{\}]
\PYG{p}{(}\PYG{p}{(}\PYG{l+m}{0.237}\PYG{+w}{ }\PYG{o}{\PYGZhy{}}\PYG{+w}{ }\PYG{l+m}{0.003}\PYG{p}{)}\PYG{o}{/}\PYG{l+m}{0.237}\PYG{p}{)}\PYG{o}{*}\PYG{l+m}{100}
\end{sphinxVerbatim}

\end{sphinxuseclass}\end{sphinxVerbatimInput}
\begin{sphinxVerbatimOutput}

\begin{sphinxuseclass}{cell_output}\begin{equation*}
\begin{split}98.7341772151899\end{split}
\end{equation*}
\end{sphinxuseclass}\end{sphinxVerbatimOutput}

\end{sphinxuseclass}
\sphinxAtStartPar
This is a reduction of roughly 99 \% percent. One could also say that the mortality rate decreased by a factor of 78. If you express the change as a relative change in percent you see how enormous the reduction was over a period of 160 years. This is next to a miracle, after centuries of extremely high rates.

\sphinxAtStartPar
In case you are uncertain how to compute percentage changes, let us discuss some examples:

\sphinxAtStartPar
Example 1: Suppose you have worked in a Supermarket for a wage of 10 (in some monetary unit) and your pay is increased to 12. What is your pay increase in percent? The way to compute the percentage increase is:
\textbackslash{}begin\{equation*\}
\textbackslash{}frac\{(12\sphinxhyphen{}10)\}\{10\}
\textbackslash{}end\{equation*\}

\sphinxAtStartPar
which is \(0.20\). Take this times 100 and you have the increase in percent, which is \(20\) \%.

\sphinxAtStartPar
Example 2: The staff of a company decreased from 40 to 29. What was the percentage decrease in its labor force?
\textbackslash{}begin\{equation*\}
\textbackslash{}frac\{(29 \sphinxhyphen{} 40)\}\{29\}
\textbackslash{}end\{equation*\}

\sphinxAtStartPar
which is \(0.275\). Take this by 100 and you have  \(27.5\) \%.

\sphinxAtStartPar
Did you see the general pattern. When computing a percentage change, the formula you apply is

\sphinxAtStartPar
\textbackslash{}begin\{equation*\}
\textbackslash{}frac\{(final \sphinxhyphen{} initial)\}\{initial\} \textbackslash{}times 100
\textbackslash{}end\{equation*\}

\sphinxAtStartPar
This is the formula for a percentage change. When this change is positive, we say naturally that we have an increase, when it is negative we say we have a decrease and fold the minus sign into the workd decrease. So in the second example, we would say that the labor force of the company decreased by\(27.5\) \%.

\sphinxAtStartPar
When the percentage change is 100 it means that something doubled, when it is 1000 it means that something has increased 10 fold, or as in the case of infant mortality in Austria between 1860 and 2020 has decreased 78 fold.

\sphinxAtStartPar
We can also do the computations with R based on what we have learned in the course so far. Let’s first store the values in two objects, which we call ‘mr\_1860’ and ‘mr\_2020’.

\begin{sphinxuseclass}{cell}\begin{sphinxVerbatimInput}

\begin{sphinxuseclass}{cell_input}
\begin{sphinxVerbatim}[commandchars=\\\{\}]
\PYG{n}{mr\PYGZus{}1860}\PYG{+w}{  }\PYG{o}{\PYGZlt{}\PYGZhy{}}\PYG{+w}{ }\PYG{n+nf}{c}\PYG{p}{(}\PYG{l+m}{0.237}\PYG{p}{,}\PYG{+w}{ }\PYG{l+m}{0.139}\PYG{p}{,}\PYG{+w}{ }\PYG{l+m}{0.136}\PYG{p}{,}\PYG{+w}{ }\PYG{l+m}{0.15}\PYG{p}{,}\PYG{+w}{ }\PYG{l+m}{0.26}\PYG{p}{,}\PYG{+w}{ }\PYG{l+m}{0.102}\PYG{p}{,}\PYG{+w}{ }\PYG{l+m}{0.174}\PYG{p}{,}\PYG{+w}{ }\PYG{l+m}{0.124}\PYG{p}{)}
\PYG{n}{mr\PYGZus{}2020}\PYG{+w}{  }\PYG{o}{\PYGZlt{}\PYGZhy{}}\PYG{+w}{ }\PYG{n+nf}{c}\PYG{p}{(}\PYG{l+m}{0.003}\PYG{p}{,}\PYG{+w}{ }\PYG{l+m}{0.0034}\PYG{p}{,}\PYG{+w}{ }\PYG{l+m}{0.0031}\PYG{p}{,}\PYG{+w}{ }\PYG{l+m}{0.0034}\PYG{p}{,}\PYG{+w}{ }\PYG{l+m}{0.0031}\PYG{p}{,}\PYG{+w}{ }\PYG{l+m}{0.0018}\PYG{p}{,}\PYG{+w}{ }\PYG{l+m}{0.0027}\PYG{p}{,}\PYG{+w}{ }\PYG{l+m}{0.0021}\PYG{p}{)}
\end{sphinxVerbatim}

\end{sphinxuseclass}\end{sphinxVerbatimInput}

\end{sphinxuseclass}
\sphinxAtStartPar
Now we use the fact that R can compute component wise and apply the formula for the percentage change and round to 2 digits:

\begin{sphinxuseclass}{cell}\begin{sphinxVerbatimInput}

\begin{sphinxuseclass}{cell_input}
\begin{sphinxVerbatim}[commandchars=\\\{\}]
\PYG{n+nf}{round}\PYG{p}{(}\PYG{p}{(}\PYG{p}{(}\PYG{n}{mr\PYGZus{}1860}\PYG{+w}{ }\PYG{o}{\PYGZhy{}}\PYG{+w}{ }\PYG{n}{mr\PYGZus{}2020}\PYG{p}{)}\PYG{o}{/}\PYG{n}{mr\PYGZus{}1860}\PYG{p}{)}\PYG{o}{*}\PYG{l+m}{100}\PYG{p}{,}\PYG{+w}{ }\PYG{l+m}{2}\PYG{p}{)}
\end{sphinxVerbatim}

\end{sphinxuseclass}\end{sphinxVerbatimInput}
\begin{sphinxVerbatimOutput}

\begin{sphinxuseclass}{cell_output}\begin{equation*}
\begin{split}\begin{enumerate*}
\item 98.73
\item 97.55
\item 97.72
\item 97.73
\item 98.81
\item 98.24
\item 98.45
\item 98.31
\end{enumerate*}\end{split}
\end{equation*}
\end{sphinxuseclass}\end{sphinxVerbatimOutput}

\end{sphinxuseclass}
\sphinxAtStartPar
So this means we have:


\begin{savenotes}\sphinxattablestart
\centering
\begin{tabulary}{\linewidth}[t]{|T|T|}
\hline
\sphinxstyletheadfamily 
\sphinxAtStartPar
Country
&\sphinxstyletheadfamily 
\sphinxAtStartPar
Percentage decrease in infant mortality from 1860 \sphinxhyphen{} 2020
\\
\hline
\sphinxAtStartPar
Austria
&
\sphinxAtStartPar
98.73 \%
\\
\hline
\sphinxAtStartPar
Belgium
&
\sphinxAtStartPar
97.55 \%
\\
\hline
\sphinxAtStartPar
Denmark
&
\sphinxAtStartPar
97.72 \%
\\
\hline
\sphinxAtStartPar
France
&
\sphinxAtStartPar
97.73 \%
\\
\hline
\sphinxAtStartPar
Germany
&
\sphinxAtStartPar
98.81 \%
\\
\hline
\sphinxAtStartPar
Norway
&
\sphinxAtStartPar
98.24 \%
\\
\hline
\sphinxAtStartPar
Spain
&
\sphinxAtStartPar
98.45 \%
\\
\hline
\sphinxAtStartPar
Sweden
&
\sphinxAtStartPar
98.31 \%
\\
\hline
\end{tabulary}
\par
\sphinxattableend\end{savenotes}

\sphinxAtStartPar
Usually you dp not encounter percentage number like this, because the difference in the mortality rates are so enormous. Suppose you were a bit unsure whether this could be indeed correct. Here is a way to check your calculation. A decrease by, for example of 98.73 \%, is a decrease by 0.9873 in decimal notation. So if your rate was initially 0.237 and you subtract \(0.237 \times 0.9873\) you get 0.0030099 as you should if rounded to the last four digits.

\sphinxAtStartPar
I encourage you as facilitators to come back to this simple example if appropriate in the coming two units, once students have learned and practiced more with R. Here is a way how you could retrieve the necessary data to do the computation with R

\begin{sphinxuseclass}{cell}\begin{sphinxVerbatimInput}

\begin{sphinxuseclass}{cell_input}
\begin{sphinxVerbatim}[commandchars=\\\{\}]
\PYG{n+nf}{library}\PYG{p}{(}\PYG{n}{JWL}\PYG{p}{)}

\PYG{c+c1}{\PYGZsh{} load infant mortality data}

\PYG{n}{infant\PYGZus{}mortality}\PYG{+w}{ }\PYG{o}{\PYGZlt{}\PYGZhy{}}\PYG{+w}{ }\PYG{n}{infant\PYGZus{}mortality\PYGZus{}data}

\PYG{c+c1}{\PYGZsh{} make table for the year 1860 and 2020 and throw away all countries where there are no observations}
\PYG{c+c1}{\PYGZsh{} and select the variables Country, Continent, Mortality}

\PYG{n}{my\PYGZus{}countries}\PYG{+w}{  }\PYG{o}{\PYGZlt{}\PYGZhy{}}\PYG{+w}{ }\PYG{n+nf}{c}\PYG{p}{(}\PYG{l+s}{\PYGZdq{}}\PYG{l+s}{Austria\PYGZdq{}}\PYG{p}{,}\PYG{+w}{ }\PYG{l+s}{\PYGZdq{}}\PYG{l+s}{Belgium\PYGZdq{}}\PYG{p}{,}\PYG{+w}{ }\PYG{l+s}{\PYGZdq{}}\PYG{l+s}{Denmark\PYGZdq{}}\PYG{p}{,}\PYG{+w}{ }\PYG{l+s}{\PYGZdq{}}\PYG{l+s}{France\PYGZdq{}}\PYG{p}{,}\PYG{+w}{ }\PYG{l+s}{\PYGZdq{}}\PYG{l+s}{Germany\PYGZdq{}}\PYG{p}{,}\PYG{+w}{ }\PYG{l+s}{\PYGZdq{}}\PYG{l+s}{Norway\PYGZdq{}}\PYG{p}{,}\PYG{+w}{ }\PYG{l+s}{\PYGZdq{}}\PYG{l+s}{Spain\PYGZdq{}}\PYG{p}{,}\PYG{+w}{ }\PYG{l+s}{\PYGZdq{}}\PYG{l+s}{Sweden\PYGZdq{}}\PYG{p}{)}

\PYG{n}{dat\PYGZus{}1860}\PYG{+w}{ }\PYG{o}{\PYGZlt{}\PYGZhy{}}\PYG{+w}{ }\PYG{n}{infant\PYGZus{}mortality}\PYG{p}{[}\PYG{n}{infant\PYGZus{}mortality}\PYG{o}{\PYGZdl{}}\PYG{n}{Country}\PYG{+w}{ }\PYG{o}{\PYGZpc{}in\PYGZpc{}}\PYG{+w}{ }\PYG{n}{my\PYGZus{}countries}\PYG{+w}{ }\PYG{o}{\PYGZam{}}\PYG{+w}{ }\PYG{n}{infant\PYGZus{}mortality}\PYG{o}{\PYGZdl{}}\PYG{n}{Year}\PYG{+w}{ }\PYG{o}{==}\PYG{+w}{ }\PYG{l+m}{1860}\PYG{p}{,}\PYG{+w}{ }\PYG{p}{]}
\PYG{n}{dat\PYGZus{}2020}\PYG{+w}{ }\PYG{o}{\PYGZlt{}\PYGZhy{}}\PYG{+w}{ }\PYG{n}{infant\PYGZus{}mortality}\PYG{p}{[}\PYG{n}{infant\PYGZus{}mortality}\PYG{o}{\PYGZdl{}}\PYG{n}{Country}\PYG{+w}{ }\PYG{o}{\PYGZpc{}in\PYGZpc{}}\PYG{+w}{ }\PYG{n}{my\PYGZus{}countries}\PYG{+w}{ }\PYG{o}{\PYGZam{}}\PYG{+w}{ }\PYG{n}{infant\PYGZus{}mortality}\PYG{o}{\PYGZdl{}}\PYG{n}{Year}\PYG{+w}{ }\PYG{o}{==}\PYG{+w}{ }\PYG{l+m}{2020}\PYG{p}{,}\PYG{+w}{ }\PYG{p}{]}

\PYG{n}{rate\PYGZus{}1860}\PYG{+w}{ }\PYG{o}{\PYGZlt{}\PYGZhy{}}\PYG{+w}{ }\PYG{n}{dat\PYGZus{}1860}\PYG{p}{[}\PYG{+w}{ }\PYG{p}{,}\PYG{+w}{ }\PYG{n+nf}{c}\PYG{p}{(}\PYG{l+s}{\PYGZdq{}}\PYG{l+s}{Country\PYGZdq{}}\PYG{p}{,}\PYG{+w}{ }\PYG{l+s}{\PYGZdq{}}\PYG{l+s}{Continent\PYGZdq{}}\PYG{p}{,}\PYG{+w}{ }\PYG{l+s}{\PYGZdq{}}\PYG{l+s}{Mortality\PYGZdq{}}\PYG{p}{)}\PYG{p}{]}\PYG{+w}{ }\PYG{o}{|}\PYG{o}{\PYGZgt{}}\PYG{+w}{ }\PYG{n+nf}{na.omit}\PYG{p}{(}\PYG{p}{)}
\PYG{n}{rate\PYGZus{}2020}\PYG{+w}{ }\PYG{o}{\PYGZlt{}\PYGZhy{}}\PYG{+w}{ }\PYG{n}{dat\PYGZus{}2020}\PYG{p}{[}\PYG{+w}{ }\PYG{p}{,}\PYG{+w}{ }\PYG{n+nf}{c}\PYG{p}{(}\PYG{l+s}{\PYGZdq{}}\PYG{l+s}{Mortality\PYGZdq{}}\PYG{p}{)}\PYG{p}{]}\PYG{+w}{ }\PYG{o}{|}\PYG{o}{\PYGZgt{}}\PYG{+w}{ }\PYG{n+nf}{na.omit}\PYG{p}{(}\PYG{p}{)}

\PYG{c+c1}{\PYGZsh{} make a table by binding the columns together in one dataframe}

\PYG{n}{tab}\PYG{+w}{  }\PYG{o}{\PYGZlt{}\PYGZhy{}}\PYG{+w}{ }\PYG{n+nf}{cbind}\PYG{p}{(}\PYG{n}{rate\PYGZus{}1860}\PYG{p}{,}\PYG{+w}{ }\PYG{n}{rate\PYGZus{}2020}\PYG{p}{)}
\PYG{n+nf}{names}\PYG{p}{(}\PYG{n}{tab}\PYG{p}{)}\PYG{+w}{  }\PYG{o}{\PYGZlt{}\PYGZhy{}}\PYG{+w}{ }\PYG{n+nf}{c}\PYG{p}{(}\PYG{l+s}{\PYGZdq{}}\PYG{l+s}{Contry\PYGZdq{}}\PYG{p}{,}\PYG{+w}{ }\PYG{l+s}{\PYGZdq{}}\PYG{l+s}{Continent\PYGZdq{}}\PYG{p}{,}\PYG{+w}{ }\PYG{l+s}{\PYGZdq{}}\PYG{l+s}{MR\PYGZhy{}1860\PYGZdq{}}\PYG{p}{,}\PYG{+w}{ }\PYG{l+s}{\PYGZdq{}}\PYG{l+s}{MR\PYGZhy{}2020\PYGZdq{}}\PYG{p}{)}
\PYG{n}{tab}
\end{sphinxVerbatim}

\end{sphinxuseclass}\end{sphinxVerbatimInput}
\begin{sphinxVerbatimOutput}

\begin{sphinxuseclass}{cell_output}\begin{equation*}
\begin{split}A data.frame: 8 × 4
\begin{tabular}{r|llll}
  & Contry & Continent & MR-1860 & MR-2020\\
  & <chr> & <fct> & <dbl> & <dbl>\\
\hline
	1970 & Austria & Europe & 0.237 & 0.0030\\
	3260 & Belgium & Europe & 0.139 & 0.0034\\
	7810 & Denmark & Europe & 0.136 & 0.0031\\
	10396 & France  & Europe & 0.150 & 0.0034\\
	11040 & Germany & Europe & 0.260 & 0.0031\\
	22590 & Norway  & Europe & 0.102 & 0.0018\\
	28200 & Spain   & Europe & 0.174 & 0.0027\\
	28829 & Sweden  & Europe & 0.124 & 0.0021\\
\end{tabular}\end{split}
\end{equation*}
\end{sphinxuseclass}\end{sphinxVerbatimOutput}

\end{sphinxuseclass}

\subsection{Exercise 3: Framing}
\label{\detokenize{exercises_unit_1:exercise-3-framing}}
\sphinxAtStartPar
One example could for instance be:

\sphinxAtStartPar
While looking for a disinfectant, you choose a product that kills 95\% of all the germs (positive frame), over one which claims that 5\% of the germs will survive (negative frame).

\sphinxAtStartPar
or:

\sphinxAtStartPar
You must decide between two elective courses in your studies at JWL. You are determined to maintain your
good grades, and you seek more information about both courses. One learning faclitator tells you thatin course one 20\% of the students achieve the best possible grade (positive frame) while in the other course 80\% fail to get the best possible grade (negative frame). You choose the former over the latter.

\sphinxAtStartPar
or:

\sphinxAtStartPar
Saying “Save 100 Euro a year by using energy\sphinxhyphen{}efficient appliances” (gain frame) rather than “Lose 100 Euro a year by not using energy\sphinxhyphen{}efficient appliances” (loss frame). The gain\sphinxhyphen{}framed message may motivate more people to act.


\subsection{Exercise 4: Communicating counts and proportions}
\label{\detokenize{exercises_unit_1:exercise-4-communicating-counts-and-proportions}}
\sphinxAtStartPar
One way to think about this might be as follows: The statment that 99 \% of young Londoner’s do not comit serious youth violence is equivalent to the information that there is 1 \% who do commit serious violence. Now you might think about this as an actual crowd of people. Note that order of magnitude is decisive here not the very exact numbers. The population of London is around 9 Million. Of these 9 Million about 1 million is aged between 15 and 25. 1 \% of 1 million is 10000. Would you feel comfortable riding on the London underground, knwoing that 10000 seriously violent youngsters are traveling in this system at the same time? Now we have used two twists here: First we have turned a positive into a negative frame and then tunred a percentage into an actual number of people.

\sphinxAtStartPar
To provide information impartially it is advisable to think of both negative and positive frames.

\sphinxAtStartPar
This is an exercise that would be very good for class discussion. The example of London, will require students to think about estimating orders of magnitude or think about how they could find out. For example: Where could I find out the number of inhabitants of London? (easy), how many young people are living there (what is young? compare to the introductory example of what is a tree?) etc.


\section{Computer exercises}
\label{\detokenize{exercises_unit_1:computer-exercises}}

\section{Exercise 1: Transforming percentages into rates}
\label{\detokenize{exercises_unit_1:exercise-1-transforming-percentages-into-rates}}
\sphinxAtStartPar
This is the table of infant mortality data in percent.


\begin{savenotes}\sphinxattablestart
\centering
\begin{tabulary}{\linewidth}[t]{|T|T|T|T|}
\hline
\sphinxstyletheadfamily 
\sphinxAtStartPar
Country
&\sphinxstyletheadfamily 
\sphinxAtStartPar
Continent
&\sphinxstyletheadfamily 
\sphinxAtStartPar
MR\sphinxhyphen{}1860
&\sphinxstyletheadfamily 
\sphinxAtStartPar
MR\sphinxhyphen{}2020
\\
\hline
\sphinxAtStartPar
Austria
&
\sphinxAtStartPar
Europe
&
\sphinxAtStartPar
23.7 \%
&
\sphinxAtStartPar
0.30 \%
\\
\hline
\sphinxAtStartPar
Belgium
&
\sphinxAtStartPar
Europe
&
\sphinxAtStartPar
13.9 \%
&
\sphinxAtStartPar
0.34 \%
\\
\hline
\sphinxAtStartPar
Denmark
&
\sphinxAtStartPar
Europe
&
\sphinxAtStartPar
13.6 \%
&
\sphinxAtStartPar
0.31 \%
\\
\hline
\sphinxAtStartPar
France
&
\sphinxAtStartPar
Europe
&
\sphinxAtStartPar
15.0 \%
&
\sphinxAtStartPar
0.34 \%
\\
\hline
\sphinxAtStartPar
Germany
&
\sphinxAtStartPar
Europe
&
\sphinxAtStartPar
26.0 \%
&
\sphinxAtStartPar
0.31 \%
\\
\hline
\sphinxAtStartPar
Norway
&
\sphinxAtStartPar
Europe
&
\sphinxAtStartPar
10.2 \%
&
\sphinxAtStartPar
0.18 \%
\\
\hline
\sphinxAtStartPar
Spain
&
\sphinxAtStartPar
Europe
&
\sphinxAtStartPar
17.4 \%
&
\sphinxAtStartPar
0.27 \%
\\
\hline
\sphinxAtStartPar
Sweden
&
\sphinxAtStartPar
Europe
&
\sphinxAtStartPar
12.4 \%
&
\sphinxAtStartPar
0.21 \%
\\
\hline
\end{tabulary}
\par
\sphinxattableend\end{savenotes}

\sphinxAtStartPar
Let’s start with saving the percentages in an R object in decimal notation:

\begin{sphinxuseclass}{cell}\begin{sphinxVerbatimInput}

\begin{sphinxuseclass}{cell_input}
\begin{sphinxVerbatim}[commandchars=\\\{\}]
\PYG{n}{mr\PYGZus{}1860}\PYG{+w}{  }\PYG{o}{\PYGZlt{}\PYGZhy{}}\PYG{+w}{ }\PYG{n+nf}{c}\PYG{p}{(}\PYG{l+m}{0.003}\PYG{p}{,}\PYG{+w}{ }\PYG{l+m}{0.0034}\PYG{p}{,}\PYG{+w}{ }\PYG{l+m}{0.0031}\PYG{p}{,}\PYG{+w}{ }\PYG{l+m}{0.0034}\PYG{p}{,}\PYG{+w}{ }\PYG{l+m}{0.0031}\PYG{p}{,}\PYG{+w}{ }\PYG{l+m}{0.0018}\PYG{p}{,}\PYG{+w}{ }\PYG{l+m}{0.0027}\PYG{p}{,}\PYG{+w}{ }\PYG{l+m}{0.0021}\PYG{p}{)}
\PYG{n}{mr\PYGZus{}2020}\PYG{+w}{  }\PYG{o}{\PYGZlt{}\PYGZhy{}}\PYG{+w}{ }\PYG{n+nf}{c}\PYG{p}{(}\PYG{l+m}{0.237}\PYG{p}{,}\PYG{+w}{ }\PYG{l+m}{0.139}\PYG{p}{,}\PYG{+w}{ }\PYG{l+m}{0.136}\PYG{p}{,}\PYG{+w}{ }\PYG{l+m}{0.15}\PYG{p}{,}\PYG{+w}{ }\PYG{l+m}{0.26}\PYG{p}{,}\PYG{+w}{ }\PYG{l+m}{0.102}\PYG{p}{,}\PYG{+w}{ }\PYG{l+m}{0.174}\PYG{p}{,}\PYG{+w}{ }\PYG{l+m}{0.124}\PYG{p}{)}
\end{sphinxVerbatim}

\end{sphinxuseclass}\end{sphinxVerbatimInput}

\end{sphinxuseclass}
\sphinxAtStartPar
Now we multiply every entry with 1000 and this gives us the number of infant death per 1000. One nice feature of R is that it can apply a mathematcal operation to a whole object. So if you multiply the object by 1000 this multiplication is applied to every component in the vector. Let’s do this and save the results in a new object
called rates\_1860 and rates\_2020. Then display the results and write them into a new table.

\begin{sphinxuseclass}{cell}\begin{sphinxVerbatimInput}

\begin{sphinxuseclass}{cell_input}
\begin{sphinxVerbatim}[commandchars=\\\{\}]
\PYG{n}{rates\PYGZus{}1860}\PYG{+w}{  }\PYG{o}{\PYGZlt{}\PYGZhy{}}\PYG{+w}{ }\PYG{n}{mr\PYGZus{}1860}\PYG{o}{*}\PYG{l+m}{1000}
\PYG{n}{rates\PYGZus{}2020}\PYG{+w}{  }\PYG{o}{\PYGZlt{}\PYGZhy{}}\PYG{+w}{ }\PYG{n}{mr\PYGZus{}2020}\PYG{o}{*}\PYG{l+m}{1000}

\PYG{n}{rates\PYGZus{}1860}
\PYG{n}{rates\PYGZus{}2020}
\end{sphinxVerbatim}

\end{sphinxuseclass}\end{sphinxVerbatimInput}
\begin{sphinxVerbatimOutput}

\begin{sphinxuseclass}{cell_output}\begin{equation*}
\begin{split}\begin{enumerate*}
\item 3
\item 3.4
\item 3.1
\item 3.4
\item 3.1
\item 1.8
\item 2.7
\item 2.1
\end{enumerate*}\end{split}
\end{equation*}\begin{equation*}
\begin{split}\begin{enumerate*}
\item 237
\item 139
\item 136
\item 150
\item 260
\item 102
\item 174
\item 124
\end{enumerate*}\end{split}
\end{equation*}
\end{sphinxuseclass}\end{sphinxVerbatimOutput}

\end{sphinxuseclass}
\sphinxAtStartPar
This is the table of infant mortality data per 1000 of life births.


\begin{savenotes}\sphinxattablestart
\centering
\begin{tabulary}{\linewidth}[t]{|T|T|T|T|}
\hline
\sphinxstyletheadfamily 
\sphinxAtStartPar
Country
&\sphinxstyletheadfamily 
\sphinxAtStartPar
Continent
&\sphinxstyletheadfamily 
\sphinxAtStartPar
MR\sphinxhyphen{}1860
&\sphinxstyletheadfamily 
\sphinxAtStartPar
MR\sphinxhyphen{}2020
\\
\hline
\sphinxAtStartPar
Austria
&
\sphinxAtStartPar
Europe
&
\sphinxAtStartPar
237
&
\sphinxAtStartPar
3.0
\\
\hline
\sphinxAtStartPar
Belgium
&
\sphinxAtStartPar
Europe
&
\sphinxAtStartPar
139
&
\sphinxAtStartPar
3.4
\\
\hline
\sphinxAtStartPar
Denmark
&
\sphinxAtStartPar
Europe
&
\sphinxAtStartPar
136
&
\sphinxAtStartPar
3.1
\\
\hline
\sphinxAtStartPar
France
&
\sphinxAtStartPar
Europe
&
\sphinxAtStartPar
150
&
\sphinxAtStartPar
3.4
\\
\hline
\sphinxAtStartPar
Germany
&
\sphinxAtStartPar
Europe
&
\sphinxAtStartPar
260
&
\sphinxAtStartPar
3.1
\\
\hline
\sphinxAtStartPar
Norway
&
\sphinxAtStartPar
Europe
&
\sphinxAtStartPar
102
&
\sphinxAtStartPar
1.8
\\
\hline
\sphinxAtStartPar
Spain
&
\sphinxAtStartPar
Europe
&
\sphinxAtStartPar
174
&
\sphinxAtStartPar
2.7
\\
\hline
\sphinxAtStartPar
Sweden
&
\sphinxAtStartPar
Europe
&
\sphinxAtStartPar
124
&
\sphinxAtStartPar
2.1
\\
\hline
\end{tabulary}
\par
\sphinxattableend\end{savenotes}


\subsection{Exercise 2: Your first bar plot}
\label{\detokenize{exercises_unit_1:exercise-2-your-first-bar-plot}}
\sphinxAtStartPar
Let’s redo the barplot for motrality rates with the 2020 numbers. Let us store the rates in an R object first:

\begin{sphinxuseclass}{cell}\begin{sphinxVerbatimInput}

\begin{sphinxuseclass}{cell_input}
\begin{sphinxVerbatim}[commandchars=\\\{\}]
\PYG{n}{mr\PYGZus{}2020}\PYG{+w}{  }\PYG{o}{\PYGZlt{}\PYGZhy{}}\PYG{+w}{ }\PYG{n+nf}{c}\PYG{p}{(}\PYG{l+m}{0.237}\PYG{p}{,}\PYG{+w}{ }\PYG{l+m}{0.139}\PYG{p}{,}\PYG{+w}{ }\PYG{l+m}{0.136}\PYG{p}{,}\PYG{+w}{ }\PYG{l+m}{0.15}\PYG{p}{,}\PYG{+w}{ }\PYG{l+m}{0.26}\PYG{p}{,}\PYG{+w}{ }\PYG{l+m}{0.102}\PYG{p}{,}\PYG{+w}{ }\PYG{l+m}{0.174}\PYG{p}{,}\PYG{+w}{ }\PYG{l+m}{0.124}\PYG{p}{)}
\end{sphinxVerbatim}

\end{sphinxuseclass}\end{sphinxVerbatimInput}

\end{sphinxuseclass}
\sphinxAtStartPar
Now create a second object with the country names:

\begin{sphinxuseclass}{cell}\begin{sphinxVerbatimInput}

\begin{sphinxuseclass}{cell_input}
\begin{sphinxVerbatim}[commandchars=\\\{\}]
\PYG{n}{cntr}\PYG{+w}{  }\PYG{o}{\PYGZlt{}\PYGZhy{}}\PYG{+w}{ }\PYG{n+nf}{c}\PYG{p}{(}\PYG{l+s}{\PYGZdq{}}\PYG{l+s}{Austria\PYGZdq{}}\PYG{p}{,}\PYG{+w}{ }\PYG{l+s}{\PYGZdq{}}\PYG{l+s}{Belgium\PYGZdq{}}\PYG{p}{,}\PYG{+w}{ }\PYG{l+s}{\PYGZdq{}}\PYG{l+s}{Denmark\PYGZdq{}}\PYG{p}{,}\PYG{+w}{ }\PYG{l+s}{\PYGZdq{}}\PYG{l+s}{France\PYGZdq{}}\PYG{p}{,}\PYG{+w}{ }\PYG{l+s}{\PYGZdq{}}\PYG{l+s}{Germany\PYGZdq{}}\PYG{p}{,}\PYG{+w}{ }\PYG{l+s}{\PYGZdq{}}\PYG{l+s}{Norway\PYGZdq{}}\PYG{p}{,}\PYG{+w}{ }\PYG{l+s}{\PYGZdq{}}\PYG{l+s}{Spain\PYGZdq{}}\PYG{p}{,}\PYG{+w}{ }\PYG{l+s}{\PYGZdq{}}\PYG{l+s}{Sweden\PYGZdq{}}\PYG{p}{)}
\end{sphinxVerbatim}

\end{sphinxuseclass}\end{sphinxVerbatimInput}

\end{sphinxuseclass}
\sphinxAtStartPar
Note that the order of countries in this needs to be congruent with the order of the numbers, so that they fit together. One advantage of the computer is that such steps can be automated when you know more R. Such manual procedures, which we use here are error prone and should be usually avoided. Here we use it because we do not yet know enough R to avoid this. Another approach to annotate the countries would be to use the international ISO2 codes for the countries, like this:

\begin{sphinxuseclass}{cell}\begin{sphinxVerbatimInput}

\begin{sphinxuseclass}{cell_input}
\begin{sphinxVerbatim}[commandchars=\\\{\}]
\PYG{n}{cntr\PYGZus{}iso}\PYG{+w}{  }\PYG{o}{\PYGZlt{}\PYGZhy{}}\PYG{+w}{ }\PYG{n+nf}{c}\PYG{p}{(}\PYG{l+s}{\PYGZdq{}}\PYG{l+s}{AT\PYGZdq{}}\PYG{p}{,}\PYG{+w}{ }\PYG{l+s}{\PYGZdq{}}\PYG{l+s}{BE\PYGZdq{}}\PYG{p}{,}\PYG{+w}{ }\PYG{l+s}{\PYGZdq{}}\PYG{l+s}{DN\PYGZdq{}}\PYG{p}{,}\PYG{+w}{ }\PYG{l+s}{\PYGZdq{}}\PYG{l+s}{FR\PYGZdq{}}\PYG{p}{,}\PYG{+w}{ }\PYG{l+s}{\PYGZdq{}}\PYG{l+s}{GE\PYGZdq{}}\PYG{p}{,}\PYG{+w}{ }\PYG{l+s}{\PYGZdq{}}\PYG{l+s}{NO\PYGZdq{}}\PYG{p}{,}\PYG{+w}{ }\PYG{l+s}{\PYGZdq{}}\PYG{l+s}{ES\PYGZdq{}}\PYG{p}{,}\PYG{+w}{ }\PYG{l+s}{\PYGZdq{}}\PYG{l+s}{SE\PYGZdq{}}\PYG{p}{)}
\end{sphinxVerbatim}

\end{sphinxuseclass}\end{sphinxVerbatimInput}

\end{sphinxuseclass}
\sphinxAtStartPar
Now for the barplot, proceed as in the lecture and give the barplot a title:

\begin{sphinxuseclass}{cell}\begin{sphinxVerbatimInput}

\begin{sphinxuseclass}{cell_input}
\begin{sphinxVerbatim}[commandchars=\\\{\}]
\PYG{n+nf}{barplot}\PYG{p}{(}\PYG{n}{mr\PYGZus{}2020}\PYG{p}{,}\PYG{+w}{ }\PYG{n}{names.arg}\PYG{+w}{ }\PYG{o}{=}\PYG{+w}{ }\PYG{n}{cntr\PYGZus{}iso}\PYG{p}{,}\PYG{+w}{ }\PYG{n}{main}\PYG{+w}{ }\PYG{o}{=}\PYG{+w}{ }\PYG{l+s}{\PYGZdq{}}\PYG{l+s}{Infant mortality rates in percent\PYGZdq{}}\PYG{p}{)}
\end{sphinxVerbatim}

\end{sphinxuseclass}\end{sphinxVerbatimInput}
\begin{sphinxVerbatimOutput}

\begin{sphinxuseclass}{cell_output}
\noindent\sphinxincludegraphics{{1ff41213b52b9f18e0accecfde6ce5b9d34d856984e09f620c6af08b7248703f}.png}

\end{sphinxuseclass}\end{sphinxVerbatimOutput}

\end{sphinxuseclass}
\sphinxAtStartPar
Or flip and use the full country names:

\begin{sphinxuseclass}{cell}\begin{sphinxVerbatimInput}

\begin{sphinxuseclass}{cell_input}
\begin{sphinxVerbatim}[commandchars=\\\{\}]
\PYG{n+nf}{barplot}\PYG{p}{(}\PYG{n}{mr\PYGZus{}2020}\PYG{p}{,}\PYG{+w}{ }\PYG{n}{names.arg}\PYG{+w}{ }\PYG{o}{=}\PYG{+w}{ }\PYG{n}{cntr}\PYG{p}{,}\PYG{+w}{ }\PYG{n}{horiz}\PYG{+w}{ }\PYG{o}{=}\PYG{+w}{ }\PYG{k+kc}{TRUE}\PYG{p}{,}\PYG{+w}{ }\PYG{n}{las}\PYG{+w}{ }\PYG{o}{=}\PYG{+w}{ }\PYG{l+m}{1}\PYG{p}{,}\PYG{+w}{ }\PYG{n}{main}\PYG{+w}{ }\PYG{o}{=}\PYG{+w}{ }\PYG{l+s}{\PYGZdq{}}\PYG{l+s}{Infant mortality rates in percent in 2020\PYGZdq{}}\PYG{p}{)}
\end{sphinxVerbatim}

\end{sphinxuseclass}\end{sphinxVerbatimInput}
\begin{sphinxVerbatimOutput}

\begin{sphinxuseclass}{cell_output}
\noindent\sphinxincludegraphics{{12c1868628887baa4b6adbf3e875d67e7a58b28c4271cc308ddda988266e192a}.png}

\end{sphinxuseclass}\end{sphinxVerbatimOutput}

\end{sphinxuseclass}
\sphinxAtStartPar
The names of Germany and Denmark are cut off a bit at the margin. An option to adjust plot margins in a boxplot to use the R function \sphinxcode{\sphinxupquote{par}} with the argument \sphinxcode{\sphinxupquote{mar}} for margin. \sphinxcode{\sphinxupquote{mar}}is a numerical vector of the form c(bottom, left, top, right) which gives the number of lines of margin to be specified on the four sides of the plot. The default is c(5, 4, 4, 2) + 0.1.

\begin{sphinxuseclass}{cell}\begin{sphinxVerbatimInput}

\begin{sphinxuseclass}{cell_input}
\begin{sphinxVerbatim}[commandchars=\\\{\}]
\PYG{n+nf}{par}\PYG{p}{(}\PYG{n}{mar}\PYG{o}{=}\PYG{n+nf}{c}\PYG{p}{(}\PYG{l+m}{5}\PYG{p}{,}\PYG{l+m}{6}\PYG{p}{,}\PYG{l+m}{4}\PYG{p}{,}\PYG{l+m}{4}\PYG{p}{)}\PYG{p}{)}
\PYG{n+nf}{barplot}\PYG{p}{(}\PYG{n}{mr\PYGZus{}2020}\PYG{p}{,}\PYG{+w}{ }\PYG{n}{names.arg}\PYG{+w}{ }\PYG{o}{=}\PYG{+w}{ }\PYG{n}{cntr}\PYG{p}{,}\PYG{+w}{ }\PYG{n}{horiz}\PYG{+w}{ }\PYG{o}{=}\PYG{+w}{ }\PYG{k+kc}{TRUE}\PYG{p}{,}\PYG{+w}{ }\PYG{n}{las}\PYG{+w}{ }\PYG{o}{=}\PYG{+w}{ }\PYG{l+m}{1}\PYG{p}{,}\PYG{+w}{ }\PYG{n}{main}\PYG{+w}{ }\PYG{o}{=}\PYG{+w}{ }\PYG{l+s}{\PYGZdq{}}\PYG{l+s}{Infant mortality rates in percent in 2020\PYGZdq{}}\PYG{p}{)}
\end{sphinxVerbatim}

\end{sphinxuseclass}\end{sphinxVerbatimInput}
\begin{sphinxVerbatimOutput}

\begin{sphinxuseclass}{cell_output}
\noindent\sphinxincludegraphics{{de1502ed056d83d0fbd1fb14443e288dc9cd6525ba6da632443093da8d7eb02e}.png}

\end{sphinxuseclass}\end{sphinxVerbatimOutput}

\end{sphinxuseclass}
\sphinxAtStartPar
Now you can see the full names. This is an advanced graphic option. You do not need to confuse the students with these details but it might be good to know an answer, in case somebody asks.

\sphinxAtStartPar
If there are students who are quickly at ease with using R and who find this too easy, point them to the R help function and encourage them to play with options to change the graph. Looking for help on \sphinxcode{\sphinxupquote{boxplot}} for example would need the function call:

\begin{sphinxuseclass}{cell}\begin{sphinxVerbatimInput}

\begin{sphinxuseclass}{cell_input}
\begin{sphinxVerbatim}[commandchars=\\\{\}]
\PYG{o}{?}\PYG{n}{boxplot}
\end{sphinxVerbatim}

\end{sphinxuseclass}\end{sphinxVerbatimInput}

\end{sphinxuseclass}
\sphinxAtStartPar
This displays the R help manual page for boxplot with all adjustment options.


\subsection{Exercise 3: Barchart on the causes of infant mortality}
\label{\detokenize{exercises_unit_1:exercise-3-barchart-on-the-causes-of-infant-mortality}}
\sphinxAtStartPar
Here again the point is to familiarize yourself with creating a barplot in R. Since we do by now know only very little R we do not yet know how to get the data we need for the plot automatically. Here the assumption is that we only have the table from the study material and we want to produce a boxplot on this base by manually entering the data. Of course the usual situation will be that data rerieval and entry will be done by the computer. We will soon do so ourselves. In general manual entry of data should be mimimized as a matter of principle. It is not only tedious and slow but also error prone.

\sphinxAtStartPar
Now, assuming, manual entry, we would first create two R objects, one character vector with the names of the causes and one with the share in all known causes and produce the boxplot from there. So let’s do this. The way you name your objects is by the way up to you. You choose the names you want and that work for you. Remember the naming rules, however. If you have forgotten these rules (Do not start a name by a . or \_ or a number etc.) look them up again and learn them by heart,

\begin{sphinxuseclass}{cell}\begin{sphinxVerbatimInput}

\begin{sphinxuseclass}{cell_input}
\begin{sphinxVerbatim}[commandchars=\\\{\}]
\PYG{n}{causes}\PYG{+w}{  }\PYG{o}{\PYGZlt{}\PYGZhy{}}\PYG{+w}{ }\PYG{n+nf}{c}\PYG{p}{(}
\PYG{l+s}{\PYGZdq{}}\PYG{l+s}{Preterm birth\PYGZdq{}}\PYG{p}{,}
\PYG{l+s}{\PYGZdq{}}\PYG{l+s}{Encephalopathy due to birth asphyxia and trauma\PYGZdq{}}\PYG{p}{,}
\PYG{l+s}{\PYGZdq{}}\PYG{l+s}{Lower respiratory infections\PYGZdq{}}\PYG{p}{,}
\PYG{l+s}{\PYGZdq{}}\PYG{l+s}{Birth defects\PYGZdq{}}\PYG{p}{,}
\PYG{l+s}{\PYGZdq{}}\PYG{l+s}{Diarrheal diseases\PYGZdq{}}\PYG{p}{,}
\PYG{l+s}{\PYGZdq{}}\PYG{l+s}{Heart anomalies\PYGZdq{}}\PYG{p}{,}
\PYG{l+s}{\PYGZdq{}}\PYG{l+s}{Malaria\PYGZdq{}}\PYG{p}{,}
\PYG{l+s}{\PYGZdq{}}\PYG{l+s}{Syphilis\PYGZdq{}}\PYG{p}{,}
\PYG{l+s}{\PYGZdq{}}\PYG{l+s}{Meningitis\PYGZdq{}}\PYG{p}{,}
\PYG{l+s}{\PYGZdq{}}\PYG{l+s}{Whooping cough\PYGZdq{}}\PYG{p}{,}
\PYG{l+s}{\PYGZdq{}}\PYG{l+s}{Nutritional deficiencies\PYGZdq{}}\PYG{p}{,}
\PYG{l+s}{\PYGZdq{}}\PYG{l+s}{Digestive anomalies\PYGZdq{}}\PYG{p}{,}
\PYG{l+s}{\PYGZdq{}}\PYG{l+s}{Sudden infant death syndrome\PYGZdq{}}\PYG{p}{,}
\PYG{l+s}{\PYGZdq{}}\PYG{l+s}{Tuberculosis\PYGZdq{}}\PYG{p}{,}
\PYG{l+s}{\PYGZdq{}}\PYG{l+s}{Measles\PYGZdq{}}\PYG{p}{,}
\PYG{l+s}{\PYGZdq{}}\PYG{l+s}{HIV/AIDS\PYGZdq{}}\PYG{p}{,}
\PYG{l+s}{\PYGZdq{}}\PYG{l+s}{Digestive diseases\PYGZdq{}}\PYG{p}{,}
\PYG{l+s}{\PYGZdq{}}\PYG{l+s}{Tetanus\PYGZdq{}}\PYG{p}{,}
\PYG{l+s}{\PYGZdq{}}\PYG{l+s}{Encephalitis\PYGZdq{}}\PYG{p}{,}
\PYG{l+s}{\PYGZdq{}}\PYG{l+s}{Acute hepatitis\PYGZdq{}}\PYG{p}{,}
\PYG{l+s}{\PYGZdq{}}\PYG{l+s}{Diabetes and kidney diseases\PYGZdq{}}
\PYG{p}{)}

\PYG{n}{shares}\PYG{+w}{  }\PYG{o}{\PYGZlt{}\PYGZhy{}}\PYG{+w}{ }\PYG{n+nf}{c}\PYG{p}{(}\PYG{l+m}{0.211}\PYG{p}{,}\PYG{+w}{ }\PYG{l+m}{0.180}\PYG{p}{,}\PYG{+w}{ }\PYG{l+m}{0.161}\PYG{p}{,}\PYG{+w}{ }\PYG{l+m}{0.128}\PYG{p}{,}\PYG{+w}{ }\PYG{l+m}{0.092}\PYG{p}{,}\PYG{+w}{ }\PYG{l+m}{0.048}\PYG{p}{,}\PYG{+w}{ }\PYG{l+m}{0.044}\PYG{p}{,}\PYG{+w}{ }\PYG{l+m}{0.026}\PYG{p}{,}\PYG{+w}{ }\PYG{l+m}{0.020}\PYG{p}{,}\PYG{+w}{ }\PYG{l+m}{0.016}\PYG{p}{,}\PYG{+w}{ }\PYG{l+m}{0.015}\PYG{p}{,}\PYG{+w}{ }\PYG{l+m}{0.013}\PYG{p}{,}\PYG{+w}{ }\PYG{l+m}{0.009}\PYG{p}{,}\PYG{+w}{ }\PYG{l+m}{0.008}\PYG{p}{,}\PYG{+w}{ }
\PYG{+w}{             }\PYG{l+m}{0.006}\PYG{p}{,}\PYG{+w}{ }\PYG{l+m}{0.006}\PYG{p}{,}\PYG{+w}{ }\PYG{l+m}{0.005}\PYG{p}{,}\PYG{+w}{ }\PYG{l+m}{0.005}\PYG{p}{,}\PYG{+w}{ }\PYG{l+m}{0.003}\PYG{p}{,}\PYG{+w}{ }\PYG{l+m}{0.002}\PYG{p}{,}\PYG{+w}{ }\PYG{l+m}{0.002}\PYG{p}{)}
\end{sphinxVerbatim}

\end{sphinxuseclass}\end{sphinxVerbatimInput}

\end{sphinxuseclass}
\sphinxAtStartPar
To reproduce the boxplot form the study materials we would need the following instruction for R, assuming that I give also a title to the plot, in my case “Causes of infant mortality”.

\begin{sphinxuseclass}{cell}\begin{sphinxVerbatimInput}

\begin{sphinxuseclass}{cell_input}
\begin{sphinxVerbatim}[commandchars=\\\{\}]
\PYG{n+nf}{barplot}\PYG{p}{(}\PYG{n}{shares}\PYG{p}{,}\PYG{+w}{ }\PYG{n}{names.arg}\PYG{+w}{ }\PYG{o}{=}\PYG{+w}{ }\PYG{n}{causes}\PYG{p}{,}\PYG{+w}{ }\PYG{n}{horiz}\PYG{+w}{ }\PYG{o}{=}\PYG{+w}{ }\PYG{k+kc}{TRUE}\PYG{p}{,}\PYG{+w}{ }\PYG{n}{las}\PYG{+w}{ }\PYG{o}{=}\PYG{+w}{ }\PYG{l+m}{1}\PYG{p}{,}\PYG{+w}{ }\PYG{n}{main}\PYG{+w}{ }\PYG{o}{=}\PYG{+w}{ }\PYG{l+s}{\PYGZdq{}}\PYG{l+s}{Causes of infant mortality\PYGZdq{}}\PYG{p}{)}
\end{sphinxVerbatim}

\end{sphinxuseclass}\end{sphinxVerbatimInput}
\begin{sphinxVerbatimOutput}

\begin{sphinxuseclass}{cell_output}
\noindent\sphinxincludegraphics{{80cc8e6bd7273a8377e0dc6267c256d02e09730c3d721ff4fb29f202eed4223f}.png}

\end{sphinxuseclass}\end{sphinxVerbatimOutput}

\end{sphinxuseclass}
\sphinxAtStartPar
Here we have again the problem that the names are too long for the default vaues of the boxplot boundaries. This might be an option to explain how these boundaries can be adjusted in R.

\begin{sphinxuseclass}{cell}\begin{sphinxVerbatimInput}

\begin{sphinxuseclass}{cell_input}
\begin{sphinxVerbatim}[commandchars=\\\{\}]
\PYG{n+nf}{par}\PYG{p}{(}\PYG{n}{mar}\PYG{o}{=}\PYG{n+nf}{c}\PYG{p}{(}\PYG{l+m}{5}\PYG{p}{,}\PYG{l+m}{20}\PYG{p}{,}\PYG{l+m}{4}\PYG{p}{,}\PYG{l+m}{4}\PYG{p}{)}\PYG{p}{)}
\PYG{n+nf}{barplot}\PYG{p}{(}\PYG{n}{shares}\PYG{p}{,}\PYG{+w}{ }\PYG{n}{names.arg}\PYG{+w}{ }\PYG{o}{=}\PYG{+w}{ }\PYG{n}{causes}\PYG{p}{,}\PYG{+w}{ }\PYG{n}{horiz}\PYG{+w}{ }\PYG{o}{=}\PYG{+w}{ }\PYG{k+kc}{TRUE}\PYG{p}{,}\PYG{+w}{ }\PYG{n}{las}\PYG{+w}{ }\PYG{o}{=}\PYG{+w}{ }\PYG{l+m}{1}\PYG{p}{,}\PYG{+w}{ }\PYG{n}{main}\PYG{+w}{ }\PYG{o}{=}\PYG{+w}{ }\PYG{l+s}{\PYGZdq{}}\PYG{l+s}{Causes of infant mortality\PYGZdq{}}\PYG{p}{)}
\end{sphinxVerbatim}

\end{sphinxuseclass}\end{sphinxVerbatimInput}
\begin{sphinxVerbatimOutput}

\begin{sphinxuseclass}{cell_output}
\noindent\sphinxincludegraphics{{f3dfa9f6308657f1d0e83b44eb5d502aeb640c267f6e9cca7f384e03d6024a1a}.png}

\end{sphinxuseclass}\end{sphinxVerbatimOutput}

\end{sphinxuseclass}

\section{Project: People count: The future of humanity in pictures and numbers}
\label{\detokenize{exercises_unit_1:project-people-count-the-future-of-humanity-in-pictures-and-numbers}}

\subsection{Counting people}
\label{\detokenize{exercises_unit_1:counting-people}}
\sphinxAtStartPar
This is the begin of a data project which will stretch all over the course. In this first part we
are applying what we have learned in this unit by counting people in Kenya. Like in all countries
in the world, national statistical offices describe people living in a country
by age, sex, race, education, and a number of other factors. Are these descriptions important?

\sphinxAtStartPar
In particular in this project we ask whether the description of a country
by age groups does help us better understand a
country? We are going to study the “shape” of Keynia based on the counts of people living in the country
by age.

\sphinxAtStartPar
As in real life national statistics we are goint to produce data visualizations that
display the data by way of barplots and other summaries. These descriptions
begin to unpack stories about people living in this
country. Behind these numbers and graphs are real people, their sisters and brothers and parents and nieces and
nephews. The stories revealed in these data help us better understand a particular country.

\sphinxAtStartPar
The following table shows the counts of women and men in Kenya in the year 2022. The data are made available through an international demographic database, which compiles the local infromation from national statistical authorities and provide the data publicly via a website hosted by the Census Bureau of the United States. The website is here and I have retrieved the data from this site.


\begin{savenotes}\sphinxattablestart
\centering
\begin{tabulary}{\linewidth}[t]{|T|T|T|}
\hline
\sphinxstyletheadfamily 
\sphinxAtStartPar
Age
&\sphinxstyletheadfamily 
\sphinxAtStartPar
Number of Females
&\sphinxstyletheadfamily 
\sphinxAtStartPar
Number of Males
\\
\hline
\sphinxAtStartPar
0\sphinxhyphen{}4
&
\sphinxAtStartPar
3487490
&
\sphinxAtStartPar
3531278
\\
\hline
\sphinxAtStartPar
5\sphinxhyphen{}9
&
\sphinxAtStartPar
3404421
&
\sphinxAtStartPar
3431166
\\
\hline
\sphinxAtStartPar
10\sphinxhyphen{}14
&
\sphinxAtStartPar
3444606
&
\sphinxAtStartPar
3473103
\\
\hline
\sphinxAtStartPar
15\sphinxhyphen{}19
&
\sphinxAtStartPar
3225971
&
\sphinxAtStartPar
3249738
\\
\hline
\sphinxAtStartPar
20\sphinxhyphen{}24
&
\sphinxAtStartPar
2656730
&
\sphinxAtStartPar
2664966
\\
\hline
\sphinxAtStartPar
30\sphinxhyphen{}34
&
\sphinxAtStartPar
1946994
&
\sphinxAtStartPar
1926219
\\
\hline
\sphinxAtStartPar
40\sphinxhyphen{}44
&
\sphinxAtStartPar
1606763
&
\sphinxAtStartPar
1610246
\\
\hline
\sphinxAtStartPar
45\sphinxhyphen{}49
&
\sphinxAtStartPar
1170869
&
\sphinxAtStartPar
1196519
\\
\hline
\sphinxAtStartPar
50\sphinxhyphen{}54
&
\sphinxAtStartPar
858998
&
\sphinxAtStartPar
890662
\\
\hline
\sphinxAtStartPar
55\sphinxhyphen{}59
&
\sphinxAtStartPar
672025
&
\sphinxAtStartPar
679943
\\
\hline
\sphinxAtStartPar
60\sphinxhyphen{}64
&
\sphinxAtStartPar
516769
&
\sphinxAtStartPar
488074
\\
\hline
\sphinxAtStartPar
65\sphinxhyphen{}69
&
\sphinxAtStartPar
387773
&
\sphinxAtStartPar
340546
\\
\hline
\sphinxAtStartPar
70\sphinxhyphen{}74
&
\sphinxAtStartPar
272320
&
\sphinxAtStartPar
229912
\\
\hline
\sphinxAtStartPar
75\sphinxhyphen{}79
&
\sphinxAtStartPar
168499
&
\sphinxAtStartPar
138357
\\
\hline
\sphinxAtStartPar
80\sphinxhyphen{}84
&
\sphinxAtStartPar
93724
&
\sphinxAtStartPar
73093
\\
\hline
\sphinxAtStartPar
85\sphinxhyphen{}89
&
\sphinxAtStartPar
39226
&
\sphinxAtStartPar
28473
\\
\hline
\sphinxAtStartPar
90\sphinxhyphen{}94
&
\sphinxAtStartPar
10461
&
\sphinxAtStartPar
6918
\\
\hline
\sphinxAtStartPar
95\sphinxhyphen{}99
&
\sphinxAtStartPar
1520
&
\sphinxAtStartPar
902
\\
\hline
\end{tabulary}
\par
\sphinxattableend\end{savenotes}
\begin{enumerate}
\sphinxsetlistlabels{\arabic}{enumi}{enumii}{}{.}%
\item {} 
\sphinxAtStartPar
Produce a barplot of the total population with R. Also produce a seperate barplot for man and women.

\end{enumerate}

\sphinxAtStartPar
Comment for learning facilitators: First store the data in appropriate R objects. It is perhaps best for this task to create an object for the man counts and the female counts seperately and then use R to create an object for the total. It is also important that studnets understand that Age is of data\sphinxhyphen{}type character and thus has to be entered using “” whereas Number of Females and Number of Males is of type numeric and can be entered as numbers.

\sphinxAtStartPar
Create R objects for the elements needed to produce the boxplot:

\begin{sphinxuseclass}{cell}\begin{sphinxVerbatimInput}

\begin{sphinxuseclass}{cell_input}
\begin{sphinxVerbatim}[commandchars=\\\{\}]
\PYG{n}{age}\PYG{+w}{  }\PYG{o}{\PYGZlt{}\PYGZhy{}}\PYG{+w}{ }\PYG{n+nf}{c}\PYG{p}{(}\PYG{l+s}{\PYGZdq{}}\PYG{l+s}{0\PYGZhy{}4\PYGZdq{}}\PYG{p}{,}\PYG{+w}{ }\PYG{l+s}{\PYGZdq{}}\PYG{l+s}{5\PYGZhy{}9\PYGZdq{}}\PYG{p}{,}\PYG{+w}{ }\PYG{l+s}{\PYGZdq{}}\PYG{l+s}{10\PYGZhy{}14\PYGZdq{}}\PYG{p}{,}\PYG{+w}{ }\PYG{l+s}{\PYGZdq{}}\PYG{l+s}{15\PYGZhy{}19\PYGZdq{}}\PYG{p}{,}\PYG{+w}{ }\PYG{l+s}{\PYGZdq{}}\PYG{l+s}{20\PYGZhy{}24\PYGZdq{}}\PYG{p}{,}\PYG{+w}{ }\PYG{l+s}{\PYGZdq{}}\PYG{l+s}{30\PYGZhy{}34\PYGZdq{}}\PYG{p}{,}\PYG{+w}{ }\PYG{l+s}{\PYGZdq{}}\PYG{l+s}{40\PYGZhy{}44\PYGZdq{}}\PYG{p}{,}\PYG{+w}{ }\PYG{l+s}{\PYGZdq{}}\PYG{l+s}{45\PYGZhy{}49\PYGZdq{}}\PYG{p}{,}\PYG{+w}{ }\PYG{l+s}{\PYGZdq{}}\PYG{l+s}{50\PYGZhy{}54\PYGZdq{}}\PYG{p}{,}\PYG{+w}{ }\PYG{l+s}{\PYGZdq{}}\PYG{l+s}{55\PYGZhy{}59\PYGZdq{}}\PYG{p}{,}\PYG{+w}{ }\PYG{l+s}{\PYGZdq{}}\PYG{l+s}{60\PYGZhy{}64\PYGZdq{}}\PYG{p}{,}
\PYG{+w}{         }\PYG{l+s}{\PYGZdq{}}\PYG{l+s}{65\PYGZhy{}69\PYGZdq{}}\PYG{p}{,}\PYG{l+s}{\PYGZdq{}}\PYG{l+s}{70\PYGZhy{}74\PYGZdq{}}\PYG{p}{,}\PYG{+w}{ }\PYG{l+s}{\PYGZdq{}}\PYG{l+s}{75\PYGZhy{}79\PYGZdq{}}\PYG{p}{,}\PYG{+w}{ }\PYG{l+s}{\PYGZdq{}}\PYG{l+s}{80\PYGZhy{}84\PYGZdq{}}\PYG{p}{,}\PYG{+w}{ }\PYG{l+s}{\PYGZdq{}}\PYG{l+s}{85\PYGZhy{}89\PYGZdq{}}\PYG{p}{,}\PYG{+w}{ }\PYG{l+s}{\PYGZdq{}}\PYG{l+s}{90\PYGZhy{}94\PYGZdq{}}\PYG{p}{,}\PYG{+w}{ }\PYG{l+s}{\PYGZdq{}}\PYG{l+s}{95\PYGZhy{}99 \PYGZdq{}}\PYG{p}{)}

\PYG{n}{num\PYGZus{}f}\PYG{+w}{  }\PYG{o}{\PYGZlt{}\PYGZhy{}}\PYG{+w}{ }\PYG{n+nf}{c}\PYG{p}{(}\PYG{l+m}{3487490}\PYG{p}{,}\PYG{+w}{ }\PYG{l+m}{3404421}\PYG{p}{,}\PYG{+w}{ }\PYG{l+m}{3444606}\PYG{p}{,}\PYG{+w}{ }\PYG{l+m}{3225971}\PYG{p}{,}\PYG{+w}{ }\PYG{l+m}{2656730}\PYG{p}{,}\PYG{+w}{ }\PYG{l+m}{1946994}\PYG{p}{,}\PYG{+w}{ }\PYG{l+m}{1606763}\PYG{p}{,}\PYG{+w}{ }\PYG{l+m}{1170869}\PYG{p}{,}\PYG{+w}{ }\PYG{l+m}{858998}\PYG{p}{,}\PYG{+w}{ }\PYG{l+m}{672025}\PYG{p}{,}\PYG{+w}{  }\PYG{l+m}{516769}\PYG{p}{,}
\PYG{+w}{            }\PYG{l+m}{387773}\PYG{p}{,}\PYG{+w}{ }\PYG{l+m}{272320}\PYG{p}{,}\PYG{+w}{ }\PYG{l+m}{168499}\PYG{p}{,}\PYG{+w}{  }\PYG{l+m}{93724}\PYG{p}{,}\PYG{+w}{ }\PYG{l+m}{39226}\PYG{p}{,}\PYG{+w}{ }\PYG{l+m}{10461}\PYG{p}{,}\PYG{+w}{ }\PYG{l+m}{1520}\PYG{+w}{ }\PYG{p}{)}

\PYG{n}{num\PYGZus{}m}\PYG{+w}{  }\PYG{o}{\PYGZlt{}\PYGZhy{}}\PYG{+w}{ }\PYG{n+nf}{c}\PYG{p}{(}\PYG{l+m}{3531278}\PYG{p}{,}\PYG{+w}{ }\PYG{l+m}{3431166}\PYG{p}{,}\PYG{+w}{ }\PYG{l+m}{3473103}\PYG{p}{,}\PYG{+w}{ }\PYG{l+m}{3249738}\PYG{p}{,}\PYG{+w}{  }\PYG{l+m}{2664966}\PYG{p}{,}\PYG{+w}{ }\PYG{l+m}{1926219}\PYG{p}{,}\PYG{+w}{ }\PYG{l+m}{1610246}\PYG{p}{,}\PYG{+w}{ }\PYG{l+m}{1196519}\PYG{p}{,}\PYG{+w}{  }\PYG{l+m}{890662}\PYG{p}{,}\PYG{+w}{  }\PYG{l+m}{679943}\PYG{p}{,}\PYG{+w}{ }\PYG{l+m}{488074}\PYG{p}{,}
\PYG{+w}{            }\PYG{l+m}{340546}\PYG{p}{,}\PYG{+w}{  }\PYG{l+m}{229912}\PYG{p}{,}\PYG{+w}{  }\PYG{l+m}{138357}\PYG{p}{,}\PYG{+w}{ }\PYG{l+m}{73093}\PYG{p}{,}\PYG{+w}{  }\PYG{l+m}{28473}\PYG{p}{,}\PYG{+w}{ }\PYG{l+m}{6918}\PYG{p}{,}\PYG{+w}{ }\PYG{l+m}{902}\PYG{p}{)}
\end{sphinxVerbatim}

\end{sphinxuseclass}\end{sphinxVerbatimInput}

\end{sphinxuseclass}
\sphinxAtStartPar
The total number can be created by using the R operation of addition for R objects

\begin{sphinxuseclass}{cell}\begin{sphinxVerbatimInput}

\begin{sphinxuseclass}{cell_input}
\begin{sphinxVerbatim}[commandchars=\\\{\}]
\PYG{n}{total}\PYG{+w}{  }\PYG{o}{\PYGZlt{}\PYGZhy{}}\PYG{+w}{ }\PYG{n}{num\PYGZus{}f}\PYG{+w}{ }\PYG{o}{+}\PYG{+w}{ }\PYG{n}{num\PYGZus{}m}
\end{sphinxVerbatim}

\end{sphinxuseclass}\end{sphinxVerbatimInput}

\end{sphinxuseclass}
\sphinxAtStartPar
Now we can do a barplot as in the previous examples. Here is the plot for the total number of people:

\begin{sphinxuseclass}{cell}\begin{sphinxVerbatimInput}

\begin{sphinxuseclass}{cell_input}
\begin{sphinxVerbatim}[commandchars=\\\{\}]
\PYG{n+nf}{barplot}\PYG{p}{(}\PYG{n}{total}\PYG{p}{,}\PYG{+w}{ }\PYG{n}{names.arg}\PYG{+w}{ }\PYG{o}{=}\PYG{+w}{ }\PYG{n}{age}\PYG{p}{,}\PYG{+w}{ }\PYG{n}{main}\PYG{+w}{ }\PYG{o}{=}\PYG{+w}{ }\PYG{l+s}{\PYGZdq{}}\PYG{l+s}{Number of people in Kenya 2022 by age groups\PYGZdq{}}\PYG{p}{)}
\end{sphinxVerbatim}

\end{sphinxuseclass}\end{sphinxVerbatimInput}
\begin{sphinxVerbatimOutput}

\begin{sphinxuseclass}{cell_output}
\noindent\sphinxincludegraphics{{bcdde9677efb2a51a59f37ce14dcc04f85db9b6c86855fb865f6a2dda214b58c}.png}

\end{sphinxuseclass}\end{sphinxVerbatimOutput}

\end{sphinxuseclass}
\sphinxAtStartPar
Here is the plot for the females

\begin{sphinxuseclass}{cell}\begin{sphinxVerbatimInput}

\begin{sphinxuseclass}{cell_input}
\begin{sphinxVerbatim}[commandchars=\\\{\}]
\PYG{n+nf}{barplot}\PYG{p}{(}\PYG{n}{num\PYGZus{}f}\PYG{p}{,}\PYG{+w}{ }\PYG{n}{names.arg}\PYG{+w}{ }\PYG{o}{=}\PYG{+w}{ }\PYG{n}{age}\PYG{p}{,}\PYG{+w}{ }\PYG{n}{main}\PYG{+w}{ }\PYG{o}{=}\PYG{+w}{ }\PYG{l+s}{\PYGZdq{}}\PYG{l+s}{Number of women in Kenya 2022 by age groups\PYGZdq{}}\PYG{p}{)}
\end{sphinxVerbatim}

\end{sphinxuseclass}\end{sphinxVerbatimInput}
\begin{sphinxVerbatimOutput}

\begin{sphinxuseclass}{cell_output}
\noindent\sphinxincludegraphics{{a92171ac70a8ff569fd9f8d1bd7c3a847e0de715a693f0c903d829b8bfe5b4bf}.png}

\end{sphinxuseclass}\end{sphinxVerbatimOutput}

\end{sphinxuseclass}
\sphinxAtStartPar
Here is the number of men:

\begin{sphinxuseclass}{cell}\begin{sphinxVerbatimInput}

\begin{sphinxuseclass}{cell_input}
\begin{sphinxVerbatim}[commandchars=\\\{\}]
\PYG{n+nf}{barplot}\PYG{p}{(}\PYG{n}{num\PYGZus{}m}\PYG{p}{,}\PYG{+w}{ }\PYG{n}{names.arg}\PYG{+w}{ }\PYG{o}{=}\PYG{+w}{ }\PYG{n}{age}\PYG{p}{,}\PYG{+w}{ }\PYG{n}{main}\PYG{+w}{ }\PYG{o}{=}\PYG{+w}{ }\PYG{l+s}{\PYGZdq{}}\PYG{l+s}{Number of men in Kenya 2022 by age groups\PYGZdq{}}\PYG{p}{)}
\end{sphinxVerbatim}

\end{sphinxuseclass}\end{sphinxVerbatimInput}
\begin{sphinxVerbatimOutput}

\begin{sphinxuseclass}{cell_output}
\noindent\sphinxincludegraphics{{3b7f8d0b4db2c5ede500148d9c765abab9d262e50a7d0b0fd6c6a933ea97eb6a}.png}

\end{sphinxuseclass}\end{sphinxVerbatimOutput}

\end{sphinxuseclass}
\sphinxAtStartPar
You might discuss with the students some details of producing the barplot with R. For example, if we look at the total numbers we see that not all age buckets can be displayed because the box of the plot is too narrow. This can be changed in the same way as we did in the course, for instance by using:

\begin{sphinxuseclass}{cell}\begin{sphinxVerbatimInput}

\begin{sphinxuseclass}{cell_input}
\begin{sphinxVerbatim}[commandchars=\\\{\}]
\PYG{n+nf}{barplot}\PYG{p}{(}\PYG{n}{total}\PYG{p}{,}\PYG{+w}{ }\PYG{n}{names.arg}\PYG{+w}{ }\PYG{o}{=}\PYG{+w}{ }\PYG{n}{age}\PYG{p}{,}\PYG{+w}{ }\PYG{n}{horiz}\PYG{+w}{ }\PYG{o}{=}\PYG{+w}{ }\PYG{n+nb+bp}{T}\PYG{p}{,}\PYG{+w}{ }\PYG{n}{las}\PYG{+w}{ }\PYG{o}{=}\PYG{+w}{ }\PYG{l+m}{1}\PYG{p}{,}\PYG{+w}{ }\PYG{n}{main}\PYG{+w}{ }\PYG{o}{=}\PYG{+w}{ }\PYG{l+s}{\PYGZdq{}}\PYG{l+s}{Number of people in Kenya 2022 by age groups\PYGZdq{}}\PYG{p}{)}
\end{sphinxVerbatim}

\end{sphinxuseclass}\end{sphinxVerbatimInput}
\begin{sphinxVerbatimOutput}

\begin{sphinxuseclass}{cell_output}
\noindent\sphinxincludegraphics{{b842a9240632c646ec2338d03ccfccbf0c6c8a280ce12a4a605b1abb90072ed4}.png}

\end{sphinxuseclass}\end{sphinxVerbatimOutput}

\end{sphinxuseclass}
\sphinxAtStartPar
This looks better but now we have the problem that the \(x\)\sphinxhyphen{}tick marks are shown in scientific notatione, e.i. 1e + 06 means for example \(10^6\) or 1000000 and 7e+06 means \(7 \times 10^6\) or 7  x 1000000 etc.

\sphinxAtStartPar
What you could do here is for example changing the units by dividing totals by \(10^6\) and annotate the plot
accordingly. Here is an example:

\begin{sphinxuseclass}{cell}\begin{sphinxVerbatimInput}

\begin{sphinxuseclass}{cell_input}
\begin{sphinxVerbatim}[commandchars=\\\{\}]
\PYG{n+nf}{barplot}\PYG{p}{(}\PYG{n}{total}\PYG{o}{/}\PYG{l+m}{10}\PYG{o}{\PYGZca{}}\PYG{l+m}{6}\PYG{p}{,}\PYG{+w}{ }\PYG{n}{names.arg}\PYG{+w}{ }\PYG{o}{=}\PYG{+w}{ }\PYG{n}{age}\PYG{p}{,}\PYG{+w}{ }\PYG{n}{horiz}\PYG{+w}{ }\PYG{o}{=}\PYG{+w}{ }\PYG{n+nb+bp}{T}\PYG{p}{,}\PYG{+w}{ }\PYG{n}{las}\PYG{+w}{ }\PYG{o}{=}\PYG{+w}{ }\PYG{l+m}{1}\PYG{p}{,}\PYG{+w}{ }
\PYG{+w}{        }\PYG{n}{main}\PYG{+w}{ }\PYG{o}{=}\PYG{+w}{ }\PYG{l+s}{\PYGZdq{}}\PYG{l+s}{Number of people in Kenya 2022 by age groups in millions\PYGZdq{}}\PYG{p}{)}
\end{sphinxVerbatim}

\end{sphinxuseclass}\end{sphinxVerbatimInput}
\begin{sphinxVerbatimOutput}

\begin{sphinxuseclass}{cell_output}
\noindent\sphinxincludegraphics{{09d827fec20734e2d1d63badccc27590ef7e26d7349d6c7559f828bf782b462b}.png}

\end{sphinxuseclass}\end{sphinxVerbatimOutput}

\end{sphinxuseclass}\begin{enumerate}
\sphinxsetlistlabels{\arabic}{enumi}{enumii}{}{.}%
\setcounter{enumi}{1}
\item {} 
\sphinxAtStartPar
Demographers usually use the following terminology: A population distribution is defined by the following layers: The bottom\sphinxhyphen{}layer in terms of age group refers to the counts of people in the 0 to 24 years old age groups.
The lower middle\sphinxhyphen{}layer refers to the counts of people in the 25 to 49 years old age groups. The upper middle\sphinxhyphen{}layer refers to the count of people in the 50 to 74 years old age groups. The top layer refers to the count of people in the 75 to 100+ years old age groups. Depending on which layer is the largest a country’s “shape” is defined by demorgaphers as a bottom\sphinxhyphen{}layered a middle layered and a top layered country.

\end{enumerate}

\sphinxAtStartPar
Use R to do a horizontal box\sphinxhyphen{}plot where the number of people in each age group are shown in percent. For computing the percentages you will need the total number of people over all age groups. Fortunately R has a function for this. It is called \sphinxcode{\sphinxupquote{sum()}} and is applied as follows: Assume you have a vector \sphinxcode{\sphinxupquote{x  <\sphinxhyphen{} c(1,4,5)}}, then using the sum function \sphinxcode{\sphinxupquote{sum(x)}} you will get 10.

\sphinxAtStartPar
Compute the percentages and do the barplot with adjusted title:

\begin{sphinxuseclass}{cell}\begin{sphinxVerbatimInput}

\begin{sphinxuseclass}{cell_input}
\begin{sphinxVerbatim}[commandchars=\\\{\}]
\PYG{n}{perc}\PYG{+w}{  }\PYG{o}{\PYGZlt{}\PYGZhy{}}\PYG{+w}{ }\PYG{n}{total}\PYG{o}{/}\PYG{n+nf}{sum}\PYG{p}{(}\PYG{n}{total}\PYG{p}{)}
\end{sphinxVerbatim}

\end{sphinxuseclass}\end{sphinxVerbatimInput}

\end{sphinxuseclass}
\begin{sphinxuseclass}{cell}\begin{sphinxVerbatimInput}

\begin{sphinxuseclass}{cell_input}
\begin{sphinxVerbatim}[commandchars=\\\{\}]
\PYG{n+nf}{barplot}\PYG{p}{(}\PYG{n}{perc}\PYG{p}{,}\PYG{+w}{ }\PYG{n}{names.arg}\PYG{+w}{ }\PYG{o}{=}\PYG{+w}{ }\PYG{n}{age}\PYG{p}{,}\PYG{+w}{ }\PYG{n}{horiz}\PYG{+w}{ }\PYG{o}{=}\PYG{+w}{ }\PYG{n+nb+bp}{T}\PYG{p}{,}\PYG{+w}{ }\PYG{n}{las}\PYG{+w}{ }\PYG{o}{=}\PYG{+w}{ }\PYG{l+m}{1}\PYG{p}{,}\PYG{+w}{ }
\PYG{+w}{        }\PYG{n}{main}\PYG{+w}{ }\PYG{o}{=}\PYG{+w}{ }\PYG{l+s}{\PYGZdq{}}\PYG{l+s}{Number of people in Kenya 2022 by age groups in percent\PYGZdq{}}\PYG{p}{)}
\end{sphinxVerbatim}

\end{sphinxuseclass}\end{sphinxVerbatimInput}
\begin{sphinxVerbatimOutput}

\begin{sphinxuseclass}{cell_output}
\noindent\sphinxincludegraphics{{8b4d429e9dc0794b7b934716392a0ff7cf3540a2fc8b45955dc2720c1e2bc289}.png}

\end{sphinxuseclass}\end{sphinxVerbatimOutput}

\end{sphinxuseclass}\begin{enumerate}
\sphinxsetlistlabels{\arabic}{enumi}{enumii}{}{.}%
\setcounter{enumi}{2}
\item {} 
\sphinxAtStartPar
Identify two age groups in which the number of people in one age group is
approximately double the count in the other age group.

\end{enumerate}

\sphinxAtStartPar
Answers vary. Possible selections: the count of 45 – 49 years old is approximately
double the 60 – 64 years old. Also, the count of 0 – 4 years old is a little less than
double the count of 30 – 34 years old. Other answers should be considered.
\begin{enumerate}
\sphinxsetlistlabels{\arabic}{enumi}{enumii}{}{.}%
\setcounter{enumi}{3}
\item {} 
\sphinxAtStartPar
“Old” and “young” are subjective descriptions that in many cases are defined by
several factors other than age (for example, health status, or income status). For
this unit, however, consider the definition of “young” as people less than 10 years
old, and the definition of “old” as people who are 65 years old or older. What is the
ratio of “old” to “young” using the above definitions of young and old?

\end{enumerate}

\sphinxAtStartPar
Answer: The “old” population is 1,791,724 people and the “young” population is 20,772,064
people. The ratio is 1,791,724 to 20,772,064. The value of the ratio is approximately
0.09. You could also say that there are 9 old people for 100 young ones.

\sphinxAtStartPar
5.If there are approximately 500 students in a typical school for students who are 5 to
14 years old, estimate the number of schools needed to educate the students who
are 5 to 14 years old.

\sphinxAtStartPar
Add the count of the 5\sphinxhyphen{} 9 years old and the count of the 10 – 14 years old, or:
6,835,587 + 6,917,709 = 13,753,296 people. The estimate of the number of schools
would be:
13,753,296 divided by 500 students per school is approximately 27,507 schools.
\begin{enumerate}
\sphinxsetlistlabels{\arabic}{enumi}{enumii}{}{.}%
\setcounter{enumi}{5}
\item {} 
\sphinxAtStartPar
Here is an R challenge that you can do, if you have enough time left and if you love challenges. It will not This will not be graded. It is an option to give this challenge a try. You can also come back to it later, when you have more experience with R. When you have no time, just leave it out.

\end{enumerate}

\sphinxAtStartPar
Demographers, have a very nice visualization of population counts by 5 year age groups and by sex, which are called “population pyramids”. These graphs are a very nice example of the power of data visualization, becuase they convey at one glance whether a particular country is a botto, middle laydred or a top layered country and how the relative cohort sizes of men and women look like for each age cohort.

\sphinxAtStartPar
It looks approcimately like this:

\sphinxAtStartPar
\sphinxincludegraphics{{pop_pyramid_example}.png}

\sphinxAtStartPar
Could you construct a similar graph for the Kenya data using the barplot function and R’s graphics options? You might consult the help function, or the internet when you have internet access.

\begin{sphinxuseclass}{cell}\begin{sphinxVerbatimInput}

\begin{sphinxuseclass}{cell_input}
\begin{sphinxVerbatim}[commandchars=\\\{\}]
\PYG{n+nf}{barplot}\PYG{p}{(}\PYG{n}{num\PYGZus{}f}\PYG{p}{,}\PYG{+w}{ }\PYG{n}{names.arg}\PYG{+w}{ }\PYG{o}{=}\PYG{+w}{ }\PYG{n}{age}\PYG{p}{,}\PYG{+w}{ }\PYG{n}{horiz}\PYG{+w}{ }\PYG{o}{=}\PYG{+w}{ }\PYG{n+nb+bp}{T}\PYG{p}{,}\PYG{+w}{ }\PYG{n}{las}\PYG{+w}{ }\PYG{o}{=}\PYG{+w}{ }\PYG{l+m}{1}\PYG{p}{,}\PYG{+w}{ }\PYG{n}{col}\PYG{+w}{ }\PYG{o}{=}\PYG{+w}{ }\PYG{l+s}{\PYGZdq{}}\PYG{l+s}{red\PYGZdq{}}\PYG{p}{)}
\end{sphinxVerbatim}

\end{sphinxuseclass}\end{sphinxVerbatimInput}
\begin{sphinxVerbatimOutput}

\begin{sphinxuseclass}{cell_output}
\noindent\sphinxincludegraphics{{1edf78697260d9a28b612d03d29db929fae8d6b89ab5bf320a8fee9e8c685729}.png}

\end{sphinxuseclass}\end{sphinxVerbatimOutput}

\end{sphinxuseclass}
\begin{sphinxuseclass}{cell}\begin{sphinxVerbatimInput}

\begin{sphinxuseclass}{cell_input}
\begin{sphinxVerbatim}[commandchars=\\\{\}]
\PYG{n+nf}{barplot}\PYG{p}{(}\PYG{o}{\PYGZhy{}}\PYG{+w}{ }\PYG{n}{num\PYGZus{}m}\PYG{p}{,}\PYG{+w}{ }\PYG{n}{names.arg}\PYG{+w}{ }\PYG{o}{=}\PYG{+w}{ }\PYG{n}{age}\PYG{p}{,}\PYG{+w}{ }\PYG{n}{horiz}\PYG{+w}{ }\PYG{o}{=}\PYG{+w}{ }\PYG{n+nb+bp}{T}\PYG{p}{,}\PYG{+w}{ }\PYG{n}{las}\PYG{+w}{ }\PYG{o}{=}\PYG{+w}{ }\PYG{l+m}{1}\PYG{p}{,}\PYG{+w}{ }\PYG{n}{col}\PYG{+w}{ }\PYG{o}{=}\PYG{+w}{ }\PYG{l+s}{\PYGZdq{}}\PYG{l+s}{blue\PYGZdq{}}\PYG{p}{,}\PYG{+w}{ }\PYG{n}{yaxt}\PYG{+w}{ }\PYG{o}{=}\PYG{+w}{ }\PYG{l+s}{\PYGZsq{}}\PYG{l+s}{n\PYGZsq{}}\PYG{p}{,}\PYG{+w}{ }\PYG{n}{xaxt}\PYG{+w}{ }\PYG{o}{=}\PYG{+w}{ }\PYG{l+s}{\PYGZsq{}}\PYG{l+s}{n\PYGZsq{}}\PYG{p}{)}
\PYG{n+nf}{axis}\PYG{p}{(}\PYG{l+m}{1}\PYG{p}{,}\PYG{+w}{ }\PYG{n}{at}\PYG{+w}{ }\PYG{o}{=}\PYG{+w}{ }\PYG{n+nf}{c}\PYG{p}{(}\PYG{l+m}{0}\PYG{p}{,}\PYG{+w}{ }\PYG{l+m}{\PYGZhy{}500000}\PYG{p}{,}\PYG{+w}{ }\PYG{l+m}{\PYGZhy{}1500000}\PYG{p}{,}\PYG{+w}{ }\PYG{l+m}{\PYGZhy{}2500000}\PYG{p}{,}\PYG{+w}{ }\PYG{l+m}{\PYGZhy{}3500000}\PYG{p}{)}\PYG{p}{,}
\PYG{+w}{     }\PYG{n}{labels}\PYG{+w}{ }\PYG{o}{=}\PYG{+w}{ }\PYG{n+nf}{c}\PYG{p}{(}\PYG{l+s}{\PYGZdq{}}\PYG{l+s}{0\PYGZdq{}}\PYG{p}{,}\PYG{+w}{ }\PYG{l+s}{\PYGZdq{}}\PYG{l+s}{500000\PYGZdq{}}\PYG{p}{,}\PYG{+w}{ }\PYG{l+s}{\PYGZdq{}}\PYG{l+s}{1500000\PYGZdq{}}\PYG{p}{,}\PYG{+w}{ }\PYG{l+s}{\PYGZdq{}}\PYG{l+s}{2500000\PYGZdq{}}\PYG{p}{,}\PYG{+w}{ }\PYG{l+s}{\PYGZdq{}}\PYG{l+s}{3500000\PYGZdq{}}\PYG{p}{)}\PYG{p}{)}
\end{sphinxVerbatim}

\end{sphinxuseclass}\end{sphinxVerbatimInput}
\begin{sphinxVerbatimOutput}

\begin{sphinxuseclass}{cell_output}
\noindent\sphinxincludegraphics{{de127e6a14f262f578f393cae376748c31d2c4c7c5906f326b9ea23cde67389f}.png}

\end{sphinxuseclass}\end{sphinxVerbatimOutput}

\end{sphinxuseclass}
\begin{sphinxuseclass}{cell}\begin{sphinxVerbatimInput}

\begin{sphinxuseclass}{cell_input}
\begin{sphinxVerbatim}[commandchars=\\\{\}]
\PYG{n+nf}{par}\PYG{p}{(}\PYG{n}{mar}\PYG{o}{=}\PYG{n+nf}{c}\PYG{p}{(}\PYG{l+m}{0.5}\PYG{p}{,}\PYG{+w}{ }\PYG{l+m}{3.7}\PYG{p}{,}\PYG{+w}{ }\PYG{l+m}{0.2}\PYG{p}{,}\PYG{+w}{ }\PYG{l+m}{0.2}\PYG{p}{)}\PYG{p}{,}\PYG{+w}{ }\PYG{n}{mfrow}\PYG{o}{=}\PYG{n+nf}{c}\PYG{p}{(}\PYG{l+m}{1}\PYG{p}{,}\PYG{l+m}{2}\PYG{p}{)}\PYG{p}{,}
\PYG{+w}{     }\PYG{n}{oma}\PYG{+w}{ }\PYG{o}{=}\PYG{+w}{ }\PYG{n+nf}{c}\PYG{p}{(}\PYG{l+m}{4}\PYG{p}{,}\PYG{+w}{ }\PYG{l+m}{4}\PYG{p}{,}\PYG{+w}{ }\PYG{l+m}{0.2}\PYG{+w}{ }\PYG{p}{,}\PYG{+w}{ }\PYG{l+m}{0.2}\PYG{p}{)}\PYG{p}{)}

\PYG{n+nf}{barplot}\PYG{p}{(}\PYG{o}{\PYGZhy{}}\PYG{+w}{ }\PYG{n}{num\PYGZus{}m}\PYG{p}{,}\PYG{+w}{ }\PYG{n}{names.arg}\PYG{+w}{ }\PYG{o}{=}\PYG{+w}{ }\PYG{n}{age}\PYG{p}{,}\PYG{+w}{ }\PYG{n}{horiz}\PYG{+w}{ }\PYG{o}{=}\PYG{+w}{ }\PYG{n+nb+bp}{T}\PYG{p}{,}\PYG{+w}{ }\PYG{n}{las}\PYG{+w}{ }\PYG{o}{=}\PYG{+w}{ }\PYG{l+m}{1}\PYG{p}{,}\PYG{+w}{ }\PYG{n}{col}\PYG{+w}{ }\PYG{o}{=}\PYG{+w}{ }\PYG{l+s}{\PYGZdq{}}\PYG{l+s}{blue\PYGZdq{}}\PYG{p}{,}\PYG{+w}{ }\PYG{n}{yaxt}\PYG{+w}{ }\PYG{o}{=}\PYG{+w}{ }\PYG{l+s}{\PYGZsq{}}\PYG{l+s}{n\PYGZsq{}}\PYG{p}{,}\PYG{+w}{ }\PYG{n}{xaxt}\PYG{+w}{ }\PYG{o}{=}\PYG{+w}{ }\PYG{l+s}{\PYGZsq{}}\PYG{l+s}{n\PYGZsq{}}\PYG{p}{)}
\PYG{n+nf}{axis}\PYG{p}{(}\PYG{l+m}{1}\PYG{p}{,}\PYG{+w}{ }\PYG{n}{at}\PYG{+w}{ }\PYG{o}{=}\PYG{+w}{ }\PYG{n+nf}{c}\PYG{p}{(}\PYG{l+m}{0}\PYG{p}{,}\PYG{+w}{ }\PYG{l+m}{\PYGZhy{}500000}\PYG{p}{,}\PYG{+w}{ }\PYG{l+m}{\PYGZhy{}1500000}\PYG{p}{,}\PYG{+w}{ }\PYG{l+m}{\PYGZhy{}2500000}\PYG{p}{,}\PYG{+w}{ }\PYG{l+m}{\PYGZhy{}3500000}\PYG{p}{)}\PYG{p}{,}
\PYG{+w}{     }\PYG{n}{labels}\PYG{+w}{ }\PYG{o}{=}\PYG{+w}{ }\PYG{n+nf}{c}\PYG{p}{(}\PYG{l+s}{\PYGZdq{}}\PYG{l+s}{0\PYGZdq{}}\PYG{p}{,}\PYG{+w}{ }\PYG{l+s}{\PYGZdq{}}\PYG{l+s}{500000\PYGZdq{}}\PYG{p}{,}\PYG{+w}{ }\PYG{l+s}{\PYGZdq{}}\PYG{l+s}{1500000\PYGZdq{}}\PYG{p}{,}\PYG{+w}{ }\PYG{l+s}{\PYGZdq{}}\PYG{l+s}{2500000\PYGZdq{}}\PYG{p}{,}\PYG{+w}{ }\PYG{l+s}{\PYGZdq{}}\PYG{l+s}{3500000\PYGZdq{}}\PYG{p}{)}\PYG{p}{)}

\PYG{n+nf}{barplot}\PYG{p}{(}\PYG{n}{num\PYGZus{}f}\PYG{p}{,}\PYG{+w}{ }\PYG{n}{names.arg}\PYG{+w}{ }\PYG{o}{=}\PYG{+w}{ }\PYG{n}{age}\PYG{p}{,}\PYG{+w}{ }\PYG{n}{horiz}\PYG{+w}{ }\PYG{o}{=}\PYG{+w}{ }\PYG{n+nb+bp}{T}\PYG{p}{,}\PYG{+w}{ }\PYG{n}{las}\PYG{+w}{ }\PYG{o}{=}\PYG{+w}{ }\PYG{l+m}{1}\PYG{p}{,}\PYG{+w}{ }\PYG{n}{col}\PYG{+w}{ }\PYG{o}{=}\PYG{+w}{ }\PYG{l+s}{\PYGZdq{}}\PYG{l+s}{red\PYGZdq{}}\PYG{p}{)}
\end{sphinxVerbatim}

\end{sphinxuseclass}\end{sphinxVerbatimInput}
\begin{sphinxVerbatimOutput}

\begin{sphinxuseclass}{cell_output}
\noindent\sphinxincludegraphics{{0798f5ba9f81e178476a5647a456a9a557d9c2b019b9fa9da3e779fd4609597b}.png}

\end{sphinxuseclass}\end{sphinxVerbatimOutput}

\end{sphinxuseclass}
\sphinxstepscope


\chapter{Exercises: Unit 2, Summarizing and communication lot’s of data}
\label{\detokenize{exercises_unit_2:exercises-unit-2-summarizing-and-communication-lot-s-of-data}}\label{\detokenize{exercises_unit_2::doc}}

\section{Now you try:}
\label{\detokenize{exercises_unit_2:now-you-try}}
\sphinxAtStartPar
The standard deviation of (ii) is highter than (i). If we compute we get:

\begin{sphinxuseclass}{cell}\begin{sphinxVerbatimInput}

\begin{sphinxuseclass}{cell_input}
\begin{sphinxVerbatim}[commandchars=\\\{\}]
\PYG{n+nf}{sd}\PYG{p}{(}\PYG{n+nf}{c}\PYG{p}{(}\PYG{l+m}{9}\PYG{p}{,}\PYG{l+m}{9}\PYG{p}{,}\PYG{l+m}{10}\PYG{p}{,}\PYG{l+m}{10}\PYG{p}{,}\PYG{l+m}{10}\PYG{p}{,}\PYG{l+m}{12}\PYG{p}{)}\PYG{p}{)}\PYG{o}{*}\PYG{n+nf}{sqrt}\PYG{p}{(}\PYG{l+m}{5}\PYG{o}{/}\PYG{l+m}{6}\PYG{p}{)}
\PYG{n+nf}{sd}\PYG{p}{(}\PYG{n+nf}{c}\PYG{p}{(}\PYG{l+m}{7}\PYG{p}{,}\PYG{l+m}{8}\PYG{p}{,}\PYG{l+m}{10}\PYG{p}{,}\PYG{l+m}{11}\PYG{p}{,}\PYG{l+m}{11}\PYG{p}{,}\PYG{l+m}{13}\PYG{p}{)}\PYG{p}{)}\PYG{o}{*}\PYG{n+nf}{sqrt}\PYG{p}{(}\PYG{l+m}{5}\PYG{o}{/}\PYG{l+m}{6}\PYG{p}{)}
\end{sphinxVerbatim}

\end{sphinxuseclass}\end{sphinxVerbatimInput}
\begin{sphinxVerbatimOutput}

\begin{sphinxuseclass}{cell_output}\begin{equation*}
\begin{split}1\end{split}
\end{equation*}\begin{equation*}
\begin{split}2\end{split}
\end{equation*}
\end{sphinxuseclass}\end{sphinxVerbatimOutput}

\end{sphinxuseclass}
\sphinxAtStartPar
No, this is not possible

\sphinxAtStartPar
Yes, an example would be 1,1,16. This list has an average of 6 and a standard deviation of about 7


\section{Exercises}
\label{\detokenize{exercises_unit_2:exercises}}

\subsection{Exercise 1: Histogram of monthly wages}
\label{\detokenize{exercises_unit_2:exercise-1-histogram-of-monthly-wages}}
\sphinxAtStartPar
It must be 25 \%


\subsection{Exercise 2: Which histogram is right?}
\label{\detokenize{exercises_unit_2:exercise-2-which-histogram-is-right}}
\sphinxAtStartPar
The answer is version 2 because version 1 does not have units and version 3 has the wrong units for the relative frequency scale


\subsection{Exercise 3: Different scales in a histogram}
\label{\detokenize{exercises_unit_2:exercise-3-different-scales-in-a-histogram}}
\sphinxAtStartPar
1750, 2000, 1, 0.5. The idea on density: If you spread 10 \% evenly over 1 cm = 10 mm, there is 1 percent in each mm, i.e. 1 percent per mm.


\subsection{Exercise 4: Find the average}
\label{\detokenize{exercises_unit_2:exercise-4-find-the-average}}
\sphinxAtStartPar
The average is at 4 and the mark must be at half the distance between 3 and 5

\sphinxAtStartPar
For th next list the average is at 4.33 and the mark must be accordingly nearer to 5.

\sphinxAtStartPar
The average must exactly hald the distance between the two crosses on the line


\subsection{Exercise 5: Questions about averages}
\label{\detokenize{exercises_unit_2:exercise-5-questions-about-averages}}
\sphinxAtStartPar
If the average is 1 the list consists of 10 1s, if the average is 3 the list consists of 10 3’s. The average can not be 4. It has to be between 3 and 4.


\subsection{Exercise 6: Comparing averages without computation}
\label{\detokenize{exercises_unit_2:exercise-6-comparing-averages-without-computation}}
\sphinxAtStartPar
The average of (ii) is bigger because it has one larger entry


\subsection{Exercise 7: Adding data and the change of averages}
\label{\detokenize{exercises_unit_2:exercise-7-adding-data-and-the-change-of-averages}}
\sphinxAtStartPar
\((10 \times 1.69 + 1.96)/11 = 1.71\)

\sphinxAtStartPar
You could also reason as follows: The new person is 27 cm taller than the old average. So it adds 27/11 to the old average of 169 cm

\sphinxAtStartPar
\((1.68 \times 21 + 1.96)/22 = 1.69\)

\sphinxAtStartPar
As the number of people goes up each new person has less affect on the average

\sphinxAtStartPar
2.34 m, the person would have to be a giant. Solve the equation \((1.68 \times 21 + x)/22 = 1.71\) for \(x\).


\subsection{Exercise 8: True or false?}
\label{\detokenize{exercises_unit_2:exercise-8-true-or-false}}
\sphinxAtStartPar
The conclusion does not follow. The data are cross sectional and not over time (longitudinal). The men with higher diastolic blood pressure are likely to die earlier, they will not be represented in the graph. Furthermore men with higher blood pressure are likely to be put on medications that reduce blood pressure.


\subsection{Exercise 9: Can you explain this?}
\label{\detokenize{exercises_unit_2:exercise-9-can-you-explain-this}}
\sphinxAtStartPar
During recessions, firms tend to lay off workers with the lowest seniority, who are also usually the lowest paid. This raises the average wage of those left on the payroll. When the recession ends, these low paid workers are rehired.


\subsection{Exercise 10: Working with standard deviations}
\label{\detokenize{exercises_unit_2:exercise-10-working-with-standard-deviations}}
\sphinxAtStartPar
\((188-173)/5\). This individual is thus away 3 standard deviations from the mean. \((174.66 -173)/5\) which is 0.33 standard deviations above average. \(173 - 1.5 \times 5\) which is 165.5 cm. This would be \(173 - 2.25 \times 5\) which amounts to 161.75 cm. The highest height would thus be \(173 + 2.25 \times 5\) which amounts to 185.25 cm.


\subsection{Exercise 11: Match the height}
\label{\detokenize{exercises_unit_2:exercise-11-match-the-height}}
\sphinxAtStartPar
We compute the mean and the standard deviation

\begin{sphinxuseclass}{cell}\begin{sphinxVerbatimInput}

\begin{sphinxuseclass}{cell_input}
\begin{sphinxVerbatim}[commandchars=\\\{\}]
\PYG{n}{x}\PYG{+w}{ }\PYG{o}{\PYGZlt{}\PYGZhy{}}\PYG{+w}{ }\PYG{n+nf}{c}\PYG{p}{(}\PYG{l+m}{150}\PYG{p}{,}\PYG{+w}{ }\PYG{l+m}{130}\PYG{p}{,}\PYG{+w}{ }\PYG{l+m}{180}\PYG{p}{,}\PYG{+w}{ }\PYG{l+m}{172}\PYG{p}{)}
\PYG{n+nf}{mean}\PYG{p}{(}\PYG{n}{x}\PYG{p}{)}
\PYG{n+nf}{sd}\PYG{p}{(}\PYG{n}{x}\PYG{p}{)}\PYG{o}{*}\PYG{n+nf}{sqrt}\PYG{p}{(}\PYG{l+m}{3}\PYG{o}{/}\PYG{l+m}{4}\PYG{p}{)}
\end{sphinxVerbatim}

\end{sphinxuseclass}\end{sphinxVerbatimInput}
\begin{sphinxVerbatimOutput}

\begin{sphinxuseclass}{cell_output}\begin{equation*}
\begin{split}158\end{split}
\end{equation*}\begin{equation*}
\begin{split}19.5448202856921\end{split}
\end{equation*}
\end{sphinxuseclass}\end{sphinxVerbatimOutput}

\end{sphinxuseclass}
\sphinxAtStartPar
The mean is 158 and the standard deviation is about 20. The first individual would be about average. The second would be at the lower boundary of usual height. The third is unusually high and the fourth is also unusally high.


\subsection{Exercise 12: Recognizing the spread of lists}
\label{\detokenize{exercises_unit_2:exercise-12-recognizing-the-spread-of-lists}}
\sphinxAtStartPar
Biggest c, smallest b


\subsection{Exercise 13: Guess the standard deviation}
\label{\detokenize{exercises_unit_2:exercise-13-guess-the-standard-deviation}}
\sphinxAtStartPar
(a) 1, since all deviations from the average of 50 are plus or minus 1, (b) 2, (c) 2, (e) 10


\subsection{Exercise 14: Stylized histograms, average and standard deviation}
\label{\detokenize{exercises_unit_2:exercise-14-stylized-histograms-average-and-standard-deviation}}
\sphinxAtStartPar
(a) (i), (b) (ii), (c) (v)


\subsection{Exercise 15: Estimating the size of average and standard deviation}
\label{\detokenize{exercises_unit_2:exercise-15-estimating-the-size-of-average-and-standard-deviation}}
\sphinxAtStartPar
The avarages and standard deviations should be about the same, but the investigator with the biggest sample is likely to get the tallest man as well as the shortest. The bigger the sample, the bigger teh range between smallest and largest element.


\section{Exercises R}
\label{\detokenize{exercises_unit_2:exercises-r}}

\subsection{Exercise 1: A histogram of the Nile river flow data}
\label{\detokenize{exercises_unit_2:exercise-1-a-histogram-of-the-nile-river-flow-data}}
\begin{sphinxuseclass}{cell}\begin{sphinxVerbatimInput}

\begin{sphinxuseclass}{cell_input}
\begin{sphinxVerbatim}[commandchars=\\\{\}]
\PYG{o}{?}\PYG{n}{Nile}
\end{sphinxVerbatim}

\end{sphinxuseclass}\end{sphinxVerbatimInput}

\end{sphinxuseclass}
\sphinxAtStartPar
Data are measurments of the annual flow of the river Nile at Aswan (formerly Assuan), 1871–1970, in \(10^8 m^3\)

\begin{sphinxuseclass}{cell}\begin{sphinxVerbatimInput}

\begin{sphinxuseclass}{cell_input}
\begin{sphinxVerbatim}[commandchars=\\\{\}]
\PYG{n+nf}{hist}\PYG{p}{(}\PYG{n}{Nile}\PYG{p}{)}
\end{sphinxVerbatim}

\end{sphinxuseclass}\end{sphinxVerbatimInput}
\begin{sphinxVerbatimOutput}

\begin{sphinxuseclass}{cell_output}
\noindent\sphinxincludegraphics{{423b6590271cb77a7733a4287b3e4a1d8043762d22ae2aa5dfd19862d9823577}.png}

\end{sphinxuseclass}\end{sphinxVerbatimOutput}

\end{sphinxuseclass}
\begin{sphinxuseclass}{cell}\begin{sphinxVerbatimInput}

\begin{sphinxuseclass}{cell_input}
\begin{sphinxVerbatim}[commandchars=\\\{\}]
\PYG{n+nf}{hist}\PYG{p}{(}\PYG{n}{Nile}\PYG{p}{,}\PYG{+w}{ }\PYG{n}{main}\PYG{+w}{ }\PYG{o}{=}\PYG{+w}{ }\PYG{l+s}{\PYGZdq{}}\PYG{l+s}{Histogram of Nile river flow data at Aswan Egypt\PYGZdq{}}\PYG{p}{,}
\PYG{+w}{    }\PYG{n}{xlab}\PYG{+w}{ }\PYG{o}{=}\PYG{+w}{ }\PYG{l+s}{\PYGZdq{}}\PYG{l+s}{Annual flow is recorded are 100 millions of cubic meters\PYGZdq{}}\PYG{p}{)}
\end{sphinxVerbatim}

\end{sphinxuseclass}\end{sphinxVerbatimInput}
\begin{sphinxVerbatimOutput}

\begin{sphinxuseclass}{cell_output}
\noindent\sphinxincludegraphics{{4bfaf2cf4dd4a859d0666d0c993acc0a11b203a7458386a192e2952e1bf0833a}.png}

\end{sphinxuseclass}\end{sphinxVerbatimOutput}

\end{sphinxuseclass}

\subsection{Exercise 2: Analyze the full height dataset}
\label{\detokenize{exercises_unit_2:exercise-2-analyze-the-full-height-dataset}}
\begin{sphinxuseclass}{cell}\begin{sphinxVerbatimInput}

\begin{sphinxuseclass}{cell_input}
\begin{sphinxVerbatim}[commandchars=\\\{\}]
\PYG{n+nf}{library}\PYG{p}{(}\PYG{n}{JWL}\PYG{p}{)}
\PYG{n}{dat}\PYG{+w}{  }\PYG{o}{\PYGZlt{}\PYGZhy{}}\PYG{+w}{ }\PYG{n}{socr\PYGZus{}height\PYGZus{}weight}
\end{sphinxVerbatim}

\end{sphinxuseclass}\end{sphinxVerbatimInput}

\end{sphinxuseclass}
\begin{sphinxuseclass}{cell}\begin{sphinxVerbatimInput}

\begin{sphinxuseclass}{cell_input}
\begin{sphinxVerbatim}[commandchars=\\\{\}]
\PYG{n}{dat}\PYG{o}{\PYGZdl{}}\PYG{n}{Height}\PYG{+w}{  }\PYG{o}{\PYGZlt{}\PYGZhy{}}\PYG{+w}{ }\PYG{n}{dat}\PYG{o}{\PYGZdl{}}\PYG{n}{Height}\PYG{o}{*}\PYG{l+m}{2.54}
\PYG{n}{dat}\PYG{o}{\PYGZdl{}}\PYG{n}{Weight}\PYG{+w}{  }\PYG{o}{\PYGZlt{}\PYGZhy{}}\PYG{+w}{ }\PYG{n}{dat}\PYG{o}{\PYGZdl{}}\PYG{n}{Weight}\PYG{o}{*}\PYG{l+m}{0.4535924}
\end{sphinxVerbatim}

\end{sphinxuseclass}\end{sphinxVerbatimInput}

\end{sphinxuseclass}
\begin{sphinxuseclass}{cell}\begin{sphinxVerbatimInput}

\begin{sphinxuseclass}{cell_input}
\begin{sphinxVerbatim}[commandchars=\\\{\}]
\PYG{n+nf}{hist}\PYG{p}{(}\PYG{n}{dat}\PYG{o}{\PYGZdl{}}\PYG{n}{Height}\PYG{p}{)}
\end{sphinxVerbatim}

\end{sphinxuseclass}\end{sphinxVerbatimInput}
\begin{sphinxVerbatimOutput}

\begin{sphinxuseclass}{cell_output}
\noindent\sphinxincludegraphics{{592c8e47163b3b668ce7d8a44740dfe0424ad5695fa45b00dcafca9c6ca902ba}.png}

\end{sphinxuseclass}\end{sphinxVerbatimOutput}

\end{sphinxuseclass}
\begin{sphinxuseclass}{cell}\begin{sphinxVerbatimInput}

\begin{sphinxuseclass}{cell_input}
\begin{sphinxVerbatim}[commandchars=\\\{\}]
\PYG{n+nf}{hist}\PYG{p}{(}\PYG{n}{dat}\PYG{o}{\PYGZdl{}}\PYG{n}{Weight}\PYG{p}{)}
\end{sphinxVerbatim}

\end{sphinxuseclass}\end{sphinxVerbatimInput}
\begin{sphinxVerbatimOutput}

\begin{sphinxuseclass}{cell_output}
\noindent\sphinxincludegraphics{{9c422a85314254abde16929bb816da9412fbd9cadd3e4c7b192b10391024ea4d}.png}

\end{sphinxuseclass}\end{sphinxVerbatimOutput}

\end{sphinxuseclass}
\begin{sphinxuseclass}{cell}\begin{sphinxVerbatimInput}

\begin{sphinxuseclass}{cell_input}
\begin{sphinxVerbatim}[commandchars=\\\{\}]
\PYG{n+nf}{mean}\PYG{p}{(}\PYG{n}{dat}\PYG{o}{\PYGZdl{}}\PYG{n}{Height}\PYG{p}{)}
\PYG{n+nf}{sd}\PYG{p}{(}\PYG{n}{dat}\PYG{o}{\PYGZdl{}}\PYG{n}{Height}\PYG{p}{)}

\PYG{n+nf}{mean}\PYG{p}{(}\PYG{n}{dat}\PYG{o}{\PYGZdl{}}\PYG{n}{Weight}\PYG{p}{)}
\PYG{n+nf}{sd}\PYG{p}{(}\PYG{n}{dat}\PYG{o}{\PYGZdl{}}\PYG{n}{Weight}\PYG{p}{)}
\end{sphinxVerbatim}

\end{sphinxuseclass}\end{sphinxVerbatimInput}
\begin{sphinxVerbatimOutput}

\begin{sphinxuseclass}{cell_output}\begin{equation*}
\begin{split}172.702508535872\end{split}
\end{equation*}\begin{equation*}
\begin{split}4.83026407886225\end{split}
\end{equation*}\begin{equation*}
\begin{split}57.6422596349381\end{split}
\end{equation*}\begin{equation*}
\begin{split}5.28929451202942\end{split}
\end{equation*}
\end{sphinxuseclass}\end{sphinxVerbatimOutput}

\end{sphinxuseclass}
\begin{sphinxuseclass}{cell}\begin{sphinxVerbatimInput}

\begin{sphinxuseclass}{cell_input}
\begin{sphinxVerbatim}[commandchars=\\\{\}]
\PYG{n+nf}{mean}\PYG{p}{(}\PYG{n}{dat}\PYG{o}{\PYGZdl{}}\PYG{n}{Height}\PYG{+w}{ }\PYG{o}{\PYGZgt{}=}\PYG{+w}{ }\PYG{n+nf}{mean}\PYG{p}{(}\PYG{n}{dat}\PYG{o}{\PYGZdl{}}\PYG{n}{Height}\PYG{p}{)}\PYG{+w}{ }\PYG{o}{\PYGZhy{}}\PYG{+w}{ }\PYG{n+nf}{sd}\PYG{p}{(}\PYG{n}{dat}\PYG{o}{\PYGZdl{}}\PYG{n}{Height}\PYG{p}{)}\PYG{+w}{ }\PYG{o}{\PYGZam{}}\PYG{+w}{ }\PYG{n}{dat}\PYG{o}{\PYGZdl{}}\PYG{n}{Height}\PYG{+w}{ }\PYG{o}{\PYGZlt{}=}\PYG{+w}{ }\PYG{n+nf}{mean}\PYG{p}{(}\PYG{n}{dat}\PYG{o}{\PYGZdl{}}\PYG{n}{Height}\PYG{p}{)}\PYG{+w}{ }\PYG{o}{+}\PYG{+w}{ }\PYG{n+nf}{sd}\PYG{p}{(}\PYG{n}{dat}\PYG{o}{\PYGZdl{}}\PYG{n}{Height}\PYG{p}{)}\PYG{p}{)}
\end{sphinxVerbatim}

\end{sphinxuseclass}\end{sphinxVerbatimInput}
\begin{sphinxVerbatimOutput}

\begin{sphinxuseclass}{cell_output}\begin{equation*}
\begin{split}0.68356\end{split}
\end{equation*}
\end{sphinxuseclass}\end{sphinxVerbatimOutput}

\end{sphinxuseclass}
\begin{sphinxuseclass}{cell}\begin{sphinxVerbatimInput}

\begin{sphinxuseclass}{cell_input}
\begin{sphinxVerbatim}[commandchars=\\\{\}]
\PYG{n+nf}{mean}\PYG{p}{(}\PYG{n}{dat}\PYG{o}{\PYGZdl{}}\PYG{n}{Height}\PYG{+w}{ }\PYG{o}{\PYGZgt{}=}\PYG{+w}{ }\PYG{n+nf}{mean}\PYG{p}{(}\PYG{n}{dat}\PYG{o}{\PYGZdl{}}\PYG{n}{Height}\PYG{p}{)}\PYG{+w}{ }\PYG{o}{\PYGZhy{}}\PYG{+w}{ }\PYG{l+m}{2}\PYG{o}{*}\PYG{n+nf}{sd}\PYG{p}{(}\PYG{n}{dat}\PYG{o}{\PYGZdl{}}\PYG{n}{Height}\PYG{p}{)}\PYG{+w}{ }\PYG{o}{\PYGZam{}}\PYG{+w}{ }\PYG{n}{dat}\PYG{o}{\PYGZdl{}}\PYG{n}{Height}\PYG{+w}{ }\PYG{o}{\PYGZlt{}=}\PYG{+w}{ }\PYG{n+nf}{mean}\PYG{p}{(}\PYG{n}{dat}\PYG{o}{\PYGZdl{}}\PYG{n}{Height}\PYG{p}{)}\PYG{+w}{ }\PYG{o}{+}\PYG{+w}{ }\PYG{l+m}{2}\PYG{o}{*}\PYG{n+nf}{sd}\PYG{p}{(}\PYG{n}{dat}\PYG{o}{\PYGZdl{}}\PYG{n}{Height}\PYG{p}{)}\PYG{p}{)}
\end{sphinxVerbatim}

\end{sphinxuseclass}\end{sphinxVerbatimInput}
\begin{sphinxVerbatimOutput}

\begin{sphinxuseclass}{cell_output}\begin{equation*}
\begin{split}0.9546\end{split}
\end{equation*}
\end{sphinxuseclass}\end{sphinxVerbatimOutput}

\end{sphinxuseclass}
\begin{sphinxuseclass}{cell}\begin{sphinxVerbatimInput}

\begin{sphinxuseclass}{cell_input}
\begin{sphinxVerbatim}[commandchars=\\\{\}]
\PYG{n+nf}{mean}\PYG{p}{(}\PYG{n}{dat}\PYG{o}{\PYGZdl{}}\PYG{n}{Height}\PYG{+w}{ }\PYG{o}{\PYGZgt{}=}\PYG{+w}{ }\PYG{n+nf}{mean}\PYG{p}{(}\PYG{n}{dat}\PYG{o}{\PYGZdl{}}\PYG{n}{Height}\PYG{p}{)}\PYG{+w}{ }\PYG{o}{\PYGZhy{}}\PYG{+w}{ }\PYG{l+m}{3}\PYG{o}{*}\PYG{n+nf}{sd}\PYG{p}{(}\PYG{n}{dat}\PYG{o}{\PYGZdl{}}\PYG{n}{Height}\PYG{p}{)}\PYG{+w}{ }\PYG{o}{\PYGZam{}}\PYG{+w}{ }\PYG{n}{dat}\PYG{o}{\PYGZdl{}}\PYG{n}{Height}\PYG{+w}{ }\PYG{o}{\PYGZlt{}=}\PYG{+w}{ }\PYG{n+nf}{mean}\PYG{p}{(}\PYG{n}{dat}\PYG{o}{\PYGZdl{}}\PYG{n}{Height}\PYG{p}{)}\PYG{+w}{ }\PYG{o}{+}\PYG{+w}{ }\PYG{l+m}{3}\PYG{o}{*}\PYG{n+nf}{sd}\PYG{p}{(}\PYG{n}{dat}\PYG{o}{\PYGZdl{}}\PYG{n}{Height}\PYG{p}{)}\PYG{p}{)}
\end{sphinxVerbatim}

\end{sphinxuseclass}\end{sphinxVerbatimInput}
\begin{sphinxVerbatimOutput}

\begin{sphinxuseclass}{cell_output}\begin{equation*}
\begin{split}0.99796\end{split}
\end{equation*}
\end{sphinxuseclass}\end{sphinxVerbatimOutput}

\end{sphinxuseclass}

\subsection{Exercise 3: Energy consumption per capita accross countries in the world}
\label{\detokenize{exercises_unit_2:exercise-3-energy-consumption-per-capita-accross-countries-in-the-world}}
\begin{sphinxuseclass}{cell}\begin{sphinxVerbatimInput}

\begin{sphinxuseclass}{cell_input}
\begin{sphinxVerbatim}[commandchars=\\\{\}]
\PYG{o}{?}\PYG{n}{energy\PYGZus{}consumption\PYGZus{}per\PYGZus{}capita}
\end{sphinxVerbatim}

\end{sphinxuseclass}\end{sphinxVerbatimInput}

\end{sphinxuseclass}
\sphinxAtStartPar
Data points show primary energy consumption per capita measured in kwh per person per year for various countries around the globe from 1965 \sphinxhyphen{} 2021

\begin{sphinxuseclass}{cell}\begin{sphinxVerbatimInput}

\begin{sphinxuseclass}{cell_input}
\begin{sphinxVerbatim}[commandchars=\\\{\}]
\PYG{n}{df}\PYG{+w}{  }\PYG{o}{\PYGZlt{}\PYGZhy{}}\PYG{+w}{ }\PYG{n}{energy\PYGZus{}consumption\PYGZus{}per\PYGZus{}capita}
\end{sphinxVerbatim}

\end{sphinxuseclass}\end{sphinxVerbatimInput}

\end{sphinxuseclass}
\begin{sphinxuseclass}{cell}\begin{sphinxVerbatimInput}

\begin{sphinxuseclass}{cell_input}
\begin{sphinxVerbatim}[commandchars=\\\{\}]
\PYG{n}{p}\PYG{+w}{ }\PYG{o}{\PYGZlt{}\PYGZhy{}}\PYG{+w}{ }\PYG{n}{df}\PYG{p}{[}\PYG{n}{df}\PYG{o}{\PYGZdl{}}\PYG{n}{Year}\PYG{+w}{ }\PYG{o}{==}\PYG{+w}{ }\PYG{l+s}{\PYGZdq{}}\PYG{l+s}{2018\PYGZdq{}}\PYG{p}{,}\PYG{+w}{ }\PYG{p}{]}
\PYG{n+nf}{head}\PYG{p}{(}\PYG{n}{p}\PYG{p}{)}
\end{sphinxVerbatim}

\end{sphinxuseclass}\end{sphinxVerbatimInput}
\begin{sphinxVerbatimOutput}

\begin{sphinxuseclass}{cell_output}\begin{equation*}
\begin{split}A data.frame: 6 × 4
\begin{tabular}{r|llll}
  & Country & Code & Year & Cons\\
  & <chr> & <chr> & <int> & <dbl>\\
\hline
	39 & Afghanistan         & AFG & 2018 &  1129.595\\
	136 & Albania             & ALB & 2018 & 14455.136\\
	191 & Algeria             & DZA & 2018 & 15898.334\\
	233 & American Samoa      & ASM & 2018 & 25953.754\\
	273 & Angola              & AGO & 2018 &  3212.779\\
	313 & Antigua and Barbuda & ATG & 2018 & 31520.479\\
\end{tabular}\end{split}
\end{equation*}
\end{sphinxuseclass}\end{sphinxVerbatimOutput}

\end{sphinxuseclass}
\begin{sphinxuseclass}{cell}\begin{sphinxVerbatimInput}

\begin{sphinxuseclass}{cell_input}
\begin{sphinxVerbatim}[commandchars=\\\{\}]
\PYG{n}{df}
\end{sphinxVerbatim}

\end{sphinxuseclass}\end{sphinxVerbatimInput}
\begin{sphinxVerbatimOutput}

\begin{sphinxuseclass}{cell_output}\begin{equation*}
\begin{split}A data.frame: 9662 × 4
\begin{tabular}{r|llll}
  & Country & Code & Year & Cons\\
  & <chr> & <chr> & <int> & <dbl>\\
\hline
	1 & Afghanistan & AFG & 1980 &  583.2944\\
	2 & Afghanistan & AFG & 1981 &  666.3782\\
	3 & Afghanistan & AFG & 1982 &  725.6599\\
	4 & Afghanistan & AFG & 1983 &  912.1396\\
	5 & Afghanistan & AFG & 1984 &  941.3926\\
	6 & Afghanistan & AFG & 1985 &  939.6124\\
	7 & Afghanistan & AFG & 1986 &  976.6691\\
	8 & Afghanistan & AFG & 1987 & 1592.7023\\
	9 & Afghanistan & AFG & 1988 & 2805.6096\\
	10 & Afghanistan & AFG & 1989 & 2700.4739\\
	11 & Afghanistan & AFG & 1990 & 2557.5864\\
	12 & Afghanistan & AFG & 1991 & 1045.3979\\
	13 & Afghanistan & AFG & 1992 &  632.8918\\
	14 & Afghanistan & AFG & 1993 &  575.6830\\
	15 & Afghanistan & AFG & 1994 &  516.2827\\
	16 & Afghanistan & AFG & 1995 &  410.1951\\
	17 & Afghanistan & AFG & 1996 &  386.4446\\
	18 & Afghanistan & AFG & 1997 &  358.4202\\
	19 & Afghanistan & AFG & 1998 &  342.0968\\
	20 & Afghanistan & AFG & 1999 &  334.5757\\
	21 & Afghanistan & AFG & 2000 &  284.5822\\
	22 & Afghanistan & AFG & 2001 &  215.8596\\
	23 & Afghanistan & AFG & 2002 &  195.9296\\
	24 & Afghanistan & AFG & 2003 &  219.9100\\
	25 & Afghanistan & AFG & 2004 &  194.5417\\
	26 & Afghanistan & AFG & 2005 &  239.8551\\
	27 & Afghanistan & AFG & 2006 &  293.0181\\
	28 & Afghanistan & AFG & 2007 &  370.4378\\
	29 & Afghanistan & AFG & 2008 &  618.7231\\
	30 & Afghanistan & AFG & 2009 &  984.1227\\
	⋮ & ⋮ & ⋮ & ⋮ & ⋮\\
	10186 & Zimbabwe & ZWE & 1990 & 5872.587\\
	10187 & Zimbabwe & ZWE & 1991 & 5864.751\\
	10188 & Zimbabwe & ZWE & 1992 & 5283.862\\
	10189 & Zimbabwe & ZWE & 1993 & 5009.730\\
	10190 & Zimbabwe & ZWE & 1994 & 4723.183\\
	10191 & Zimbabwe & ZWE & 1995 & 4619.312\\
	10192 & Zimbabwe & ZWE & 1996 & 4692.042\\
	10193 & Zimbabwe & ZWE & 1997 & 4585.107\\
	10194 & Zimbabwe & ZWE & 1998 & 4584.613\\
	10195 & Zimbabwe & ZWE & 1999 & 5247.149\\
	10196 & Zimbabwe & ZWE & 2000 & 4900.144\\
	10197 & Zimbabwe & ZWE & 2001 & 4609.955\\
	10198 & Zimbabwe & ZWE & 2002 & 4572.823\\
	10199 & Zimbabwe & ZWE & 2003 & 4431.698\\
	10200 & Zimbabwe & ZWE & 2004 & 4092.676\\
	10201 & Zimbabwe & ZWE & 2005 & 4211.419\\
	10202 & Zimbabwe & ZWE & 2006 & 4194.021\\
	10203 & Zimbabwe & ZWE & 2007 & 4085.941\\
	10204 & Zimbabwe & ZWE & 2008 & 3370.732\\
	10205 & Zimbabwe & ZWE & 2009 & 3293.989\\
	10206 & Zimbabwe & ZWE & 2010 & 3632.863\\
	10207 & Zimbabwe & ZWE & 2011 & 3900.153\\
	10208 & Zimbabwe & ZWE & 2012 & 4153.979\\
	10209 & Zimbabwe & ZWE & 2013 & 4148.077\\
	10210 & Zimbabwe & ZWE & 2014 & 4018.925\\
	10211 & Zimbabwe & ZWE & 2015 & 3956.026\\
	10212 & Zimbabwe & ZWE & 2016 & 3326.073\\
	10213 & Zimbabwe & ZWE & 2017 & 3226.617\\
	10214 & Zimbabwe & ZWE & 2018 & 3289.887\\
	10215 & Zimbabwe & ZWE & 2019 & 3374.877\\
\end{tabular}\end{split}
\end{equation*}
\end{sphinxuseclass}\end{sphinxVerbatimOutput}

\end{sphinxuseclass}
\begin{sphinxuseclass}{cell}\begin{sphinxVerbatimInput}

\begin{sphinxuseclass}{cell_input}
\begin{sphinxVerbatim}[commandchars=\\\{\}]
\PYG{n+nf}{options}\PYG{p}{(}\PYG{n}{scipen}\PYG{o}{=}\PYG{l+m}{999}\PYG{p}{)}
\PYG{n+nf}{hist}\PYG{p}{(}\PYG{n}{df}\PYG{o}{\PYGZdl{}}\PYG{n}{Cons}\PYG{p}{,}\PYG{+w}{ }\PYG{n}{main}\PYG{+w}{ }\PYG{o}{=}\PYG{+w}{ }\PYG{l+s}{\PYGZdq{}}\PYG{l+s}{Primary energy consumption per capita around the world in 2018\PYGZdq{}}\PYG{p}{,}
\PYG{+w}{    }\PYG{n}{xlab}\PYG{+w}{ }\PYG{o}{=}\PYG{+w}{ }\PYG{l+s}{\PYGZdq{}}\PYG{l+s}{Primary energy consumption kwh per person per year\PYGZdq{}}\PYG{p}{)}
\end{sphinxVerbatim}

\end{sphinxuseclass}\end{sphinxVerbatimInput}
\begin{sphinxVerbatimOutput}

\begin{sphinxuseclass}{cell_output}
\noindent\sphinxincludegraphics{{ddc51b33acf2015b785f40d74ebcab9d601f888acf089d6cac4498d7028f1f0f}.png}

\end{sphinxuseclass}\end{sphinxVerbatimOutput}

\end{sphinxuseclass}
\sphinxAtStartPar
Most countries consume below 50000 and fes more than that.

\begin{sphinxuseclass}{cell}\begin{sphinxVerbatimInput}

\begin{sphinxuseclass}{cell_input}
\begin{sphinxVerbatim}[commandchars=\\\{\}]
\PYG{n+nf}{length}\PYG{p}{(}\PYG{n}{df}\PYG{o}{\PYGZdl{}}\PYG{n}{Country}\PYG{p}{[}\PYG{n}{df}\PYG{o}{\PYGZdl{}}\PYG{n}{Year}\PYG{+w}{ }\PYG{o}{==}\PYG{+w}{ }\PYG{l+m}{2018}\PYG{+w}{ }\PYG{o}{\PYGZam{}}\PYG{+w}{ }\PYG{n}{df}\PYG{o}{\PYGZdl{}}\PYG{n}{Cons}\PYG{+w}{ }\PYG{o}{\PYGZgt{}}\PYG{+w}{ }\PYG{l+m}{50000}\PYG{p}{]}\PYG{p}{)}
\end{sphinxVerbatim}

\end{sphinxuseclass}\end{sphinxVerbatimInput}
\begin{sphinxVerbatimOutput}

\begin{sphinxuseclass}{cell_output}\begin{equation*}
\begin{split}34\end{split}
\end{equation*}
\end{sphinxuseclass}\end{sphinxVerbatimOutput}

\end{sphinxuseclass}
\begin{sphinxuseclass}{cell}\begin{sphinxVerbatimInput}

\begin{sphinxuseclass}{cell_input}
\begin{sphinxVerbatim}[commandchars=\\\{\}]
\PYG{n}{df}\PYG{o}{\PYGZdl{}}\PYG{n}{Country}\PYG{p}{[}\PYG{n}{df}\PYG{o}{\PYGZdl{}}\PYG{n}{Year}\PYG{+w}{ }\PYG{o}{==}\PYG{+w}{ }\PYG{l+m}{2018}\PYG{+w}{ }\PYG{o}{\PYGZam{}}\PYG{+w}{ }\PYG{n}{df}\PYG{o}{\PYGZdl{}}\PYG{n}{Cons}\PYG{+w}{ }\PYG{o}{\PYGZgt{}}\PYG{+w}{ }\PYG{l+m}{50000}\PYG{p}{]}
\end{sphinxVerbatim}

\end{sphinxuseclass}\end{sphinxVerbatimInput}
\begin{sphinxVerbatimOutput}

\begin{sphinxuseclass}{cell_output}\begin{equation*}
\begin{split}\begin{enumerate*}
\item 'Australia'
\item 'Bahrain'
\item 'Belgium'
\item 'Bermuda'
\item 'Brunei'
\item 'Canada'
\item 'Estonia'
\item 'Faeroe Islands'
\item 'Falkland Islands'
\item 'Finland'
\item 'Greenland'
\item 'Iceland'
\item 'Kuwait'
\item 'Luxembourg'
\item 'Malta'
\item 'Netherlands'
\item 'New Caledonia'
\item 'New Zealand'
\item 'North America'
\item 'Norway'
\item 'Oman'
\item 'Qatar'
\item 'Russia'
\item 'Saint Pierre and Miquelon'
\item 'Saudi Arabia'
\item 'Singapore'
\item 'South Korea'
\item 'Sweden'
\item 'Taiwan'
\item 'Trinidad and Tobago'
\item 'Turkmenistan'
\item 'United Arab Emirates'
\item 'United States'
\item 'United States Virgin Islands'
\end{enumerate*}\end{split}
\end{equation*}
\end{sphinxuseclass}\end{sphinxVerbatimOutput}

\end{sphinxuseclass}

\section{Project People count: The future of humanity in pictures and numbers}
\label{\detokenize{exercises_unit_2:project-people-count-the-future-of-humanity-in-pictures-and-numbers}}

\subsection{The mean age of a population}
\label{\detokenize{exercises_unit_2:the-mean-age-of-a-population}}
\sphinxAtStartPar
In the previous project we had to read the population count numbers of Kenya from a table and transfer them manually to the computer to do, for example, a population bar chart by age groups. Now that you know much more R, you can leverage the power of R to read and select data by computer.

\sphinxAtStartPar
The population data we used in the previous project for unit 1 are filed and stored in a dataset called
\sphinxcode{\sphinxupquote{population\_statistics\_by\_age\_and\_sex}} and is contained in the \sphinxcode{\sphinxupquote{JWL}} package. Let us load the package and the data and store them in an object with a shorter name, so it is easier to work with these data.

\begin{sphinxuseclass}{cell}\begin{sphinxVerbatimInput}

\begin{sphinxuseclass}{cell_input}
\begin{sphinxVerbatim}[commandchars=\\\{\}]
\PYG{n+nf}{library}\PYG{p}{(}\PYG{n}{JWL}\PYG{p}{)}
\PYG{n}{dat}\PYG{+w}{  }\PYG{o}{\PYGZlt{}\PYGZhy{}}\PYG{+w}{ }\PYG{n}{population\PYGZus{}statistics\PYGZus{}by\PYGZus{}age\PYGZus{}and\PYGZus{}sex}
\end{sphinxVerbatim}

\end{sphinxuseclass}\end{sphinxVerbatimInput}

\end{sphinxuseclass}
\sphinxAtStartPar
Let us inspect the data by using the R functions \sphinxcode{\sphinxupquote{dim()}}, \sphinxcode{\sphinxupquote{head()}}and \sphinxcode{\sphinxupquote{tail()}}. These functions give us a brief glimps how the data roughly look like. With \sphinxcode{\sphinxupquote{dim()}}we get the number of rows and columns. \sphinxcode{\sphinxupquote{head()}}shows us the first few lines of the dataframe. \sphinxcode{\sphinxupquote{tail()}} is kind of mirror to \sphinxcode{\sphinxupquote{heads()}}, for it shows us the end of the dataframe.

\begin{sphinxuseclass}{cell}\begin{sphinxVerbatimInput}

\begin{sphinxuseclass}{cell_input}
\begin{sphinxVerbatim}[commandchars=\\\{\}]
\PYG{n+nf}{dim}\PYG{p}{(}\PYG{n}{dat}\PYG{p}{)}
\end{sphinxVerbatim}

\end{sphinxuseclass}\end{sphinxVerbatimInput}
\begin{sphinxVerbatimOutput}

\begin{sphinxuseclass}{cell_output}\begin{equation*}
\begin{split}\begin{enumerate*}
\item 1369480
\item 6
\end{enumerate*}\end{split}
\end{equation*}
\end{sphinxuseclass}\end{sphinxVerbatimOutput}

\end{sphinxuseclass}
\begin{sphinxuseclass}{cell}\begin{sphinxVerbatimInput}

\begin{sphinxuseclass}{cell_input}
\begin{sphinxVerbatim}[commandchars=\\\{\}]
\PYG{n+nf}{head}\PYG{p}{(}\PYG{n}{dat}\PYG{p}{)}
\end{sphinxVerbatim}

\end{sphinxuseclass}\end{sphinxVerbatimInput}
\begin{sphinxVerbatimOutput}

\begin{sphinxuseclass}{cell_output}\begin{equation*}
\begin{split}A tibble: 6 × 6
\begin{tabular}{r|llllll}
  & Country & ISO2 & Year & Sex & Age & POP\\
  & <fct> & <fct> & <dbl> & <fct> & <fct> & <dbl>\\
\hline
	1 & Andorra & AD & 1950 & F & 0-4   & NA\\
	2 & Andorra & AD & 1950 & F & 5-9   & NA\\
	3 & Andorra & AD & 1950 & F & 10-14 & NA\\
	4 & Andorra & AD & 1950 & F & 15-19 & NA\\
	5 & Andorra & AD & 1950 & F & 20-24 & NA\\
	6 & Andorra & AD & 1950 & F & 25-29 & NA\\
\end{tabular}\end{split}
\end{equation*}
\end{sphinxuseclass}\end{sphinxVerbatimOutput}

\end{sphinxuseclass}
\begin{sphinxuseclass}{cell}\begin{sphinxVerbatimInput}

\begin{sphinxuseclass}{cell_input}
\begin{sphinxVerbatim}[commandchars=\\\{\}]
\PYG{n+nf}{tail}\PYG{p}{(}\PYG{n}{dat}\PYG{p}{)}
\end{sphinxVerbatim}

\end{sphinxuseclass}\end{sphinxVerbatimInput}
\begin{sphinxVerbatimOutput}

\begin{sphinxuseclass}{cell_output}\begin{equation*}
\begin{split}A tibble: 6 × 6
\begin{tabular}{r|llllll}
  & Country & ISO2 & Year & Sex & Age & POP\\
  & <fct> & <fct> & <dbl> & <fct> & <fct> & <dbl>\\
\hline
	1369475 & Zimbabwe & ZW & 2100 & M & 70-74 & 846700\\
	1369476 & Zimbabwe & ZW & 2100 & M & 75-79 & 760492\\
	1369477 & Zimbabwe & ZW & 2100 & M & 80-84 & 616095\\
	1369478 & Zimbabwe & ZW & 2100 & M & 85-89 & 411166\\
	1369479 & Zimbabwe & ZW & 2100 & M & 90-94 & 180252\\
	1369480 & Zimbabwe & ZW & 2100 & M & 95-99 &  64552\\
\end{tabular}\end{split}
\end{equation*}
\end{sphinxuseclass}\end{sphinxVerbatimOutput}

\end{sphinxuseclass}
\sphinxAtStartPar
There are six variables, \sphinxcode{\sphinxupquote{Country}}, \sphinxcode{\sphinxupquote{ISO2}} which encode contry name and the two letter international country code. Both allow us to idnetify countries. The dataframe is huge and we can assume that this must come from the fact that
it contains most countries in the world, which are today about 200. The \sphinxcode{\sphinxupquote{Year}} variable seems to start in 1950 and end in 2100. So our data must contain historical records as well as predictions. The data are stratified by \sphinxcode{\sphinxupquote{Sex}}, here obviously encoded as \sphinxcode{\sphinxupquote{M}} for male and \sphinxcode{\sphinxupquote{F}} for female, as well as by age groups. The start at \sphinxcode{\sphinxupquote{0\sphinxhyphen{}4}}and end at \sphinxcode{\sphinxupquote{95\sphinxhyphen{}99}}. Let’s make a check whether our guess is reasonable. Assume we have roughly 200 countries and 150 years, 2 Sexes and 20 age groups, we would get around 1.200.000 observations. So we are at about the right order of magnitude. Note that in the beginning the dataframe shows entries \sphinxcode{\sphinxupquote{NA}}. This is R’s symbol to mark missing observations. \sphinxcode{\sphinxupquote{POP}}finally is the actual count of the number of people of a certain sex and age group in a country in a given year.
\begin{enumerate}
\sphinxsetlistlabels{\arabic}{enumi}{enumii}{}{.}%
\item {} 
\sphinxAtStartPar
Use the R subsetting rules to extract the population data for the year 2022 for three countries: Kenya, the
United States of America and Japan. When you have managed to do that you can uses the table
from the last   project to see whether you get the same data for Kenya.

\end{enumerate}

\begin{sphinxuseclass}{cell}\begin{sphinxVerbatimInput}

\begin{sphinxuseclass}{cell_input}
\begin{sphinxVerbatim}[commandchars=\\\{\}]
\PYG{n}{ke\PYGZus{}2022}\PYG{+w}{ }\PYG{o}{\PYGZlt{}\PYGZhy{}}\PYG{+w}{ }\PYG{n}{dat}\PYG{p}{[}\PYG{n}{dat}\PYG{o}{\PYGZdl{}}\PYG{n}{Year}\PYG{+w}{ }\PYG{o}{==}\PYG{+w}{ }\PYG{l+m}{2022}\PYG{+w}{ }\PYG{o}{\PYGZam{}}\PYG{+w}{ }\PYG{n}{dat}\PYG{o}{\PYGZdl{}}\PYG{n}{ISO2}\PYG{+w}{ }\PYG{o}{==}\PYG{+w}{ }\PYG{l+s}{\PYGZdq{}}\PYG{l+s}{KE\PYGZdq{}}\PYG{p}{,}\PYG{+w}{ }\PYG{p}{]}
\PYG{n}{us\PYGZus{}2022}\PYG{+w}{ }\PYG{o}{\PYGZlt{}\PYGZhy{}}\PYG{+w}{ }\PYG{n}{dat}\PYG{p}{[}\PYG{n}{dat}\PYG{o}{\PYGZdl{}}\PYG{n}{Year}\PYG{+w}{ }\PYG{o}{==}\PYG{+w}{ }\PYG{l+m}{2022}\PYG{+w}{ }\PYG{o}{\PYGZam{}}\PYG{+w}{ }\PYG{n}{dat}\PYG{o}{\PYGZdl{}}\PYG{n}{ISO2}\PYG{+w}{ }\PYG{o}{==}\PYG{+w}{ }\PYG{l+s}{\PYGZdq{}}\PYG{l+s}{US\PYGZdq{}}\PYG{p}{,}\PYG{+w}{ }\PYG{p}{]}
\PYG{n}{jp\PYGZus{}2022}\PYG{+w}{ }\PYG{o}{\PYGZlt{}\PYGZhy{}}\PYG{+w}{ }\PYG{n}{dat}\PYG{p}{[}\PYG{n}{dat}\PYG{o}{\PYGZdl{}}\PYG{n}{Year}\PYG{+w}{ }\PYG{o}{==}\PYG{+w}{ }\PYG{l+m}{2022}\PYG{+w}{ }\PYG{o}{\PYGZam{}}\PYG{+w}{ }\PYG{n}{dat}\PYG{o}{\PYGZdl{}}\PYG{n}{ISO2}\PYG{+w}{ }\PYG{o}{==}\PYG{+w}{ }\PYG{l+s}{\PYGZdq{}}\PYG{l+s}{JP\PYGZdq{}}\PYG{p}{,}\PYG{+w}{ }\PYG{p}{]}
\end{sphinxVerbatim}

\end{sphinxuseclass}\end{sphinxVerbatimInput}

\end{sphinxuseclass}
\begin{sphinxuseclass}{cell}\begin{sphinxVerbatimInput}

\begin{sphinxuseclass}{cell_input}
\begin{sphinxVerbatim}[commandchars=\\\{\}]
\PYG{n}{ke\PYGZus{}2022}
\end{sphinxVerbatim}

\end{sphinxuseclass}\end{sphinxVerbatimInput}
\begin{sphinxVerbatimOutput}

\begin{sphinxuseclass}{cell_output}\begin{equation*}
\begin{split}A tibble: 40 × 6
\begin{tabular}{r|llllll}
  & Country & ISO2 & Year & Sex & Age & POP\\
  & <fct> & <fct> & <dbl> & <fct> & <fct> & <dbl>\\
\hline
	612921 & Kenya & KE & 2022 & F & 0-4   & 3487490\\
	612922 & Kenya & KE & 2022 & F & 5-9   & 3404421\\
	612923 & Kenya & KE & 2022 & F & 10-14 & 3444606\\
	612924 & Kenya & KE & 2022 & F & 15-19 & 3225971\\
	612925 & Kenya & KE & 2022 & F & 20-24 & 2656730\\
	612926 & Kenya & KE & 2022 & F & 25-29 & 2139208\\
	612927 & Kenya & KE & 2022 & F & 30-34 & 1946994\\
	612928 & Kenya & KE & 2022 & F & 35-39 & 1849778\\
	612929 & Kenya & KE & 2022 & F & 40-44 & 1606763\\
	612930 & Kenya & KE & 2022 & F & 45-49 & 1170869\\
	612931 & Kenya & KE & 2022 & F & 50-54 &  858998\\
	612932 & Kenya & KE & 2022 & F & 55-59 &  672025\\
	612933 & Kenya & KE & 2022 & F & 60-64 &  516769\\
	612934 & Kenya & KE & 2022 & F & 65-69 &  387773\\
	612935 & Kenya & KE & 2022 & F & 70-74 &  272320\\
	612936 & Kenya & KE & 2022 & F & 75-79 &  168499\\
	612937 & Kenya & KE & 2022 & F & 80-84 &   93724\\
	612938 & Kenya & KE & 2022 & F & 85-89 &   39226\\
	612939 & Kenya & KE & 2022 & F & 90-94 &   10461\\
	612940 & Kenya & KE & 2022 & F & 95-99 &    1520\\
	612941 & Kenya & KE & 2022 & M & 0-4   & 3531278\\
	612942 & Kenya & KE & 2022 & M & 5-9   & 3431166\\
	612943 & Kenya & KE & 2022 & M & 10-14 & 3473103\\
	612944 & Kenya & KE & 2022 & M & 15-19 & 3249738\\
	612945 & Kenya & KE & 2022 & M & 20-24 & 2664966\\
	612946 & Kenya & KE & 2022 & M & 25-29 & 2119949\\
	612947 & Kenya & KE & 2022 & M & 30-34 & 1926219\\
	612948 & Kenya & KE & 2022 & M & 35-39 & 1830280\\
	612949 & Kenya & KE & 2022 & M & 40-44 & 1610246\\
	612950 & Kenya & KE & 2022 & M & 45-49 & 1196519\\
	612951 & Kenya & KE & 2022 & M & 50-54 &  890662\\
	612952 & Kenya & KE & 2022 & M & 55-59 &  679943\\
	612953 & Kenya & KE & 2022 & M & 60-64 &  488074\\
	612954 & Kenya & KE & 2022 & M & 65-69 &  340546\\
	612955 & Kenya & KE & 2022 & M & 70-74 &  229912\\
	612956 & Kenya & KE & 2022 & M & 75-79 &  138357\\
	612957 & Kenya & KE & 2022 & M & 80-84 &   73093\\
	612958 & Kenya & KE & 2022 & M & 85-89 &   28473\\
	612959 & Kenya & KE & 2022 & M & 90-94 &    6918\\
	612960 & Kenya & KE & 2022 & M & 95-99 &     902\\
\end{tabular}\end{split}
\end{equation*}
\end{sphinxuseclass}\end{sphinxVerbatimOutput}

\end{sphinxuseclass}
\sphinxAtStartPar
Comparing with the table shows that the numbers match. Note that we could have referred to the country also by
name “Kenya”, for example. In the case of US we needed to take some care. The US is recorded with the Country variable “United States of America”. This is one string with blanks between the words. To make R treat this string as such, we would have to refer to \sphinxcode{\sphinxupquote{dat\$"United States of America"}}. Otherwise R, in the case without “”, R would read the string as \sphinxcode{\sphinxupquote{dat\$United}} and would not find anything matching. It also would not not what to do with \sphinxcode{\sphinxupquote{States}}, \sphinxcode{\sphinxupquote{of}} and \sphinxcode{\sphinxupquote{America}}.
\begin{enumerate}
\sphinxsetlistlabels{\arabic}{enumi}{enumii}{}{.}%
\setcounter{enumi}{1}
\item {} 
\sphinxAtStartPar
Make a barplot for the total population numbers (i.e. women plus men) for
each age group for all the three countries. This requires you again to use the
R subsetting rules correctly. Describe in your own words what you see as the
main difference between the charts.

\end{enumerate}

\begin{sphinxuseclass}{cell}\begin{sphinxVerbatimInput}

\begin{sphinxuseclass}{cell_input}
\begin{sphinxVerbatim}[commandchars=\\\{\}]
\PYG{n}{numbers\PYGZus{}ke}\PYG{+w}{  }\PYG{o}{\PYGZlt{}\PYGZhy{}}\PYG{+w}{ }\PYG{n}{ke\PYGZus{}2022}\PYG{o}{\PYGZdl{}}\PYG{n}{POP}\PYG{p}{[}\PYG{n}{ke\PYGZus{}2022}\PYG{o}{\PYGZdl{}}\PYG{n}{Sex}\PYG{+w}{ }\PYG{o}{==}\PYG{+w}{ }\PYG{l+s}{\PYGZdq{}}\PYG{l+s}{F\PYGZdq{}}\PYG{p}{]}\PYG{+w}{ }\PYG{o}{+}\PYG{+w}{ }\PYG{n}{ke\PYGZus{}2022}\PYG{o}{\PYGZdl{}}\PYG{n}{POP}\PYG{p}{[}\PYG{n}{ke\PYGZus{}2022}\PYG{o}{\PYGZdl{}}\PYG{n}{Sex}\PYG{+w}{ }\PYG{o}{==}\PYG{+w}{ }\PYG{l+s}{\PYGZdq{}}\PYG{l+s}{M\PYGZdq{}}\PYG{p}{]}
\PYG{n}{numbers\PYGZus{}us}\PYG{+w}{  }\PYG{o}{\PYGZlt{}\PYGZhy{}}\PYG{+w}{ }\PYG{n}{us\PYGZus{}2022}\PYG{o}{\PYGZdl{}}\PYG{n}{POP}\PYG{p}{[}\PYG{n}{us\PYGZus{}2022}\PYG{o}{\PYGZdl{}}\PYG{n}{Sex}\PYG{+w}{ }\PYG{o}{==}\PYG{+w}{ }\PYG{l+s}{\PYGZdq{}}\PYG{l+s}{F\PYGZdq{}}\PYG{p}{]}\PYG{+w}{ }\PYG{o}{+}\PYG{+w}{ }\PYG{n}{us\PYGZus{}2022}\PYG{o}{\PYGZdl{}}\PYG{n}{POP}\PYG{p}{[}\PYG{n}{us\PYGZus{}2022}\PYG{o}{\PYGZdl{}}\PYG{n}{Sex}\PYG{+w}{ }\PYG{o}{==}\PYG{+w}{ }\PYG{l+s}{\PYGZdq{}}\PYG{l+s}{M\PYGZdq{}}\PYG{p}{]}
\PYG{n}{numbers\PYGZus{}jp}\PYG{+w}{  }\PYG{o}{\PYGZlt{}\PYGZhy{}}\PYG{+w}{ }\PYG{n}{jp\PYGZus{}2022}\PYG{o}{\PYGZdl{}}\PYG{n}{POP}\PYG{p}{[}\PYG{n}{jp\PYGZus{}2022}\PYG{o}{\PYGZdl{}}\PYG{n}{Sex}\PYG{+w}{ }\PYG{o}{==}\PYG{+w}{ }\PYG{l+s}{\PYGZdq{}}\PYG{l+s}{F\PYGZdq{}}\PYG{p}{]}\PYG{+w}{ }\PYG{o}{+}\PYG{+w}{ }\PYG{n}{jp\PYGZus{}2022}\PYG{o}{\PYGZdl{}}\PYG{n}{POP}\PYG{p}{[}\PYG{n}{jp\PYGZus{}2022}\PYG{o}{\PYGZdl{}}\PYG{n}{Sex}\PYG{+w}{ }\PYG{o}{==}\PYG{+w}{ }\PYG{l+s}{\PYGZdq{}}\PYG{l+s}{M\PYGZdq{}}\PYG{p}{]}
\end{sphinxVerbatim}

\end{sphinxuseclass}\end{sphinxVerbatimInput}

\end{sphinxuseclass}
\begin{sphinxuseclass}{cell}\begin{sphinxVerbatimInput}

\begin{sphinxuseclass}{cell_input}
\begin{sphinxVerbatim}[commandchars=\\\{\}]
\PYG{n+nf}{barplot}\PYG{p}{(}\PYG{n}{numbers\PYGZus{}ke}\PYG{p}{,}\PYG{+w}{ }\PYG{n}{names.arg}\PYG{+w}{ }\PYG{o}{=}\PYG{+w}{ }\PYG{n+nf}{levels}\PYG{p}{(}\PYG{n}{ke\PYGZus{}2022}\PYG{o}{\PYGZdl{}}\PYG{n}{Age}\PYG{p}{)}\PYG{p}{,}\PYG{+w}{ }\PYG{n}{horiz}\PYG{+w}{ }\PYG{o}{=}\PYG{+w}{ }\PYG{n+nb+bp}{T}\PYG{p}{,}\PYG{+w}{ }\PYG{n}{las}\PYG{+w}{ }\PYG{o}{=}\PYG{+w}{ }\PYG{l+m}{1}\PYG{p}{,}\PYG{+w}{ }
\PYG{+w}{       }\PYG{n}{main}\PYG{+w}{ }\PYG{o}{=}\PYG{+w}{ }\PYG{l+s}{\PYGZdq{}}\PYG{l+s}{Total population of Kenya in 2022 per age group\PYGZdq{}}\PYG{p}{)}
\end{sphinxVerbatim}

\end{sphinxuseclass}\end{sphinxVerbatimInput}
\begin{sphinxVerbatimOutput}

\begin{sphinxuseclass}{cell_output}
\noindent\sphinxincludegraphics{{f7e2e392b8a6b23742cf6eb820fce3768d604417b37f87bcd97574a377e78df1}.png}

\end{sphinxuseclass}\end{sphinxVerbatimOutput}

\end{sphinxuseclass}
\begin{sphinxuseclass}{cell}\begin{sphinxVerbatimInput}

\begin{sphinxuseclass}{cell_input}
\begin{sphinxVerbatim}[commandchars=\\\{\}]
\PYG{n+nf}{barplot}\PYG{p}{(}\PYG{n}{numbers\PYGZus{}us}\PYG{p}{,}\PYG{+w}{ }\PYG{n}{names.arg}\PYG{+w}{ }\PYG{o}{=}\PYG{+w}{ }\PYG{n+nf}{levels}\PYG{p}{(}\PYG{n}{us\PYGZus{}2022}\PYG{o}{\PYGZdl{}}\PYG{n}{Age}\PYG{p}{)}\PYG{p}{,}\PYG{+w}{ }\PYG{n}{horiz}\PYG{+w}{ }\PYG{o}{=}\PYG{+w}{ }\PYG{n+nb+bp}{T}\PYG{p}{,}\PYG{+w}{ }\PYG{n}{las}\PYG{+w}{ }\PYG{o}{=}\PYG{+w}{ }\PYG{l+m}{1}\PYG{p}{,}\PYG{+w}{ }
\PYG{+w}{       }\PYG{n}{main}\PYG{+w}{ }\PYG{o}{=}\PYG{+w}{ }\PYG{l+s}{\PYGZdq{}}\PYG{l+s}{Total population of United States in 2022 per age group\PYGZdq{}}\PYG{p}{)}
\end{sphinxVerbatim}

\end{sphinxuseclass}\end{sphinxVerbatimInput}
\begin{sphinxVerbatimOutput}

\begin{sphinxuseclass}{cell_output}
\noindent\sphinxincludegraphics{{0315fb501e4759ddb30135e7fc232c19f59f52d286f7eed2b38bea66cc8289a7}.png}

\end{sphinxuseclass}\end{sphinxVerbatimOutput}

\end{sphinxuseclass}
\begin{sphinxuseclass}{cell}\begin{sphinxVerbatimInput}

\begin{sphinxuseclass}{cell_input}
\begin{sphinxVerbatim}[commandchars=\\\{\}]
\PYG{n+nf}{barplot}\PYG{p}{(}\PYG{n}{numbers\PYGZus{}jp}\PYG{p}{,}\PYG{+w}{ }\PYG{n}{names.arg}\PYG{+w}{ }\PYG{o}{=}\PYG{+w}{ }\PYG{n+nf}{levels}\PYG{p}{(}\PYG{n}{jp\PYGZus{}2022}\PYG{o}{\PYGZdl{}}\PYG{n}{Age}\PYG{p}{)}\PYG{p}{,}\PYG{+w}{ }\PYG{n}{horiz}\PYG{+w}{ }\PYG{o}{=}\PYG{+w}{ }\PYG{n+nb+bp}{T}\PYG{p}{,}\PYG{+w}{ }\PYG{n}{las}\PYG{+w}{ }\PYG{o}{=}\PYG{+w}{ }\PYG{l+m}{1}\PYG{p}{,}\PYG{+w}{ }
\PYG{+w}{       }\PYG{n}{main}\PYG{+w}{ }\PYG{o}{=}\PYG{+w}{ }\PYG{l+s}{\PYGZdq{}}\PYG{l+s}{Total population of Kenya in 2022 per age group\PYGZdq{}}\PYG{p}{)}
\end{sphinxVerbatim}

\end{sphinxuseclass}\end{sphinxVerbatimInput}
\begin{sphinxVerbatimOutput}

\begin{sphinxuseclass}{cell_output}
\noindent\sphinxincludegraphics{{7b9b1a533adae8f0e0455e44c86af55b3e94da3ea67b31cfcdc49ba10a611530}.png}

\end{sphinxuseclass}\end{sphinxVerbatimOutput}

\end{sphinxuseclass}
\sphinxAtStartPar
Kenya has many young people, the barchart has large bars at the bottom. Demographers call this a bottom layered country. The US has the larges bars in the middle, middle layered and Japan in the top, top layered.
\begin{enumerate}
\sphinxsetlistlabels{\arabic}{enumi}{enumii}{}{.}%
\setcounter{enumi}{2}
\item {} 
\sphinxAtStartPar
Assume we would want to compute the mean age of the population in each country. Think about how you can do this given the data format you encounter here. The problem of the format is that we do not have a list of individual ages where we just have to apply the \sphinxcode{\sphinxupquote{mean()}} function but we just have age groups and counts per group. Think about how we could compute a mean age in this case and compute the mean age using your procedure for the three countries. Comment whether your result is consistent with the barcharts. Hint: When you create a new column, you need for a computation you can always append this new column to your dataframe like this. Assume we have a dataframe \sphinxcode{\sphinxupquote{df}}with variables \sphinxcode{\sphinxupquote{x}} and \sphinxcode{\sphinxupquote{y}}. We would like to add a variable \sphinxcode{\sphinxupquote{z}} which is the sum of \sphinxcode{\sphinxupquote{x}} and \sphinxcode{\sphinxupquote{y}}. In this case we would just write \sphinxcode{\sphinxupquote{df\$z  <\sphinxhyphen{} x + y}} and a new column \sphinxcode{\sphinxupquote{z}} would be appended to \sphinxcode{\sphinxupquote{df}}.

\end{enumerate}

\sphinxAtStartPar
The main idea we need to compute a mean age for the grouped data is to see that the mid point of each age group gives us the mean age in this group. When we take this mean and multiply be the number of people in this age group we get an estimate for the total age in this group, when we assume all members of the group have about the mean
age. So here is the idea. Assume we have:

\begin{sphinxuseclass}{cell}\begin{sphinxVerbatimInput}

\begin{sphinxuseclass}{cell_input}
\begin{sphinxVerbatim}[commandchars=\\\{\}]
\PYG{n}{toy\PYGZus{}age}\PYG{+w}{  }\PYG{o}{\PYGZlt{}\PYGZhy{}}\PYG{+w}{ }\PYG{n+nf}{c}\PYG{p}{(}\PYG{l+m}{0}\PYG{p}{,}\PYG{l+m}{3}\PYG{p}{,}\PYG{l+m}{1}\PYG{p}{,}\PYG{l+m}{1}\PYG{p}{,}\PYG{l+m}{3}\PYG{p}{,}\PYG{l+m}{4}\PYG{p}{,}\PYG{l+m}{2}\PYG{p}{,}\PYG{l+m}{4}\PYG{p}{,}\PYG{l+m}{3}\PYG{p}{,}\PYG{l+m}{1}\PYG{p}{,}\PYG{l+m}{0}\PYG{p}{,}\PYG{l+m}{1}\PYG{p}{)}
\end{sphinxVerbatim}

\end{sphinxuseclass}\end{sphinxVerbatimInput}

\end{sphinxuseclass}
\sphinxAtStartPar
These are all ages in the interval 0 to 4 years. The average age in the list is:

\begin{sphinxuseclass}{cell}\begin{sphinxVerbatimInput}

\begin{sphinxuseclass}{cell_input}
\begin{sphinxVerbatim}[commandchars=\\\{\}]
\PYG{n+nf}{mean}\PYG{p}{(}\PYG{n}{toy\PYGZus{}age}\PYG{p}{)}
\end{sphinxVerbatim}

\end{sphinxuseclass}\end{sphinxVerbatimInput}
\begin{sphinxVerbatimOutput}

\begin{sphinxuseclass}{cell_output}\begin{equation*}
\begin{split}1.91666666666667\end{split}
\end{equation*}
\end{sphinxuseclass}\end{sphinxVerbatimOutput}

\end{sphinxuseclass}
\sphinxAtStartPar
Not exactly 2 but also not too bad. So we do hopefully not make too big of a mistake like this. So lets go to the actual computation using this idea

\begin{sphinxuseclass}{cell}\begin{sphinxVerbatimInput}

\begin{sphinxuseclass}{cell_input}
\begin{sphinxVerbatim}[commandchars=\\\{\}]
\PYG{c+c1}{\PYGZsh{} Step 1: Create a new variable of the age group interval mid points for all age groups. Note that there are many ways}
\PYG{c+c1}{\PYGZsh{} to do this, including creating a variable by hand and adding it. Here we use the sequence function of R}

\PYG{n}{mid}\PYG{+w}{  }\PYG{o}{\PYGZlt{}\PYGZhy{}}\PYG{+w}{ }\PYG{n+nf}{seq}\PYG{p}{(}\PYG{l+m}{2}\PYG{p}{,}\PYG{l+m}{100}\PYG{p}{,}\PYG{l+m}{5}\PYG{p}{)}

\PYG{n}{ke\PYGZus{}2022}\PYG{o}{\PYGZdl{}}\PYG{n}{mid}\PYG{+w}{  }\PYG{o}{\PYGZlt{}\PYGZhy{}}\PYG{+w}{ }\PYG{n}{mid}

\PYG{c+c1}{\PYGZsh{} Step 2: Add up the product between mid and POP for all age groups and divide by the total population}

\PYG{n+nf}{sum}\PYG{p}{(}\PYG{n}{ke\PYGZus{}2022}\PYG{o}{\PYGZdl{}}\PYG{n}{mid}\PYG{o}{*}\PYG{n}{ke\PYGZus{}2022}\PYG{o}{\PYGZdl{}}\PYG{n}{POP}\PYG{p}{)}\PYG{o}{/}\PYG{n+nf}{sum}\PYG{p}{(}\PYG{n}{ke\PYGZus{}2022}\PYG{o}{\PYGZdl{}}\PYG{n}{POP}\PYG{p}{)}
\end{sphinxVerbatim}

\end{sphinxuseclass}\end{sphinxVerbatimInput}
\begin{sphinxVerbatimOutput}

\begin{sphinxuseclass}{cell_output}\begin{equation*}
\begin{split}24.2350859595261\end{split}
\end{equation*}
\end{sphinxuseclass}\end{sphinxVerbatimOutput}

\end{sphinxuseclass}
\sphinxAtStartPar
Kenya has a mean age of 24

\begin{sphinxuseclass}{cell}\begin{sphinxVerbatimInput}

\begin{sphinxuseclass}{cell_input}
\begin{sphinxVerbatim}[commandchars=\\\{\}]
\PYG{n}{us\PYGZus{}2022}\PYG{o}{\PYGZdl{}}\PYG{n}{mid}\PYG{+w}{  }\PYG{o}{\PYGZlt{}\PYGZhy{}}\PYG{+w}{ }\PYG{n}{mid}
\PYG{n+nf}{sum}\PYG{p}{(}\PYG{n}{us\PYGZus{}2022}\PYG{o}{\PYGZdl{}}\PYG{n}{mid}\PYG{o}{*}\PYG{n}{us\PYGZus{}2022}\PYG{o}{\PYGZdl{}}\PYG{n}{POP}\PYG{p}{)}\PYG{o}{/}\PYG{n+nf}{sum}\PYG{p}{(}\PYG{n}{us\PYGZus{}2022}\PYG{o}{\PYGZdl{}}\PYG{n}{POP}\PYG{p}{)}
\end{sphinxVerbatim}

\end{sphinxuseclass}\end{sphinxVerbatimInput}
\begin{sphinxVerbatimOutput}

\begin{sphinxuseclass}{cell_output}\begin{equation*}
\begin{split}39.5162642880378\end{split}
\end{equation*}
\end{sphinxuseclass}\end{sphinxVerbatimOutput}

\end{sphinxuseclass}
\sphinxAtStartPar
The US has a mean age of 40

\begin{sphinxuseclass}{cell}\begin{sphinxVerbatimInput}

\begin{sphinxuseclass}{cell_input}
\begin{sphinxVerbatim}[commandchars=\\\{\}]
\PYG{n}{jp\PYGZus{}2022}\PYG{o}{\PYGZdl{}}\PYG{n}{mid}\PYG{+w}{  }\PYG{o}{\PYGZlt{}\PYGZhy{}}\PYG{+w}{ }\PYG{n}{mid}
\PYG{n+nf}{sum}\PYG{p}{(}\PYG{n}{jp\PYGZus{}2022}\PYG{o}{\PYGZdl{}}\PYG{n}{mid}\PYG{o}{*}\PYG{n}{jp\PYGZus{}2022}\PYG{o}{\PYGZdl{}}\PYG{n}{POP}\PYG{p}{)}\PYG{o}{/}\PYG{n+nf}{sum}\PYG{p}{(}\PYG{n}{jp\PYGZus{}2022}\PYG{o}{\PYGZdl{}}\PYG{n}{POP}\PYG{p}{)}
\end{sphinxVerbatim}

\end{sphinxuseclass}\end{sphinxVerbatimInput}
\begin{sphinxVerbatimOutput}

\begin{sphinxuseclass}{cell_output}\begin{equation*}
\begin{split}47.2826355354766\end{split}
\end{equation*}
\end{sphinxuseclass}\end{sphinxVerbatimOutput}

\end{sphinxuseclass}
\sphinxAtStartPar
Japan has a mean age of 47

\sphinxAtStartPar
This is in line with the bar charts, because the countries have the bulk of people at the bottom, in the middle and at the top of age groups.
\begin{enumerate}
\sphinxsetlistlabels{\arabic}{enumi}{enumii}{}{.}%
\setcounter{enumi}{3}
\item {} 
\sphinxAtStartPar
How does your country compare with these three?

\end{enumerate}

\sphinxAtStartPar
My country is Austria:

\begin{sphinxuseclass}{cell}\begin{sphinxVerbatimInput}

\begin{sphinxuseclass}{cell_input}
\begin{sphinxVerbatim}[commandchars=\\\{\}]
\PYG{n}{at\PYGZus{}2022}\PYG{+w}{ }\PYG{o}{\PYGZlt{}\PYGZhy{}}\PYG{+w}{ }\PYG{n}{dat}\PYG{p}{[}\PYG{n}{dat}\PYG{o}{\PYGZdl{}}\PYG{n}{Year}\PYG{+w}{ }\PYG{o}{==}\PYG{+w}{ }\PYG{l+m}{2022}\PYG{+w}{ }\PYG{o}{\PYGZam{}}\PYG{+w}{ }\PYG{n}{dat}\PYG{o}{\PYGZdl{}}\PYG{n}{ISO2}\PYG{+w}{ }\PYG{o}{==}\PYG{+w}{ }\PYG{l+s}{\PYGZdq{}}\PYG{l+s}{AT\PYGZdq{}}\PYG{p}{,}\PYG{+w}{ }\PYG{p}{]}

\PYG{n}{numbers\PYGZus{}at}\PYG{+w}{  }\PYG{o}{\PYGZlt{}\PYGZhy{}}\PYG{+w}{ }\PYG{n}{at\PYGZus{}2022}\PYG{o}{\PYGZdl{}}\PYG{n}{POP}\PYG{p}{[}\PYG{n}{at\PYGZus{}2022}\PYG{o}{\PYGZdl{}}\PYG{n}{Sex}\PYG{+w}{ }\PYG{o}{==}\PYG{+w}{ }\PYG{l+s}{\PYGZdq{}}\PYG{l+s}{F\PYGZdq{}}\PYG{p}{]}\PYG{+w}{ }\PYG{o}{+}\PYG{+w}{ }\PYG{n}{at\PYGZus{}2022}\PYG{o}{\PYGZdl{}}\PYG{n}{POP}\PYG{p}{[}\PYG{n}{at\PYGZus{}2022}\PYG{o}{\PYGZdl{}}\PYG{n}{Sex}\PYG{+w}{ }\PYG{o}{==}\PYG{+w}{ }\PYG{l+s}{\PYGZdq{}}\PYG{l+s}{M\PYGZdq{}}\PYG{p}{]}

\PYG{n+nf}{barplot}\PYG{p}{(}\PYG{n}{numbers\PYGZus{}at}\PYG{p}{,}\PYG{+w}{ }\PYG{n}{names.arg}\PYG{+w}{ }\PYG{o}{=}\PYG{+w}{ }\PYG{n+nf}{levels}\PYG{p}{(}\PYG{n}{at\PYGZus{}2022}\PYG{o}{\PYGZdl{}}\PYG{n}{Age}\PYG{p}{)}\PYG{p}{,}\PYG{+w}{ }\PYG{n}{horiz}\PYG{+w}{ }\PYG{o}{=}\PYG{+w}{ }\PYG{n+nb+bp}{T}\PYG{p}{,}\PYG{+w}{ }\PYG{n}{las}\PYG{+w}{ }\PYG{o}{=}\PYG{+w}{ }\PYG{l+m}{1}\PYG{p}{,}\PYG{+w}{ }
\PYG{+w}{       }\PYG{n}{main}\PYG{+w}{ }\PYG{o}{=}\PYG{+w}{ }\PYG{l+s}{\PYGZdq{}}\PYG{l+s}{Total population of Austria in 2022 per age group\PYGZdq{}}\PYG{p}{)}
\end{sphinxVerbatim}

\end{sphinxuseclass}\end{sphinxVerbatimInput}
\begin{sphinxVerbatimOutput}

\begin{sphinxuseclass}{cell_output}
\noindent\sphinxincludegraphics{{1da23dd101f8befef7acbfb9ea40595563782452a1515e54eed9a04108ac7a0d}.png}

\end{sphinxuseclass}\end{sphinxVerbatimOutput}

\end{sphinxuseclass}
\begin{sphinxuseclass}{cell}\begin{sphinxVerbatimInput}

\begin{sphinxuseclass}{cell_input}
\begin{sphinxVerbatim}[commandchars=\\\{\}]
\PYG{n}{mid}\PYG{+w}{  }\PYG{o}{\PYGZlt{}\PYGZhy{}}\PYG{+w}{ }\PYG{n+nf}{seq}\PYG{p}{(}\PYG{l+m}{2}\PYG{p}{,}\PYG{l+m}{100}\PYG{p}{,}\PYG{l+m}{5}\PYG{p}{)}

\PYG{n}{at\PYGZus{}2022}\PYG{o}{\PYGZdl{}}\PYG{n}{mid}\PYG{+w}{  }\PYG{o}{\PYGZlt{}\PYGZhy{}}\PYG{+w}{ }\PYG{n}{mid}

\PYG{n+nf}{sum}\PYG{p}{(}\PYG{n}{at\PYGZus{}2022}\PYG{o}{\PYGZdl{}}\PYG{n}{mid}\PYG{o}{*}\PYG{n}{at\PYGZus{}2022}\PYG{o}{\PYGZdl{}}\PYG{n}{POP}\PYG{p}{)}\PYG{o}{/}\PYG{n+nf}{sum}\PYG{p}{(}\PYG{n}{at\PYGZus{}2022}\PYG{o}{\PYGZdl{}}\PYG{n}{POP}\PYG{p}{)}
\end{sphinxVerbatim}

\end{sphinxuseclass}\end{sphinxVerbatimInput}
\begin{sphinxVerbatimOutput}

\begin{sphinxuseclass}{cell_output}\begin{equation*}
\begin{split}43.430240348906\end{split}
\end{equation*}
\end{sphinxuseclass}\end{sphinxVerbatimOutput}

\end{sphinxuseclass}
\begin{sphinxuseclass}{cell}\begin{sphinxVerbatimInput}

\begin{sphinxuseclass}{cell_input}
\begin{sphinxVerbatim}[commandchars=\\\{\}]
\PYG{n}{Austria}\PYG{+w}{ }\PYG{n}{is}\PYG{+w}{ }\PYG{n}{between}\PYG{+w}{ }\PYG{n}{the}\PYG{+w}{ }\PYG{n}{US}\PYG{+w}{ }\PYG{n}{and}\PYG{+w}{ }\PYG{n}{Japan}\PYG{+w}{ }\PYG{n}{and}\PYG{+w}{ }\PYG{n}{tends}\PYG{+w}{ }\PYG{n}{towards}\PYG{+w}{ }\PYG{n}{a}\PYG{+w}{ }\PYG{n}{top}\PYG{+w}{ }\PYG{n}{layered}\PYG{+w}{ }\PYG{n}{country.}
\end{sphinxVerbatim}

\end{sphinxuseclass}\end{sphinxVerbatimInput}
\begin{sphinxVerbatimOutput}

\begin{sphinxuseclass}{cell_output}
\begin{sphinxVerbatim}[commandchars=\\\{\}]
\PYG{n}{Error} \PYG{o+ow}{in} \PYG{n}{parse}\PYG{p}{(}\PYG{n}{text} \PYG{o}{=} \PYG{n}{x}\PYG{p}{,} \PYG{n}{srcfile} \PYG{o}{=} \PYG{n}{src}\PYG{p}{)}\PYG{p}{:} \PYG{o}{\PYGZlt{}}\PYG{n}{text}\PYG{o}{\PYGZgt{}}\PYG{p}{:}\PYG{l+m+mi}{1}\PYG{p}{:}\PYG{l+m+mi}{9}\PYG{p}{:} \PYG{n}{unexpected} \PYG{n}{symbol}
\PYG{l+m+mi}{1}\PYG{p}{:} \PYG{n}{Austria} \PYG{o+ow}{is}
            \PYG{o}{\PYGZca{}}
\PYG{n+ne}{Traceback}:
\end{sphinxVerbatim}

\end{sphinxuseclass}\end{sphinxVerbatimOutput}

\end{sphinxuseclass}
\sphinxstepscope


\chapter{Exercises: Unit 3, Generalizing observations from data and knowing what causes what}
\label{\detokenize{exercises_unit_3:exercises-unit-3-generalizing-observations-from-data-and-knowing-what-causes-what}}\label{\detokenize{exercises_unit_3::doc}}

\section{Exercises}
\label{\detokenize{exercises_unit_3:exercises}}

\subsection{Exercise 1: Understanding the notions of sample, target population and study population}
\label{\detokenize{exercises_unit_3:exercise-1-understanding-the-notions-of-sample-target-population-and-study-population}}
\sphinxAtStartPar
In the first case the sample are a randomly choosen set of areas in the entirle land that can be used for the crop, which is the study population. The target population would be the entire areas of land that could be planted by such a crop. In the second example, a sample would be a selected set of schools in the jurisdiction of the ministry of education, the study population would be the set of all active schools in the jurisdiction of the ministry of eduction. The target population would be all schools that are and will be operated in this jurisdiction. In the third case a slected subset of customers from all potential customers is the sample, the study population is the large set of customers from which we can take a sample. The target population is all customers in a given market.


\subsection{Exercise 2: Examples of deductive and inductive reasoning}
\label{\detokenize{exercises_unit_3:exercise-2-examples-of-deductive-and-inductive-reasoning}}
\sphinxAtStartPar
Deductive:

\sphinxAtStartPar
Premise 1: All humans require oxygen to survive.

\sphinxAtStartPar
Premise 2: Mary is a human.

\sphinxAtStartPar
Conclusion: Therefore, Mary requires oxygen to survive.

\sphinxAtStartPar
In this deductive reasoning example:

\sphinxAtStartPar
Premise 1 represents a general principle or universal truth about humans needing oxygen.
Premise 2 is a specific statement about Mary being a human.
The conclusion logically follows from the premises, stating that Mary, being a human, must require oxygen to survive.

\sphinxAtStartPar
Inductive:

\sphinxAtStartPar
Observation: In Kenya, a study is conducted on the mobile phone usage patterns of the population, specifically focusing on individuals between the ages of 18 and 35.

\sphinxAtStartPar
Pattern Identification: The study reveals that a significant majority of young adults in Kenya use mobile phones primarily for accessing social media, online education platforms, and mobile banking services. Additionally, a notable portion of this demographic prefers smartphones with larger screens for better multimedia experiences.

\sphinxAtStartPar
Generalization: Based on the observed pattern, one could make an inductive inference that, in general, the younger population in Kenya values smartphones as versatile tools for social connectivity, education, and financial transactions. Furthermore, there is a preference for devices with larger screens.

\sphinxAtStartPar
Conclusion: Therefore, businesses, mobile phone manufacturers, and service providers operating in Kenya might consider tailoring their products and services to align with the preferences of the younger demographic, such as offering affordable smartphones with enhanced multimedia capabilities and promoting apps related to social networking, education, and mobile banking.

\sphinxAtStartPar
In this example, the inductive reasoning process starts with specific observations about the mobile phone habits of young adults in Kenya and then generalizes to make broader conclusions that could inform decisions in industries related to technology and telecommunications within the country.


\subsection{Exercise 3: Conversion to standard units with pencil and paper}
\label{\detokenize{exercises_unit_3:exercise-3-conversion-to-standard-units-with-pencil-and-paper}}
\sphinxAtStartPar
1\sphinxhyphen{}a: 1, \sphinxhyphen{}1.5, 2.5, 1\sphinxhyphen{}b: 50, 35, \sphinxhyphen{}55, 78
2: 1.55, \sphinxhyphen{}0.52, 0.52, \sphinxhyphen{}1.55, 0. The mean is 0 and the sd is 1.


\subsection{Exercise 4: The polio vaccine study}
\label{\detokenize{exercises_unit_3:exercise-4-the-polio-vaccine-study}}
\sphinxAtStartPar
a) From line 1 of the table, the polio rates in the two vaccine groups were about the same. If, for example, the consent group had been richer, their rate would have been higher.

\sphinxAtStartPar
b) From line 3 of the table, the polio rates in the two no\sphinxhyphen{}consent groups were about the same.

\sphinxAtStartPar
c) From line 2 of the table the polio rate in the NFIP control group was quite a bit lower than the rate in the other control group.

\sphinxAtStartPar
d) The no\sphinxhyphen{}consent group is predominantly lower\sphinxhyphen{}income, and the children are more resistant to polio. The NFIP control group has a range of incomes including the more vulnerable children from the higher income families.

\sphinxAtStartPar
e) The ones who consent are different from the ones who don’t.


\subsection{Exercise 5: Did the vaccine field trial cause children to get polio?}
\label{\detokenize{exercises_unit_3:exercise-5-did-the-vaccine-field-trial-cause-children-to-get-polio}}
\sphinxAtStartPar
No, because the experimental areas were selected in those parts of the country most at risk from polio.


\subsection{Exercise 6: The effects of smoking on health}
\label{\detokenize{exercises_unit_3:exercise-6-the-effects-of-smoking-on-health}}
\sphinxAtStartPar
a) Because the groups differ systematically with respect to body and behavorial traits
b) No because the current smokers are the survivors of a cohort of smokers many of whom have already died as a consequence of smoking.


\subsection{Exercise 7: Evaluating a prisoner’s rehab program}
\label{\detokenize{exercises_unit_3:exercise-7-evaluating-a-prisoner-s-rehab-program}}
\sphinxAtStartPar
a) The treatment group are the prisoners who sign up for the bootcamp. b) It is not a randomized controlled experiment because the selection into treatment and control is not random. It is an observational study. c) False, no comparison possible since no randomized controlled trial. The two groups compared differ systematically.


\section{Exercises R}
\label{\detokenize{exercises_unit_3:exercises-r}}

\subsection{Exercise 1: Select every 100th row from the height data}
\label{\detokenize{exercises_unit_3:exercise-1-select-every-100th-row-from-the-height-data}}
\begin{sphinxuseclass}{cell}\begin{sphinxVerbatimInput}

\begin{sphinxuseclass}{cell_input}
\begin{sphinxVerbatim}[commandchars=\\\{\}]
\PYG{n+nf}{library}\PYG{p}{(}\PYG{n}{JWL}\PYG{p}{)}
\PYG{n}{dat}\PYG{+w}{  }\PYG{o}{\PYGZlt{}\PYGZhy{}}\PYG{+w}{ }\PYG{n}{socr\PYGZus{}height\PYGZus{}weight}
\end{sphinxVerbatim}

\end{sphinxuseclass}\end{sphinxVerbatimInput}

\end{sphinxuseclass}
\begin{sphinxuseclass}{cell}\begin{sphinxVerbatimInput}

\begin{sphinxuseclass}{cell_input}
\begin{sphinxVerbatim}[commandchars=\\\{\}]
\PYG{n}{idx}\PYG{+w}{  }\PYG{o}{\PYGZlt{}\PYGZhy{}}\PYG{+w}{ }\PYG{n+nf}{seq}\PYG{p}{(}\PYG{n}{from}\PYG{+w}{ }\PYG{o}{=}\PYG{+w}{ }\PYG{l+m}{1}\PYG{p}{,}\PYG{+w}{ }\PYG{n}{to}\PYG{+w}{ }\PYG{o}{=}\PYG{+w}{ }\PYG{n+nf}{nrow}\PYG{p}{(}\PYG{n}{dat}\PYG{p}{)}\PYG{p}{,}\PYG{+w}{ }\PYG{n}{by}\PYG{+w}{ }\PYG{o}{=}\PYG{+w}{ }\PYG{l+m}{100}\PYG{p}{)}
\PYG{n}{sample\PYGZus{}100th}\PYG{+w}{  }\PYG{o}{\PYGZlt{}\PYGZhy{}}\PYG{+w}{ }\PYG{n}{dat}\PYG{p}{[}\PYG{n}{idx}\PYG{p}{,}\PYG{+w}{ }\PYG{p}{]}
\PYG{n}{sample\PYGZus{}100th}
\end{sphinxVerbatim}

\end{sphinxuseclass}\end{sphinxVerbatimInput}
\begin{sphinxVerbatimOutput}

\begin{sphinxuseclass}{cell_output}\begin{equation*}
\begin{split}A data.frame: 250 × 3
\begin{tabular}{r|lll}
  & Index & Height & Weight\\
  & <dbl> & <dbl> & <dbl>\\
\hline
	2 &    1 & 65.78331 & 112.9925\\
	102 &  101 & 64.87434 & 102.0927\\
	202 &  201 & 65.73160 & 121.4997\\
	302 &  301 & 69.07739 & 138.7307\\
	402 &  401 & 67.02877 & 125.4542\\
	502 &  501 & 68.68038 & 120.6936\\
	602 &  601 & 65.87671 & 134.5173\\
	702 &  701 & 70.59020 & 152.9146\\
	802 &  801 & 67.42782 & 123.4642\\
	902 &  901 & 66.58181 & 127.9357\\
	1002 & 1001 & 70.05146 & 134.6655\\
	1102 & 1101 & 65.88645 & 129.3750\\
	1202 & 1201 & 71.33959 & 130.2482\\
	1302 & 1301 & 66.40067 & 124.0382\\
	1402 & 1401 & 66.72267 & 118.0513\\
	1502 & 1501 & 71.65878 & 130.4121\\
	1602 & 1601 & 68.51270 & 116.4033\\
	1702 & 1701 & 69.18211 & 138.7521\\
	1802 & 1801 & 70.68625 & 139.1153\\
	1902 & 1901 & 69.87338 & 140.1507\\
	2002 & 2001 & 69.00812 & 125.4763\\
	2102 & 2101 & 66.60747 & 125.3684\\
	2202 & 2201 & 68.47617 & 136.7370\\
	2302 & 2301 & 67.69345 & 129.6252\\
	2402 & 2401 & 69.43735 & 133.5447\\
	2502 & 2501 & 65.72097 & 121.2620\\
	2602 & 2601 & 66.64128 & 114.0856\\
	2702 & 2701 & 69.75148 & 147.7342\\
	2802 & 2801 & 68.04613 & 118.4149\\
	2902 & 2901 & 67.61783 & 125.5300\\
	⋮ & ⋮ & ⋮ & ⋮\\
	22002 & 22001 & 67.45991 & 131.0766\\
	22102 & 22101 & 67.55783 & 117.3321\\
	22202 & 22201 & 69.62638 & 141.8247\\
	22302 & 22301 & 66.34329 & 135.2088\\
	22402 & 22401 & 65.39474 & 118.9921\\
	22502 & 22501 & 65.12040 & 118.9654\\
	22602 & 22601 & 66.80934 & 127.5672\\
	22702 & 22701 & 67.37224 & 135.8829\\
	22802 & 22801 & 68.62226 & 119.5030\\
	22902 & 22901 & 70.30626 & 127.5549\\
	23002 & 23001 & 66.03227 & 105.2789\\
	23102 & 23101 & 67.16624 & 108.0636\\
	23202 & 23201 & 69.92772 & 136.1304\\
	23302 & 23301 & 67.95420 & 120.0739\\
	23402 & 23401 & 69.02855 & 137.7087\\
	23502 & 23501 & 69.66929 & 126.4905\\
	23602 & 23601 & 69.83869 & 146.7883\\
	23702 & 23701 & 66.25028 & 122.8633\\
	23802 & 23801 & 66.94796 & 113.8070\\
	23902 & 23901 & 66.82731 & 118.8480\\
	24002 & 24001 & 71.05722 & 129.2659\\
	24102 & 24101 & 71.87029 & 151.8397\\
	24202 & 24201 & 69.88715 & 147.5768\\
	24302 & 24301 & 67.57486 & 136.4100\\
	24402 & 24401 & 67.70655 & 136.4734\\
	24502 & 24501 & 67.87052 & 128.4015\\
	24602 & 24601 & 67.67595 & 123.2158\\
	24702 & 24701 & 66.91088 & 107.3063\\
	24802 & 24801 & 69.44092 & 144.8549\\
	24902 & 24901 & 65.25580 & 116.3788\\
\end{tabular}\end{split}
\end{equation*}
\end{sphinxuseclass}\end{sphinxVerbatimOutput}

\end{sphinxuseclass}

\subsection{Exercise 2: Sampling numbers from a list or numbers at random}
\label{\detokenize{exercises_unit_3:exercise-2-sampling-numbers-from-a-list-or-numbers-at-random}}
\begin{sphinxuseclass}{cell}\begin{sphinxVerbatimInput}

\begin{sphinxuseclass}{cell_input}
\begin{sphinxVerbatim}[commandchars=\\\{\}]
\PYG{n}{list\PYGZus{}of\PYGZus{}numbers}\PYG{+w}{  }\PYG{o}{\PYGZlt{}\PYGZhy{}}\PYG{+w}{ }\PYG{l+m}{1}\PYG{o}{:}\PYG{l+m}{24}
\PYG{n+nf}{sample}\PYG{p}{(}\PYG{n}{list\PYGZus{}of\PYGZus{}numbers}\PYG{p}{,}\PYG{+w}{ }\PYG{n}{size}\PYG{+w}{ }\PYG{o}{=}\PYG{+w}{ }\PYG{l+m}{8}\PYG{p}{)}
\end{sphinxVerbatim}

\end{sphinxuseclass}\end{sphinxVerbatimInput}
\begin{sphinxVerbatimOutput}

\begin{sphinxuseclass}{cell_output}\begin{equation*}
\begin{split}\begin{enumerate*}
\item 24
\item 7
\item 12
\item 4
\item 22
\item 15
\item 13
\item 8
\end{enumerate*}\end{split}
\end{equation*}
\end{sphinxuseclass}\end{sphinxVerbatimOutput}

\end{sphinxuseclass}
\sphinxAtStartPar
Note this is just one random draw, if we draw again the numbers will look different and when you took a random sample the specific list chosen will look yet different again.


\subsection{Exercise 3: Sample mean and sample standard deviation}
\label{\detokenize{exercises_unit_3:exercise-3-sample-mean-and-sample-standard-deviation}}
\begin{sphinxuseclass}{cell}\begin{sphinxVerbatimInput}

\begin{sphinxuseclass}{cell_input}
\begin{sphinxVerbatim}[commandchars=\\\{\}]
\PYG{c+c1}{\PYGZsh{} The Height data in cm}

\PYG{n}{data}\PYG{+w}{  }\PYG{o}{\PYGZlt{}\PYGZhy{}}\PYG{+w}{ }\PYG{n}{socr\PYGZus{}height\PYGZus{}weight}

\PYG{n}{emp\PYGZus{}distr\PYGZus{}height}\PYG{+w}{  }\PYG{o}{\PYGZlt{}\PYGZhy{}}\PYG{+w}{ }\PYG{n+nf}{function}\PYG{p}{(}\PYG{n}{n}\PYG{p}{)}\PYG{p}{\PYGZob{}}
\PYG{+w}{    }
\PYG{+w}{    }\PYG{n}{data}\PYG{p}{[}\PYG{n+nf}{sample}\PYG{p}{(}\PYG{n}{x}\PYG{+w}{ }\PYG{o}{=}\PYG{+w}{ }\PYG{l+m}{1}\PYG{o}{:}\PYG{n+nf}{nrow}\PYG{p}{(}\PYG{n}{data}\PYG{p}{)}\PYG{p}{,}\PYG{+w}{ }\PYG{n}{size}\PYG{+w}{ }\PYG{o}{=}\PYG{+w}{ }\PYG{n}{n}\PYG{p}{,}\PYG{+w}{ }\PYG{n}{replace}\PYG{+w}{ }\PYG{o}{=}\PYG{+w}{ }\PYG{n+nb+bp}{T}\PYG{p}{)}\PYG{p}{,}\PYG{+w}{ }\PYG{p}{]}
\PYG{+w}{    }
\PYG{p}{\PYGZcb{}}

\PYG{n}{sample\PYGZus{}dat}\PYG{+w}{  }\PYG{o}{\PYGZlt{}\PYGZhy{}}\PYG{+w}{ }\PYG{n+nf}{emp\PYGZus{}distr\PYGZus{}height}\PYG{p}{(}\PYG{n}{n}\PYG{+w}{ }\PYG{o}{=}\PYG{+w}{ }\PYG{l+m}{5000}\PYG{p}{)}

\PYG{c+c1}{\PYGZsh{} We convert the units to cm}

\PYG{n+nf}{mean}\PYG{p}{(}\PYG{n}{sample\PYGZus{}dat}\PYG{o}{\PYGZdl{}}\PYG{n}{Height}\PYG{o}{*}\PYG{l+m}{2.54}\PYG{p}{)}
\PYG{n+nf}{sd}\PYG{p}{(}\PYG{n}{sample\PYGZus{}dat}\PYG{o}{\PYGZdl{}}\PYG{n}{Height}\PYG{o}{*}\PYG{l+m}{2.54}\PYG{p}{)}
\end{sphinxVerbatim}

\end{sphinxuseclass}\end{sphinxVerbatimInput}
\begin{sphinxVerbatimOutput}

\begin{sphinxuseclass}{cell_output}\begin{equation*}
\begin{split}172.60294609644\end{split}
\end{equation*}\begin{equation*}
\begin{split}4.8792281797397\end{split}
\end{equation*}
\end{sphinxuseclass}\end{sphinxVerbatimOutput}

\end{sphinxuseclass}
\sphinxAtStartPar
When we use the full sample we get:

\begin{sphinxuseclass}{cell}\begin{sphinxVerbatimInput}

\begin{sphinxuseclass}{cell_input}
\begin{sphinxVerbatim}[commandchars=\\\{\}]
\PYG{n+nf}{mean}\PYG{p}{(}\PYG{n}{data}\PYG{o}{\PYGZdl{}}\PYG{n}{Height}\PYG{o}{*}\PYG{l+m}{2.54}\PYG{p}{)}
\PYG{n+nf}{sd}\PYG{p}{(}\PYG{n}{data}\PYG{o}{\PYGZdl{}}\PYG{n}{Height}\PYG{o}{*}\PYG{l+m}{2.54}\PYG{p}{)}
\end{sphinxVerbatim}

\end{sphinxuseclass}\end{sphinxVerbatimInput}
\begin{sphinxVerbatimOutput}

\begin{sphinxuseclass}{cell_output}\begin{equation*}
\begin{split}172.702508535872\end{split}
\end{equation*}\begin{equation*}
\begin{split}4.83026407886225\end{split}
\end{equation*}
\end{sphinxuseclass}\end{sphinxVerbatimOutput}

\end{sphinxuseclass}
\sphinxAtStartPar
The sample is already pretty close to the true parameter values in the population.


\subsection{Exercise 4: Rolling two dice}
\label{\detokenize{exercises_unit_3:exercise-4-rolling-two-dice}}
\begin{sphinxuseclass}{cell}\begin{sphinxVerbatimInput}

\begin{sphinxuseclass}{cell_input}
\begin{sphinxVerbatim}[commandchars=\\\{\}]
\PYG{n}{roll}\PYG{+w}{ }\PYG{o}{\PYGZlt{}\PYGZhy{}}\PYG{+w}{ }\PYG{n+nf}{function}\PYG{p}{(}\PYG{p}{)}\PYG{p}{\PYGZob{}}
\PYG{+w}{  }
\PYG{+w}{  }\PYG{c+c1}{\PYGZsh{} create a die}
\PYG{+w}{  }
\PYG{+w}{  }\PYG{n}{die}\PYG{+w}{ }\PYG{o}{\PYGZlt{}\PYGZhy{}}\PYG{+w}{ }\PYG{l+m}{1}\PYG{o}{:}\PYG{l+m}{6}
\PYG{+w}{  }
\PYG{+w}{  }\PYG{c+c1}{\PYGZsh{} roll the dice by making use of sample. We roll two dice, therefore size has to be equal to 2}
\PYG{+w}{  }\PYG{c+c1}{\PYGZsh{} note that we need to set replace to TRUE because both dice should be able to show all possible}
\PYG{+w}{  }\PYG{c+c1}{\PYGZsh{} points}
\PYG{+w}{  }
\PYG{+w}{  }\PYG{n}{dice}\PYG{+w}{ }\PYG{o}{\PYGZlt{}\PYGZhy{}}\PYG{+w}{ }\PYG{n+nf}{sample}\PYG{p}{(}\PYG{n}{die}\PYG{p}{,}\PYG{+w}{ }\PYG{n}{size}\PYG{+w}{ }\PYG{o}{=}\PYG{+w}{ }\PYG{l+m}{2}\PYG{p}{,}\PYG{+w}{ }\PYG{n}{replace}\PYG{+w}{ }\PYG{o}{=}\PYG{+w}{ }\PYG{k+kc}{TRUE}\PYG{p}{)}
\PYG{+w}{  }
\PYG{+w}{  }\PYG{c+c1}{\PYGZsh{} sum up the points of the two dice}
\PYG{+w}{  }
\PYG{+w}{  }\PYG{n+nf}{sum}\PYG{p}{(}\PYG{n}{dice}\PYG{p}{)}
\PYG{+w}{  }
\PYG{p}{\PYGZcb{}}
\end{sphinxVerbatim}

\end{sphinxuseclass}\end{sphinxVerbatimInput}

\end{sphinxuseclass}
\sphinxAtStartPar
Comment: Note that when you choose to sample twice from the points 1 \sphinxhyphen{} 6 using the \sphinxcode{\sphinxupquote{sample()}}
function, you need to set the replace argument to \sphinxcode{\sphinxupquote{TRUE}}. Check the notes or the help for
sample if you need to look up what this means. If replace is set to \sphinxcode{\sphinxupquote{FALSE}} \sphinxhyphen{} as is the default with sample \sphinxhyphen{} then you have a situation where it is impossible that when the first die shows
a particular number, the second dice shows it as well. This is clearly not what we have in mind
if we think of dice throws. Of course we want to allow, for example, that both dice ,may show 6.

\begin{sphinxuseclass}{cell}\begin{sphinxVerbatimInput}

\begin{sphinxuseclass}{cell_input}
\begin{sphinxVerbatim}[commandchars=\\\{\}]
\PYG{n}{rolls}\PYG{+w}{ }\PYG{o}{\PYGZlt{}\PYGZhy{}}\PYG{+w}{ }\PYG{n+nf}{replicate}\PYG{p}{(}\PYG{l+m}{10000}\PYG{p}{,}\PYG{+w}{ }\PYG{n+nf}{roll}\PYG{p}{(}\PYG{p}{)}\PYG{p}{)}
\end{sphinxVerbatim}

\end{sphinxuseclass}\end{sphinxVerbatimInput}

\end{sphinxuseclass}
\begin{sphinxuseclass}{cell}\begin{sphinxVerbatimInput}

\begin{sphinxuseclass}{cell_input}
\begin{sphinxVerbatim}[commandchars=\\\{\}]
\PYG{n+nf}{hist}\PYG{p}{(}\PYG{n}{rolls}\PYG{p}{)}
\end{sphinxVerbatim}

\end{sphinxuseclass}\end{sphinxVerbatimInput}
\begin{sphinxVerbatimOutput}

\begin{sphinxuseclass}{cell_output}
\noindent\sphinxincludegraphics{{3e05a3d75f70aadeb25e456504994352b7296ed936b4aba52d7e85f4da339943}.png}

\end{sphinxuseclass}\end{sphinxVerbatimOutput}

\end{sphinxuseclass}
\sphinxAtStartPar
There is no visible bias in the outcomes. The dice seem fair.


\subsection{Exercise 5: Conversion into standard units}
\label{\detokenize{exercises_unit_3:exercise-5-conversion-into-standard-units}}
\begin{sphinxuseclass}{cell}\begin{sphinxVerbatimInput}

\begin{sphinxuseclass}{cell_input}
\begin{sphinxVerbatim}[commandchars=\\\{\}]
\PYG{n}{val}\PYG{+w}{  }\PYG{o}{\PYGZlt{}\PYGZhy{}}\PYG{+w}{ }\PYG{n+nf}{c}\PYG{p}{(}\PYG{l+m}{175.9599}\PYG{p}{,}\PYG{+w}{ }\PYG{l+m}{165.3240}\PYG{p}{,}\PYG{+w}{ }\PYG{l+m}{169.0384}\PYG{p}{)}

\PYG{n}{z}\PYG{+w}{  }\PYG{o}{\PYGZlt{}\PYGZhy{}}\PYG{+w}{ }\PYG{p}{(}\PYG{n}{val}\PYG{+w}{ }\PYG{o}{\PYGZhy{}}\PYG{+w}{ }\PYG{n+nf}{mean}\PYG{p}{(}\PYG{n}{val}\PYG{p}{)}\PYG{p}{)}\PYG{o}{/}\PYG{p}{(}\PYG{n+nf}{sd}\PYG{p}{(}\PYG{n}{val}\PYG{p}{)}\PYG{o}{*}\PYG{n+nf}{sqrt}\PYG{p}{(}\PYG{l+m}{2}\PYG{o}{/}\PYG{l+m}{3}\PYG{p}{)}\PYG{p}{)}

\PYG{n}{z}
\end{sphinxVerbatim}

\end{sphinxuseclass}\end{sphinxVerbatimInput}
\begin{sphinxVerbatimOutput}

\begin{sphinxuseclass}{cell_output}\begin{equation*}
\begin{split}\begin{enumerate*}
\item 1.32787386211651
\item -1.08531948258458
\item -0.242554379531931
\end{enumerate*}\end{split}
\end{equation*}
\end{sphinxuseclass}\end{sphinxVerbatimOutput}

\end{sphinxuseclass}
\begin{sphinxuseclass}{cell}\begin{sphinxVerbatimInput}

\begin{sphinxuseclass}{cell_input}
\begin{sphinxVerbatim}[commandchars=\\\{\}]
\PYG{n+nf}{mean}\PYG{p}{(}\PYG{n}{z}\PYG{p}{)}
\PYG{n+nf}{sd}\PYG{p}{(}\PYG{n}{z}\PYG{p}{)}\PYG{o}{*}\PYG{n+nf}{sqrt}\PYG{p}{(}\PYG{l+m}{2}\PYG{o}{/}\PYG{l+m}{3}\PYG{p}{)}
\end{sphinxVerbatim}

\end{sphinxuseclass}\end{sphinxVerbatimInput}
\begin{sphinxVerbatimOutput}

\begin{sphinxuseclass}{cell_output}\begin{equation*}
\begin{split}-2.14645376848723e-15\end{split}
\end{equation*}\begin{equation*}
\begin{split}1\end{split}
\end{equation*}
\end{sphinxuseclass}\end{sphinxVerbatimOutput}

\end{sphinxuseclass}

\subsection{Exercise 6: Find the area under the normal curve}
\label{\detokenize{exercises_unit_3:exercise-6-find-the-area-under-the-normal-curve}}
\begin{sphinxuseclass}{cell}\begin{sphinxVerbatimInput}

\begin{sphinxuseclass}{cell_input}
\begin{sphinxVerbatim}[commandchars=\\\{\}]
\PYG{l+m}{1}\PYG{+w}{ }\PYG{o}{\PYGZhy{}}\PYG{+w}{ }\PYG{n+nf}{pnorm}\PYG{p}{(}\PYG{l+m}{1.25}\PYG{p}{)}
\end{sphinxVerbatim}

\end{sphinxuseclass}\end{sphinxVerbatimInput}
\begin{sphinxVerbatimOutput}

\begin{sphinxuseclass}{cell_output}\begin{equation*}
\begin{split}0.105649773666855\end{split}
\end{equation*}
\end{sphinxuseclass}\end{sphinxVerbatimOutput}

\end{sphinxuseclass}
\begin{sphinxuseclass}{cell}\begin{sphinxVerbatimInput}

\begin{sphinxuseclass}{cell_input}
\begin{sphinxVerbatim}[commandchars=\\\{\}]
\PYG{n+nf}{pnorm}\PYG{p}{(}\PYG{l+m}{0.8}\PYG{p}{)}
\end{sphinxVerbatim}

\end{sphinxuseclass}\end{sphinxVerbatimInput}
\begin{sphinxVerbatimOutput}

\begin{sphinxuseclass}{cell_output}\begin{equation*}
\begin{split}0.788144601416603\end{split}
\end{equation*}
\end{sphinxuseclass}\end{sphinxVerbatimOutput}

\end{sphinxuseclass}
\begin{sphinxuseclass}{cell}\begin{sphinxVerbatimInput}

\begin{sphinxuseclass}{cell_input}
\begin{sphinxVerbatim}[commandchars=\\\{\}]
\PYG{n+nf}{pnorm}\PYG{p}{(}\PYG{l+m}{0.9}\PYG{p}{)}\PYG{+w}{ }\PYG{o}{\PYGZhy{}}\PYG{+w}{ }\PYG{n+nf}{pnorm}\PYG{p}{(}\PYG{l+m}{\PYGZhy{}0.3}\PYG{p}{)}
\end{sphinxVerbatim}

\end{sphinxuseclass}\end{sphinxVerbatimInput}
\begin{sphinxVerbatimOutput}

\begin{sphinxuseclass}{cell_output}\begin{equation*}
\begin{split}0.433851296842193\end{split}
\end{equation*}
\end{sphinxuseclass}\end{sphinxVerbatimOutput}

\end{sphinxuseclass}
\begin{sphinxuseclass}{cell}\begin{sphinxVerbatimInput}

\begin{sphinxuseclass}{cell_input}
\begin{sphinxVerbatim}[commandchars=\\\{\}]
\PYG{n+nf}{pnorm}\PYG{p}{(}\PYG{l+m}{\PYGZhy{}0.4}\PYG{p}{)}
\end{sphinxVerbatim}

\end{sphinxuseclass}\end{sphinxVerbatimInput}
\begin{sphinxVerbatimOutput}

\begin{sphinxuseclass}{cell_output}\begin{equation*}
\begin{split}0.344578258389676\end{split}
\end{equation*}
\end{sphinxuseclass}\end{sphinxVerbatimOutput}

\end{sphinxuseclass}
\begin{sphinxuseclass}{cell}\begin{sphinxVerbatimInput}

\begin{sphinxuseclass}{cell_input}
\begin{sphinxVerbatim}[commandchars=\\\{\}]
\PYG{n+nf}{pnorm}\PYG{p}{(}\PYG{l+m}{1.3}\PYG{p}{)}\PYG{o}{\PYGZhy{}}\PYG{n+nf}{pnorm}\PYG{p}{(}\PYG{l+m}{0.4}\PYG{p}{)}
\end{sphinxVerbatim}

\end{sphinxuseclass}\end{sphinxVerbatimInput}
\begin{sphinxVerbatimOutput}

\begin{sphinxuseclass}{cell_output}\begin{equation*}
\begin{split}0.247777773804066\end{split}
\end{equation*}
\end{sphinxuseclass}\end{sphinxVerbatimOutput}

\end{sphinxuseclass}
\begin{sphinxuseclass}{cell}\begin{sphinxVerbatimInput}

\begin{sphinxuseclass}{cell_input}
\begin{sphinxVerbatim}[commandchars=\\\{\}]
\PYG{l+m}{1}\PYG{+w}{ }\PYG{o}{\PYGZhy{}}\PYG{+w}{ }\PYG{p}{(}\PYG{n+nf}{pnorm}\PYG{p}{(}\PYG{l+m}{1.5}\PYG{p}{)}\PYG{+w}{ }\PYG{o}{\PYGZhy{}}\PYG{+w}{ }\PYG{n+nf}{pnorm}\PYG{p}{(}\PYG{l+m}{\PYGZhy{}1.5}\PYG{p}{)}\PYG{p}{)}
\end{sphinxVerbatim}

\end{sphinxuseclass}\end{sphinxVerbatimInput}
\begin{sphinxVerbatimOutput}

\begin{sphinxuseclass}{cell_output}\begin{equation*}
\begin{split}0.133614402537716\end{split}
\end{equation*}
\end{sphinxuseclass}\end{sphinxVerbatimOutput}

\end{sphinxuseclass}
\sphinxAtStartPar
a) 1 stanndard deviation

\sphinxAtStartPar
b) 1.15


\subsection{Exercise 7: Percentiles, median, interquartile range}
\label{\detokenize{exercises_unit_3:exercise-7-percentiles-median-interquartile-range}}
\begin{sphinxuseclass}{cell}\begin{sphinxVerbatimInput}

\begin{sphinxuseclass}{cell_input}
\begin{sphinxVerbatim}[commandchars=\\\{\}]
\PYG{c+c1}{\PYGZsh{} height in cm}

\PYG{n+nf}{round}\PYG{p}{(}\PYG{n+nf}{quantile}\PYG{p}{(}\PYG{p}{(}\PYG{n}{data}\PYG{o}{\PYGZdl{}}\PYG{n}{Height}\PYG{o}{*}\PYG{l+m}{2.45}\PYG{p}{)}\PYG{p}{,}\PYG{+w}{ }\PYG{n+nf}{c}\PYG{p}{(}\PYG{l+m}{0.01}\PYG{p}{,}\PYG{+w}{ }\PYG{l+m}{0.1}\PYG{p}{,}\PYG{+w}{ }\PYG{l+m}{0.25}\PYG{p}{,}\PYG{+w}{ }\PYG{l+m}{0.5}\PYG{p}{,}\PYG{+w}{ }\PYG{l+m}{0.75}\PYG{p}{,}\PYG{+w}{ }\PYG{l+m}{0.9}\PYG{p}{,}\PYG{+w}{ }\PYG{l+m}{0.99}\PYG{p}{)}\PYG{p}{,}\PYG{+w}{ }\PYG{l+m}{2}\PYG{p}{)}\PYG{p}{)}
\end{sphinxVerbatim}

\end{sphinxuseclass}\end{sphinxVerbatimInput}
\begin{sphinxVerbatimOutput}

\begin{sphinxuseclass}{cell_output}\begin{equation*}
\begin{split}\begin{description*}
\item[1\textbackslash{}\%] 156
\item[10\textbackslash{}\%] 161
\item[25\textbackslash{}\%] 163
\item[50\textbackslash{}\%] 167
\item[75\textbackslash{}\%] 170
\item[90\textbackslash{}\%] 173
\item[99\textbackslash{}\%] 177
\end{description*}\end{split}
\end{equation*}
\end{sphinxuseclass}\end{sphinxVerbatimOutput}

\end{sphinxuseclass}
\begin{sphinxuseclass}{cell}\begin{sphinxVerbatimInput}

\begin{sphinxuseclass}{cell_input}
\begin{sphinxVerbatim}[commandchars=\\\{\}]
\PYG{n+nf}{mean}\PYG{p}{(}\PYG{n}{data}\PYG{o}{\PYGZdl{}}\PYG{n}{Height}\PYG{o}{*}\PYG{l+m}{2.45}\PYG{p}{)}
\end{sphinxVerbatim}

\end{sphinxuseclass}\end{sphinxVerbatimInput}
\begin{sphinxVerbatimOutput}

\begin{sphinxuseclass}{cell_output}\begin{equation*}
\begin{split}166.58312831216\end{split}
\end{equation*}
\end{sphinxuseclass}\end{sphinxVerbatimOutput}

\end{sphinxuseclass}
\begin{sphinxuseclass}{cell}\begin{sphinxVerbatimInput}

\begin{sphinxuseclass}{cell_input}
\begin{sphinxVerbatim}[commandchars=\\\{\}]
\PYG{n+nf}{sd}\PYG{p}{(}\PYG{n}{data}\PYG{o}{\PYGZdl{}}\PYG{n}{Height}\PYG{o}{*}\PYG{l+m}{2.45}\PYG{p}{)}
\end{sphinxVerbatim}

\end{sphinxuseclass}\end{sphinxVerbatimInput}
\begin{sphinxVerbatimOutput}

\begin{sphinxuseclass}{cell_output}\begin{equation*}
\begin{split}4.65911298945375\end{split}
\end{equation*}
\end{sphinxuseclass}\end{sphinxVerbatimOutput}

\end{sphinxuseclass}
\begin{sphinxuseclass}{cell}\begin{sphinxVerbatimInput}

\begin{sphinxuseclass}{cell_input}
\begin{sphinxVerbatim}[commandchars=\\\{\}]
\PYG{n+nf}{IQR}\PYG{p}{(}\PYG{n}{data}\PYG{o}{\PYGZdl{}}\PYG{n}{Height}\PYG{p}{)}
\end{sphinxVerbatim}

\end{sphinxuseclass}\end{sphinxVerbatimInput}
\begin{sphinxVerbatimOutput}

\begin{sphinxuseclass}{cell_output}\begin{equation*}
\begin{split}2.56856000000001\end{split}
\end{equation*}
\end{sphinxuseclass}\end{sphinxVerbatimOutput}

\end{sphinxuseclass}
\sphinxAtStartPar
Median and mean coincide because the distribution is symmetric. Since it is at the same time approximated by a normal curve very well, the sd is larger than the IQR because within 1 sd from the mean are 68 \% of all values and in the IQR are only 50\%, so it must be narrower.


\subsection{Exercise 8: Practcing the tapply function}
\label{\detokenize{exercises_unit_3:exercise-8-practcing-the-tapply-function}}
\begin{sphinxuseclass}{cell}\begin{sphinxVerbatimInput}

\begin{sphinxuseclass}{cell_input}
\begin{sphinxVerbatim}[commandchars=\\\{\}]
\PYG{n+nf}{library}\PYG{p}{(}\PYG{n}{JWL}\PYG{p}{)}
\PYG{n}{dhs\PYGZus{}data}\PYG{+w}{  }\PYG{o}{\PYGZlt{}\PYGZhy{}}\PYG{+w}{ }\PYG{n}{children\PYGZus{}nutrition\PYGZus{}data}

\PYG{n}{dhs\PYGZus{}data}\PYG{o}{\PYGZdl{}}\PYG{n}{weighted\PYGZus{}nt\PYGZus{}ch\PYGZus{}stunt}\PYG{+w}{  }\PYG{o}{\PYGZlt{}\PYGZhy{}}\PYG{+w}{ }\PYG{n}{dhs\PYGZus{}data}\PYG{o}{\PYGZdl{}}\PYG{n}{nt\PYGZus{}ch\PYGZus{}stunt}\PYG{o}{*}\PYG{n}{dhs\PYGZus{}data}\PYG{o}{\PYGZdl{}}\PYG{n}{wt}
\end{sphinxVerbatim}

\end{sphinxuseclass}\end{sphinxVerbatimInput}

\end{sphinxuseclass}
\begin{sphinxuseclass}{cell}\begin{sphinxVerbatimInput}

\begin{sphinxuseclass}{cell_input}
\begin{sphinxVerbatim}[commandchars=\\\{\}]
\PYG{n+nf}{tapply}\PYG{p}{(}\PYG{n}{dhs\PYGZus{}data}\PYG{o}{\PYGZdl{}}\PYG{n}{weighted\PYGZus{}nt\PYGZus{}ch\PYGZus{}stunt}\PYG{p}{,}\PYG{+w}{ }\PYG{n}{dhs\PYGZus{}data}\PYG{o}{\PYGZdl{}}\PYG{n}{region}\PYG{p}{,}\PYG{+w}{ }\PYG{n}{sum}\PYG{p}{)}
\end{sphinxVerbatim}

\end{sphinxuseclass}\end{sphinxVerbatimInput}
\begin{sphinxVerbatimOutput}

\begin{sphinxuseclass}{cell_output}\begin{equation*}
\begin{split}\begin{description*}
\item[region 1] 360.464472
\item[region 2] 221.903655
\item[region 3] 105.694042
\item[region 4] 207.273796
\end{description*}\end{split}
\end{equation*}
\end{sphinxuseclass}\end{sphinxVerbatimOutput}

\end{sphinxuseclass}
\begin{sphinxuseclass}{cell}\begin{sphinxVerbatimInput}

\begin{sphinxuseclass}{cell_input}
\begin{sphinxVerbatim}[commandchars=\\\{\}]
\PYG{n+nf}{tapply}\PYG{p}{(}\PYG{n}{dhs\PYGZus{}data}\PYG{o}{\PYGZdl{}}\PYG{n}{wt}\PYG{p}{,}\PYG{+w}{ }\PYG{n}{dhs\PYGZus{}data}\PYG{o}{\PYGZdl{}}\PYG{n}{region}\PYG{p}{,}\PYG{+w}{ }\PYG{n}{sum}\PYG{p}{)}
\end{sphinxVerbatim}

\end{sphinxuseclass}\end{sphinxVerbatimInput}
\begin{sphinxVerbatimOutput}

\begin{sphinxuseclass}{cell_output}\begin{equation*}
\begin{split}\begin{description*}
\item[region 1] 1070.344578
\item[region 2] 544.079987
\item[region 3] 352.525147
\item[region 4] 527.152629
\end{description*}\end{split}
\end{equation*}
\end{sphinxuseclass}\end{sphinxVerbatimOutput}

\end{sphinxuseclass}
\begin{sphinxuseclass}{cell}\begin{sphinxVerbatimInput}

\begin{sphinxuseclass}{cell_input}
\begin{sphinxVerbatim}[commandchars=\\\{\}]
\PYG{n+nf}{tapply}\PYG{p}{(}\PYG{n}{dhs\PYGZus{}data}\PYG{o}{\PYGZdl{}}\PYG{n}{weighted\PYGZus{}nt\PYGZus{}ch\PYGZus{}stunt}\PYG{p}{,}\PYG{+w}{ }\PYG{n}{dhs\PYGZus{}data}\PYG{o}{\PYGZdl{}}\PYG{n}{region}\PYG{p}{,}\PYG{+w}{ }\PYG{n}{sum}\PYG{p}{)}\PYG{o}{/}\PYG{n+nf}{tapply}\PYG{p}{(}\PYG{n}{dhs\PYGZus{}data}\PYG{o}{\PYGZdl{}}\PYG{n}{wt}\PYG{p}{,}\PYG{+w}{ }\PYG{n}{dhs\PYGZus{}data}\PYG{o}{\PYGZdl{}}\PYG{n}{region}\PYG{p}{,}\PYG{+w}{ }\PYG{n}{sum}\PYG{p}{)}
\end{sphinxVerbatim}

\end{sphinxuseclass}\end{sphinxVerbatimInput}
\begin{sphinxVerbatimOutput}

\begin{sphinxuseclass}{cell_output}\begin{equation*}
\begin{split}\begin{description*}
\item[region 1] 0.33677423084961
\item[region 2] 0.407851162149068
\item[region 3] 0.29981986504923
\item[region 4] 0.393195034222242
\end{description*}\end{split}
\end{equation*}
\end{sphinxuseclass}\end{sphinxVerbatimOutput}

\end{sphinxuseclass}
\sphinxAtStartPar
Note that the complication comes from the weighting variable wt. We first created a new variable which weights
the stunted cases with wt. The new variable is called here \sphinxcode{\sphinxupquote{weighte\_nt\_ch\_stunt}} and is appended to the dataframe by \sphinxcode{\sphinxupquote{dhs\_data\$weighted\_nt\_ch\_stunt  <\sphinxhyphen{} dhs\_data\$nt\_ch\_stunt*dhs\_data\$wt}} and then use \sphinxcode{\sphinxupquote{tapply()}} twice once for the numerator and then for the denominator and do the necessary division to get the proportion at the end. If you compare we have saved a lot of typing compared with our original solution.


\section{Project: The present and future of humanity in pictures and numbers}
\label{\detokenize{exercises_unit_3:project-the-present-and-future-of-humanity-in-pictures-and-numbers}}

\subsection{Age percentiles in Kenya, the United States and Japan}
\label{\detokenize{exercises_unit_3:age-percentiles-in-kenya-the-united-states-and-japan}}
\sphinxAtStartPar
In the last step of our project, people count, we were trying to
compute the mean age for Kenya, the US and Japan. But if we look
at the distribution of the number of people across age groups, we
see not so much symmetric but rather asymmetric distributions. So let’s
continue our analysis here by considering percentiles and
look at the median as one percentile, which can be alternatively
used to characterize the center of a distribution. Let us
also study spreads using the interquartile range.

\sphinxAtStartPar
In order to do so, we can again load the data and make country subsets,
as we did last time. Those of you who successfully completed this task
last time, can now just recycle the code from the last unit. The others
can give it a second trial here. Note the advantage of the notebook. You can now reuse your old code.

\sphinxAtStartPar
So let’s recycle

\begin{sphinxuseclass}{cell}\begin{sphinxVerbatimInput}

\begin{sphinxuseclass}{cell_input}
\begin{sphinxVerbatim}[commandchars=\\\{\}]
\PYG{c+c1}{\PYGZsh{} Load data and give them a simpler name}

\PYG{n+nf}{library}\PYG{p}{(}\PYG{n}{JWL}\PYG{p}{)}
\PYG{n}{dat}\PYG{+w}{  }\PYG{o}{\PYGZlt{}\PYGZhy{}}\PYG{+w}{ }\PYG{n}{population\PYGZus{}statistics\PYGZus{}by\PYGZus{}age\PYGZus{}and\PYGZus{}sex}

\PYG{c+c1}{\PYGZsh{} Form country sets}

\PYG{n}{ke\PYGZus{}2022}\PYG{+w}{ }\PYG{o}{\PYGZlt{}\PYGZhy{}}\PYG{+w}{ }\PYG{n}{dat}\PYG{p}{[}\PYG{n}{dat}\PYG{o}{\PYGZdl{}}\PYG{n}{Year}\PYG{+w}{ }\PYG{o}{==}\PYG{+w}{ }\PYG{l+m}{2022}\PYG{+w}{ }\PYG{o}{\PYGZam{}}\PYG{+w}{ }\PYG{n}{dat}\PYG{o}{\PYGZdl{}}\PYG{n}{ISO2}\PYG{+w}{ }\PYG{o}{==}\PYG{+w}{ }\PYG{l+s}{\PYGZdq{}}\PYG{l+s}{KE\PYGZdq{}}\PYG{p}{,}\PYG{+w}{ }\PYG{p}{]}
\PYG{n}{us\PYGZus{}2022}\PYG{+w}{ }\PYG{o}{\PYGZlt{}\PYGZhy{}}\PYG{+w}{ }\PYG{n}{dat}\PYG{p}{[}\PYG{n}{dat}\PYG{o}{\PYGZdl{}}\PYG{n}{Year}\PYG{+w}{ }\PYG{o}{==}\PYG{+w}{ }\PYG{l+m}{2022}\PYG{+w}{ }\PYG{o}{\PYGZam{}}\PYG{+w}{ }\PYG{n}{dat}\PYG{o}{\PYGZdl{}}\PYG{n}{ISO2}\PYG{+w}{ }\PYG{o}{==}\PYG{+w}{ }\PYG{l+s}{\PYGZdq{}}\PYG{l+s}{US\PYGZdq{}}\PYG{p}{,}\PYG{+w}{ }\PYG{p}{]}
\PYG{n}{jp\PYGZus{}2022}\PYG{+w}{ }\PYG{o}{\PYGZlt{}\PYGZhy{}}\PYG{+w}{ }\PYG{n}{dat}\PYG{p}{[}\PYG{n}{dat}\PYG{o}{\PYGZdl{}}\PYG{n}{Year}\PYG{+w}{ }\PYG{o}{==}\PYG{+w}{ }\PYG{l+m}{2022}\PYG{+w}{ }\PYG{o}{\PYGZam{}}\PYG{+w}{ }\PYG{n}{dat}\PYG{o}{\PYGZdl{}}\PYG{n}{ISO2}\PYG{+w}{ }\PYG{o}{==}\PYG{+w}{ }\PYG{l+s}{\PYGZdq{}}\PYG{l+s}{JP\PYGZdq{}}\PYG{p}{,}\PYG{+w}{ }\PYG{p}{]}
\PYG{+w}{               }
\PYG{c+c1}{\PYGZsh{} And your own country}
\PYG{+w}{               }
\PYG{n}{at\PYGZus{}2022}\PYG{+w}{  }\PYG{o}{\PYGZlt{}\PYGZhy{}}\PYG{+w}{  }\PYG{n}{dat}\PYG{p}{[}\PYG{n}{dat}\PYG{o}{\PYGZdl{}}\PYG{n}{Year}\PYG{+w}{ }\PYG{o}{==}\PYG{+w}{ }\PYG{l+m}{2022}\PYG{+w}{ }\PYG{o}{\PYGZam{}}\PYG{+w}{ }\PYG{n}{dat}\PYG{o}{\PYGZdl{}}\PYG{n}{ISO2}\PYG{+w}{ }\PYG{o}{==}\PYG{+w}{ }\PYG{l+s}{\PYGZdq{}}\PYG{l+s}{AT\PYGZdq{}}\PYG{p}{,}\PYG{+w}{ }\PYG{p}{]}\PYG{+w}{              }
\end{sphinxVerbatim}

\end{sphinxuseclass}\end{sphinxVerbatimInput}

\end{sphinxuseclass}\begin{enumerate}
\sphinxsetlistlabels{\arabic}{enumi}{enumii}{}{.}%
\item {} 
\sphinxAtStartPar
Find the median age group for Kenya, the United States,
Japan and for your
own country, by constructing a dataframe extended by the cumulative count of
people across age groups and the proportion of the cumulative count
in the total population. Example: If the age group 0\sphinxhyphen{}4 has 100,
5\sphinxhyphen{}9 120 and 10\sphinxhyphen{}14 200, the cumulative count is 100, 220, 420. R has a function doing
cumulative counts. It is called \sphinxcode{\sphinxupquote{cumsum()}}. You might use this function.
Where are the 25 and the 75 percentile?

\end{enumerate}

\begin{sphinxuseclass}{cell}\begin{sphinxVerbatimInput}

\begin{sphinxuseclass}{cell_input}
\begin{sphinxVerbatim}[commandchars=\\\{\}]
\PYG{n}{table\PYGZus{}ke}\PYG{+w}{  }\PYG{o}{\PYGZlt{}\PYGZhy{}}\PYG{+w}{ }\PYG{n+nf}{data.frame}\PYG{p}{(}\PYG{n}{Country}\PYG{+w}{ }\PYG{o}{=}\PYG{+w}{ }\PYG{n}{ke\PYGZus{}2022}\PYG{o}{\PYGZdl{}}\PYG{n}{Country}\PYG{p}{[}\PYG{l+m}{1}\PYG{o}{:}\PYG{p}{(}\PYG{n+nf}{nrow}\PYG{p}{(}\PYG{n}{ke\PYGZus{}2022}\PYG{p}{)}\PYG{o}{/}\PYG{l+m}{2}\PYG{p}{)}\PYG{p}{]}\PYG{p}{,}
\PYG{+w}{                        }\PYG{n}{Age}\PYG{+w}{ }\PYG{o}{=}\PYG{+w}{ }\PYG{n}{ke\PYGZus{}2022}\PYG{o}{\PYGZdl{}}\PYG{n}{Age}\PYG{p}{[}\PYG{l+m}{1}\PYG{o}{:}\PYG{p}{(}\PYG{n+nf}{nrow}\PYG{p}{(}\PYG{n}{ke\PYGZus{}2022}\PYG{p}{)}\PYG{o}{/}\PYG{l+m}{2}\PYG{p}{)}\PYG{p}{]}\PYG{p}{,}
\PYG{+w}{                        }\PYG{n}{POP}\PYG{+w}{ }\PYG{o}{=}\PYG{+w}{ }\PYG{p}{(}\PYG{n}{ke\PYGZus{}2022}\PYG{o}{\PYGZdl{}}\PYG{n}{POP}\PYG{p}{[}\PYG{n}{ke\PYGZus{}2022}\PYG{o}{\PYGZdl{}}\PYG{n}{Sex}\PYG{+w}{ }\PYG{o}{==}\PYG{+w}{ }\PYG{l+s}{\PYGZdq{}}\PYG{l+s}{F\PYGZdq{}}\PYG{p}{]}\PYG{+w}{ }\PYG{o}{+}\PYG{+w}{ }\PYG{n}{ke\PYGZus{}2022}\PYG{o}{\PYGZdl{}}\PYG{n}{POP}\PYG{p}{[}\PYG{n}{ke\PYGZus{}2022}\PYG{o}{\PYGZdl{}}\PYG{n}{Sex}\PYG{+w}{ }\PYG{o}{==}\PYG{+w}{ }\PYG{l+s}{\PYGZdq{}}\PYG{l+s}{M\PYGZdq{}}\PYG{p}{]}\PYG{p}{)}\PYG{p}{)}

\PYG{n}{table\PYGZus{}ke}\PYG{o}{\PYGZdl{}}\PYG{n}{CumProp}\PYG{+w}{  }\PYG{o}{\PYGZlt{}\PYGZhy{}}\PYG{+w}{ }\PYG{n+nf}{cumsum}\PYG{p}{(}\PYG{n}{table\PYGZus{}ke}\PYG{o}{\PYGZdl{}}\PYG{n}{POP}\PYG{p}{)}\PYG{o}{/}\PYG{n+nf}{sum}\PYG{p}{(}\PYG{n}{table\PYGZus{}ke}\PYG{o}{\PYGZdl{}}\PYG{n}{POP}\PYG{p}{)}
\PYG{+w}{                            }

\PYG{n}{table\PYGZus{}ke}
\end{sphinxVerbatim}

\end{sphinxuseclass}\end{sphinxVerbatimInput}
\begin{sphinxVerbatimOutput}

\begin{sphinxuseclass}{cell_output}\begin{equation*}
\begin{split}A data.frame: 20 × 4
\begin{tabular}{llll}
 Country & Age & POP & CumProp\\
 <fct> & <fct> & <dbl> & <dbl>\\
\hline
	 Kenya & 0-4   & 7018768 & 0.1256392\\
	 Kenya & 5-9   & 6835587 & 0.2479993\\
	 Kenya & 10-14 & 6917709 & 0.3718295\\
	 Kenya & 15-19 & 6475709 & 0.4877476\\
	 Kenya & 20-24 & 5321696 & 0.5830084\\
	 Kenya & 25-29 & 4259157 & 0.6592493\\
	 Kenya & 30-34 & 3873213 & 0.7285816\\
	 Kenya & 35-39 & 3680058 & 0.7944563\\
	 Kenya & 40-44 & 3217009 & 0.8520423\\
	 Kenya & 45-49 & 2367388 & 0.8944196\\
	 Kenya & 50-54 & 1749660 & 0.9257393\\
	 Kenya & 55-59 & 1351968 & 0.9499402\\
	 Kenya & 60-64 & 1004843 & 0.9679273\\
	 Kenya & 65-69 &  728319 & 0.9809646\\
	 Kenya & 70-74 &  502232 & 0.9899547\\
	 Kenya & 75-79 &  306856 & 0.9954476\\
	 Kenya & 80-84 &  166817 & 0.9984337\\
	 Kenya & 85-89 &   67699 & 0.9996456\\
	 Kenya & 90-94 &   17379 & 0.9999566\\
	 Kenya & 95-99 &    2422 & 1.0000000\\
\end{tabular}\end{split}
\end{equation*}
\end{sphinxuseclass}\end{sphinxVerbatimOutput}

\end{sphinxuseclass}
\sphinxAtStartPar
The median age is within the age group 15\sphinxhyphen{}19, because in this age group there are approximately 50\% of the population in younger age groups and 50\% in higher age groups. This is lower than the mean age. In the data the mean is pulled up by the older age groups. The difference between median and mean is quite large for Kenya. The 25 \% percentile is at the age group 5\sphinxhyphen{}9, the 75 \% percentile is at around the age group 30\sphinxhyphen{}34.

\begin{sphinxuseclass}{cell}\begin{sphinxVerbatimInput}

\begin{sphinxuseclass}{cell_input}
\begin{sphinxVerbatim}[commandchars=\\\{\}]
\PYG{n}{table\PYGZus{}us}\PYG{+w}{  }\PYG{o}{\PYGZlt{}\PYGZhy{}}\PYG{+w}{ }\PYG{n+nf}{data.frame}\PYG{p}{(}\PYG{n}{Country}\PYG{+w}{ }\PYG{o}{=}\PYG{+w}{ }\PYG{n}{us\PYGZus{}2022}\PYG{o}{\PYGZdl{}}\PYG{n}{Country}\PYG{p}{[}\PYG{l+m}{1}\PYG{o}{:}\PYG{p}{(}\PYG{n+nf}{nrow}\PYG{p}{(}\PYG{n}{us\PYGZus{}2022}\PYG{p}{)}\PYG{o}{/}\PYG{l+m}{2}\PYG{p}{)}\PYG{p}{]}\PYG{p}{,}
\PYG{+w}{                        }\PYG{n}{Age}\PYG{+w}{ }\PYG{o}{=}\PYG{+w}{ }\PYG{n}{us\PYGZus{}2022}\PYG{o}{\PYGZdl{}}\PYG{n}{Age}\PYG{p}{[}\PYG{l+m}{1}\PYG{o}{:}\PYG{p}{(}\PYG{n+nf}{nrow}\PYG{p}{(}\PYG{n}{ke\PYGZus{}2022}\PYG{p}{)}\PYG{o}{/}\PYG{l+m}{2}\PYG{p}{)}\PYG{p}{]}\PYG{p}{,}
\PYG{+w}{                        }\PYG{n}{POP}\PYG{+w}{ }\PYG{o}{=}\PYG{+w}{ }\PYG{p}{(}\PYG{n}{us\PYGZus{}2022}\PYG{o}{\PYGZdl{}}\PYG{n}{POP}\PYG{p}{[}\PYG{n}{us\PYGZus{}2022}\PYG{o}{\PYGZdl{}}\PYG{n}{Sex}\PYG{+w}{ }\PYG{o}{==}\PYG{+w}{ }\PYG{l+s}{\PYGZdq{}}\PYG{l+s}{F\PYGZdq{}}\PYG{p}{]}\PYG{+w}{ }\PYG{o}{+}\PYG{+w}{ }\PYG{n}{us\PYGZus{}2022}\PYG{o}{\PYGZdl{}}\PYG{n}{POP}\PYG{p}{[}\PYG{n}{us\PYGZus{}2022}\PYG{o}{\PYGZdl{}}\PYG{n}{Sex}\PYG{+w}{ }\PYG{o}{==}\PYG{+w}{ }\PYG{l+s}{\PYGZdq{}}\PYG{l+s}{M\PYGZdq{}}\PYG{p}{]}\PYG{p}{)}\PYG{p}{)}

\PYG{n}{table\PYGZus{}us}\PYG{o}{\PYGZdl{}}\PYG{n}{CumProp}\PYG{+w}{  }\PYG{o}{\PYGZlt{}\PYGZhy{}}\PYG{+w}{ }\PYG{n+nf}{cumsum}\PYG{p}{(}\PYG{n}{table\PYGZus{}us}\PYG{o}{\PYGZdl{}}\PYG{n}{POP}\PYG{p}{)}\PYG{o}{/}\PYG{n+nf}{sum}\PYG{p}{(}\PYG{n}{table\PYGZus{}us}\PYG{o}{\PYGZdl{}}\PYG{n}{POP}\PYG{p}{)}
\PYG{+w}{                            }

\PYG{n}{table\PYGZus{}us}
\end{sphinxVerbatim}

\end{sphinxuseclass}\end{sphinxVerbatimInput}
\begin{sphinxVerbatimOutput}

\begin{sphinxuseclass}{cell_output}\begin{equation*}
\begin{split}A data.frame: 20 × 4
\begin{tabular}{llll}
 Country & Age & POP & CumProp\\
 <fct> & <fct> & <dbl> & <dbl>\\
\hline
	 United States of America & 0-4   & 18538353 & 0.05563755\\
	 United States of America & 5-9   & 20009195 & 0.11568942\\
	 United States of America & 10-14 & 20889839 & 0.17838428\\
	 United States of America & 15-19 & 21635792 & 0.24331791\\
	 United States of America & 20-24 & 22705779 & 0.31146280\\
	 United States of America & 25-29 & 22193164 & 0.37806922\\
	 United States of America & 30-34 & 23308136 & 0.44802191\\
	 United States of America & 35-39 & 22267949 & 0.51485277\\
	 United States of America & 40-44 & 21427416 & 0.57916102\\
	 United States of America & 45-49 & 19624098 & 0.63805713\\
	 United States of America & 50-54 & 20807547 & 0.70050501\\
	 United States of America & 55-59 & 20967014 & 0.76343150\\
	 United States of America & 60-64 & 21118423 & 0.82681239\\
	 United States of America & 65-69 & 18631422 & 0.88272926\\
	 United States of America & 70-74 & 15157017 & 0.92821870\\
	 United States of America & 75-79 & 10861000 & 0.96081488\\
	 United States of America & 80-84 &  6659545 & 0.98080160\\
	 United States of America & 85-89 &  3906331 & 0.99252533\\
	 United States of America & 90-94 &  1908941 & 0.99825447\\
	 United States of America & 95-99 &   581608 & 1.00000000\\
\end{tabular}\end{split}
\end{equation*}
\end{sphinxuseclass}\end{sphinxVerbatimOutput}

\end{sphinxuseclass}
\sphinxAtStartPar
The median age group in the US is at the age group 35 to 39 pretty much at the mean age. This is because the distribution of age groups in the US is less skewed since the US is a middle layered country. The 25 \% percentile is at the age group 15\sphinxhyphen{}19, the 75 \% percentile is at the age group 55\sphinxhyphen{}59.

\begin{sphinxuseclass}{cell}\begin{sphinxVerbatimInput}

\begin{sphinxuseclass}{cell_input}
\begin{sphinxVerbatim}[commandchars=\\\{\}]
\PYG{n}{table\PYGZus{}jp}\PYG{+w}{  }\PYG{o}{\PYGZlt{}\PYGZhy{}}\PYG{+w}{ }\PYG{n+nf}{data.frame}\PYG{p}{(}\PYG{n}{Country}\PYG{+w}{ }\PYG{o}{=}\PYG{+w}{ }\PYG{n}{jp\PYGZus{}2022}\PYG{o}{\PYGZdl{}}\PYG{n}{Country}\PYG{p}{[}\PYG{l+m}{1}\PYG{o}{:}\PYG{p}{(}\PYG{n+nf}{nrow}\PYG{p}{(}\PYG{n}{jp\PYGZus{}2022}\PYG{p}{)}\PYG{o}{/}\PYG{l+m}{2}\PYG{p}{)}\PYG{p}{]}\PYG{p}{,}
\PYG{+w}{                        }\PYG{n}{Age}\PYG{+w}{ }\PYG{o}{=}\PYG{+w}{ }\PYG{n}{jp\PYGZus{}2022}\PYG{o}{\PYGZdl{}}\PYG{n}{Age}\PYG{p}{[}\PYG{l+m}{1}\PYG{o}{:}\PYG{p}{(}\PYG{n+nf}{nrow}\PYG{p}{(}\PYG{n}{jp\PYGZus{}2022}\PYG{p}{)}\PYG{o}{/}\PYG{l+m}{2}\PYG{p}{)}\PYG{p}{]}\PYG{p}{,}
\PYG{+w}{                        }\PYG{n}{POP}\PYG{+w}{ }\PYG{o}{=}\PYG{+w}{ }\PYG{p}{(}\PYG{n}{jp\PYGZus{}2022}\PYG{o}{\PYGZdl{}}\PYG{n}{POP}\PYG{p}{[}\PYG{n}{jp\PYGZus{}2022}\PYG{o}{\PYGZdl{}}\PYG{n}{Sex}\PYG{+w}{ }\PYG{o}{==}\PYG{+w}{ }\PYG{l+s}{\PYGZdq{}}\PYG{l+s}{F\PYGZdq{}}\PYG{p}{]}\PYG{+w}{ }\PYG{o}{+}\PYG{+w}{ }\PYG{n}{jp\PYGZus{}2022}\PYG{o}{\PYGZdl{}}\PYG{n}{POP}\PYG{p}{[}\PYG{n}{jp\PYGZus{}2022}\PYG{o}{\PYGZdl{}}\PYG{n}{Sex}\PYG{+w}{ }\PYG{o}{==}\PYG{+w}{ }\PYG{l+s}{\PYGZdq{}}\PYG{l+s}{M\PYGZdq{}}\PYG{p}{]}\PYG{p}{)}\PYG{p}{)}

\PYG{n}{table\PYGZus{}jp}\PYG{o}{\PYGZdl{}}\PYG{n}{CumProp}\PYG{+w}{  }\PYG{o}{\PYGZlt{}\PYGZhy{}}\PYG{+w}{ }\PYG{n+nf}{cumsum}\PYG{p}{(}\PYG{n}{table\PYGZus{}jp}\PYG{o}{\PYGZdl{}}\PYG{n}{POP}\PYG{p}{)}\PYG{o}{/}\PYG{n+nf}{sum}\PYG{p}{(}\PYG{n}{table\PYGZus{}jp}\PYG{o}{\PYGZdl{}}\PYG{n}{POP}\PYG{p}{)}
\PYG{+w}{                            }

\PYG{n}{table\PYGZus{}jp}
\end{sphinxVerbatim}

\end{sphinxuseclass}\end{sphinxVerbatimInput}
\begin{sphinxVerbatimOutput}

\begin{sphinxuseclass}{cell_output}\begin{equation*}
\begin{split}A data.frame: 20 × 4
\begin{tabular}{llll}
 Country & Age & POP & CumProp\\
 <fct> & <fct> & <dbl> & <dbl>\\
\hline
	 Japan & 0-4   & 4556555 & 0.03671436\\
	 Japan & 5-9   & 5278748 & 0.07924778\\
	 Japan & 10-14 & 5632437 & 0.12463104\\
	 Japan & 15-19 & 5625932 & 0.16996189\\
	 Japan & 20-24 & 5959064 & 0.21797695\\
	 Japan & 25-29 & 6167973 & 0.26767528\\
	 Japan & 30-34 & 6348230 & 0.31882603\\
	 Japan & 35-39 & 7275888 & 0.37745137\\
	 Japan & 40-44 & 7911658 & 0.44119942\\
	 Japan & 45-49 & 9180119 & 0.51516807\\
	 Japan & 50-54 & 9069125 & 0.58824238\\
	 Japan & 55-59 & 7817264 & 0.65122985\\
	 Japan & 60-64 & 7347215 & 0.71042991\\
	 Japan & 65-69 & 7575423 & 0.77146874\\
	 Japan & 70-74 & 9360496 & 0.84689078\\
	 Japan & 75-79 & 6971647 & 0.90306470\\
	 Japan & 80-84 & 5652823 & 0.94861222\\
	 Japan & 85-89 & 3841456 & 0.97956469\\
	 Japan & 90-94 & 1930225 & 0.99511744\\
	 Japan & 95-99 &  605966 & 1.00000000\\
\end{tabular}\end{split}
\end{equation*}
\end{sphinxuseclass}\end{sphinxVerbatimOutput}

\end{sphinxuseclass}
\sphinxAtStartPar
Here the median is at the ager group 45\sphinxhyphen{}49 also more congruent with the mean. The 25 \% percentile is at 25 \sphinxhyphen{} 29 and the 75 \% percentile is at 60\sphinxhyphen{}64.

\begin{sphinxuseclass}{cell}\begin{sphinxVerbatimInput}

\begin{sphinxuseclass}{cell_input}
\begin{sphinxVerbatim}[commandchars=\\\{\}]
\PYG{n}{table\PYGZus{}at}\PYG{+w}{  }\PYG{o}{\PYGZlt{}\PYGZhy{}}\PYG{+w}{ }\PYG{n+nf}{data.frame}\PYG{p}{(}\PYG{n}{Country}\PYG{+w}{ }\PYG{o}{=}\PYG{+w}{ }\PYG{n}{at\PYGZus{}2022}\PYG{o}{\PYGZdl{}}\PYG{n}{Country}\PYG{p}{[}\PYG{l+m}{1}\PYG{o}{:}\PYG{p}{(}\PYG{n+nf}{nrow}\PYG{p}{(}\PYG{n}{at\PYGZus{}2022}\PYG{p}{)}\PYG{o}{/}\PYG{l+m}{2}\PYG{p}{)}\PYG{p}{]}\PYG{p}{,}
\PYG{+w}{                        }\PYG{n}{Age}\PYG{+w}{ }\PYG{o}{=}\PYG{+w}{ }\PYG{n}{at\PYGZus{}2022}\PYG{o}{\PYGZdl{}}\PYG{n}{Age}\PYG{p}{[}\PYG{l+m}{1}\PYG{o}{:}\PYG{p}{(}\PYG{n+nf}{nrow}\PYG{p}{(}\PYG{n}{at\PYGZus{}2022}\PYG{p}{)}\PYG{o}{/}\PYG{l+m}{2}\PYG{p}{)}\PYG{p}{]}\PYG{p}{,}
\PYG{+w}{                        }\PYG{n}{POP}\PYG{+w}{ }\PYG{o}{=}\PYG{+w}{ }\PYG{p}{(}\PYG{n}{at\PYGZus{}2022}\PYG{o}{\PYGZdl{}}\PYG{n}{POP}\PYG{p}{[}\PYG{n}{at\PYGZus{}2022}\PYG{o}{\PYGZdl{}}\PYG{n}{Sex}\PYG{+w}{ }\PYG{o}{==}\PYG{+w}{ }\PYG{l+s}{\PYGZdq{}}\PYG{l+s}{F\PYGZdq{}}\PYG{p}{]}\PYG{+w}{ }\PYG{o}{+}\PYG{+w}{ }\PYG{n}{at\PYGZus{}2022}\PYG{o}{\PYGZdl{}}\PYG{n}{POP}\PYG{p}{[}\PYG{n}{at\PYGZus{}2022}\PYG{o}{\PYGZdl{}}\PYG{n}{Sex}\PYG{+w}{ }\PYG{o}{==}\PYG{+w}{ }\PYG{l+s}{\PYGZdq{}}\PYG{l+s}{M\PYGZdq{}}\PYG{p}{]}\PYG{p}{)}\PYG{p}{)}

\PYG{n}{table\PYGZus{}at}\PYG{o}{\PYGZdl{}}\PYG{n}{CumProp}\PYG{+w}{  }\PYG{o}{\PYGZlt{}\PYGZhy{}}\PYG{+w}{ }\PYG{n+nf}{cumsum}\PYG{p}{(}\PYG{n}{table\PYGZus{}at}\PYG{o}{\PYGZdl{}}\PYG{n}{POP}\PYG{p}{)}\PYG{o}{/}\PYG{n+nf}{sum}\PYG{p}{(}\PYG{n}{table\PYGZus{}at}\PYG{o}{\PYGZdl{}}\PYG{n}{POP}\PYG{p}{)}
\PYG{+w}{                            }

\PYG{n}{table\PYGZus{}at}
\end{sphinxVerbatim}

\end{sphinxuseclass}\end{sphinxVerbatimInput}
\begin{sphinxVerbatimOutput}

\begin{sphinxuseclass}{cell_output}\begin{equation*}
\begin{split}A data.frame: 20 × 4
\begin{tabular}{llll}
 Country & Age & POP & CumProp\\
 <fct> & <fct> & <dbl> & <dbl>\\
\hline
	 Austria & 0-4   & 423299 & 0.04752226\\
	 Austria & 5-9   & 419075 & 0.09457030\\
	 Austria & 10-14 & 410364 & 0.14064039\\
	 Austria & 15-19 & 424481 & 0.18829535\\
	 Austria & 20-24 & 466160 & 0.24062946\\
	 Austria & 25-29 & 566333 & 0.30420963\\
	 Austria & 30-34 & 601929 & 0.37178604\\
	 Austria & 35-39 & 597758 & 0.43889418\\
	 Austria & 40-44 & 581444 & 0.50417081\\
	 Austria & 45-49 & 577204 & 0.56897143\\
	 Austria & 50-54 & 687682 & 0.64617502\\
	 Austria & 55-59 & 712628 & 0.72617920\\
	 Austria & 60-64 & 621333 & 0.79593403\\
	 Austria & 65-69 & 493024 & 0.85128406\\
	 Austria & 70-74 & 428190 & 0.89935541\\
	 Austria & 75-79 & 341471 & 0.93769113\\
	 Austria & 80-84 & 308711 & 0.97234901\\
	 Austria & 85-89 & 142899 & 0.98839177\\
	 Austria & 90-94 &  75808 & 0.99690246\\
	 Austria & 95-99 &  27591 & 1.00000000\\
\end{tabular}\end{split}
\end{equation*}
\end{sphinxuseclass}\end{sphinxVerbatimOutput}

\end{sphinxuseclass}
\sphinxAtStartPar
Here the median age is in the group 40\sphinxhyphen{}44 quite congruent with the mean. The 25 \% percentile is at 20\sphinxhyphen{}24, the 75 \% percentile is at around age group 55 \sphinxhyphen{} 59
\begin{enumerate}
\sphinxsetlistlabels{\arabic}{enumi}{enumii}{}{.}%
\setcounter{enumi}{1}
\item {} 
\sphinxAtStartPar
Can you write a function, which takes as argument the entire dataframe, country and year and automatically returns the approximate median age\sphinxhyphen{}group?

\end{enumerate}

\begin{sphinxuseclass}{cell}\begin{sphinxVerbatimInput}

\begin{sphinxuseclass}{cell_input}
\begin{sphinxVerbatim}[commandchars=\\\{\}]
\PYG{n}{median\PYGZus{}age\PYGZus{}group}\PYG{+w}{  }\PYG{o}{\PYGZlt{}\PYGZhy{}}\PYG{+w}{ }\PYG{n+nf}{function}\PYG{p}{(}\PYG{n}{country}\PYG{p}{,}\PYG{+w}{ }\PYG{n}{year}\PYG{p}{,}\PYG{+w}{ }\PYG{n}{data}\PYG{p}{)}\PYG{p}{\PYGZob{}}
\PYG{+w}{    }
\PYG{+w}{    }\PYG{c+c1}{\PYGZsh{} form the subset of data for the year and the coutry selected in the function argument}
\PYG{+w}{    }
\PYG{+w}{    }\PYG{n}{dat}\PYG{+w}{  }\PYG{o}{\PYGZlt{}\PYGZhy{}}\PYG{+w}{ }\PYG{n}{data}\PYG{p}{[}\PYG{n}{data}\PYG{o}{\PYGZdl{}}\PYG{n}{Year}\PYG{+w}{ }\PYG{o}{==}\PYG{+w}{ }\PYG{n}{year}\PYG{+w}{ }\PYG{o}{\PYGZam{}}\PYG{+w}{ }\PYG{n}{dat}\PYG{o}{\PYGZdl{}}\PYG{n}{Country}\PYG{+w}{ }\PYG{o}{==}\PYG{+w}{ }\PYG{n}{country}\PYG{p}{,}\PYG{+w}{ }\PYG{p}{]}
\PYG{+w}{    }
\PYG{+w}{    }\PYG{c+c1}{\PYGZsh{} form the aggregate populatio count of women and men for all age groups}
\PYG{+w}{    }
\PYG{+w}{    }\PYG{n}{pop}\PYG{+w}{  }\PYG{o}{\PYGZlt{}\PYGZhy{}}\PYG{+w}{ }\PYG{n}{dat}\PYG{o}{\PYGZdl{}}\PYG{n}{POP}\PYG{p}{[}\PYG{n}{dat}\PYG{o}{\PYGZdl{}}\PYG{n}{Sex}\PYG{+w}{ }\PYG{o}{==}\PYG{+w}{ }\PYG{l+s}{\PYGZdq{}}\PYG{l+s}{F\PYGZdq{}}\PYG{p}{]}\PYG{+w}{ }\PYG{o}{+}\PYG{+w}{ }\PYG{n}{dat}\PYG{o}{\PYGZdl{}}\PYG{n}{POP}\PYG{p}{[}\PYG{n}{dat}\PYG{o}{\PYGZdl{}}\PYG{n}{Sex}\PYG{+w}{ }\PYG{o}{==}\PYG{+w}{ }\PYG{l+s}{\PYGZdq{}}\PYG{l+s}{M\PYGZdq{}}\PYG{p}{]}
\PYG{+w}{    }
\PYG{+w}{    }\PYG{c+c1}{\PYGZsh{} compute the cumulative percentage share of the population count}
\PYG{+w}{    }
\PYG{+w}{    }\PYG{n}{test}\PYG{+w}{ }\PYG{o}{\PYGZlt{}\PYGZhy{}}\PYG{+w}{ }\PYG{n+nf}{cumsum}\PYG{p}{(}\PYG{n}{pop}\PYG{p}{)}\PYG{o}{/}\PYG{n+nf}{sum}\PYG{p}{(}\PYG{n}{pop}\PYG{p}{)}
\PYG{+w}{    }
\PYG{+w}{    }\PYG{c+c1}{\PYGZsh{} copute the number of the element which is approximately nearest to 50 percent. We round the test}
\PYG{+w}{    }\PYG{c+c1}{\PYGZsh{} variable to two digits and give the result to the which function. which will return all row numbers which }
\PYG{+w}{    }\PYG{c+c1}{\PYGZsh{} fulfill the condition. The last one is nearest in the data to 50 \PYGZpc{}. We select the last element by the}
\PYG{+w}{    }\PYG{c+c1}{\PYGZsh{} tail function. This gives us the index allowing to select the age group that approximates the median age}
\PYG{+w}{    }\PYG{c+c1}{\PYGZsh{} group.}
\PYG{+w}{    }
\PYG{+w}{    }\PYG{n}{idx}\PYG{+w}{  }\PYG{o}{\PYGZlt{}\PYGZhy{}}\PYG{+w}{ }\PYG{n+nf}{tail}\PYG{p}{(}\PYG{n+nf}{which}\PYG{p}{(}\PYG{n+nf}{round}\PYG{p}{(}\PYG{n}{test}\PYG{p}{,}\PYG{l+m}{2}\PYG{p}{)}\PYG{+w}{ }\PYG{o}{\PYGZlt{}=}\PYG{+w}{ }\PYG{l+m}{0.50}\PYG{p}{)}\PYG{p}{,}\PYG{l+m}{1}\PYG{p}{)}
\PYG{+w}{    }
\PYG{+w}{    }\PYG{c+c1}{\PYGZsh{} Select the median age group. Since this is the last expression in the function it will be the output of}
\PYG{+w}{    }\PYG{c+c1}{\PYGZsh{} the function call.}
\PYG{+w}{    }
\PYG{+w}{    }\PYG{n}{data}\PYG{o}{\PYGZdl{}}\PYG{n}{Age}\PYG{p}{[}\PYG{n}{idx}\PYG{p}{]}
\PYG{+w}{    }
\PYG{p}{\PYGZcb{}}
\end{sphinxVerbatim}

\end{sphinxuseclass}\end{sphinxVerbatimInput}

\end{sphinxuseclass}
\begin{sphinxuseclass}{cell}\begin{sphinxVerbatimInput}

\begin{sphinxuseclass}{cell_input}
\begin{sphinxVerbatim}[commandchars=\\\{\}]
\PYG{n+nf}{median\PYGZus{}age\PYGZus{}group}\PYG{p}{(}\PYG{l+s}{\PYGZdq{}}\PYG{l+s}{Kenya\PYGZdq{}}\PYG{p}{,}\PYG{+w}{ }\PYG{l+m}{2022}\PYG{p}{,}\PYG{+w}{ }\PYG{n}{dat}\PYG{p}{)}
\end{sphinxVerbatim}

\end{sphinxuseclass}\end{sphinxVerbatimInput}
\begin{sphinxVerbatimOutput}

\begin{sphinxuseclass}{cell_output}\begin{equation*}
\begin{split}15-19
\emph{Levels}: \begin{enumerate*}
\item '0-4'
\item '5-9'
\item '10-14'
\item '15-19'
\item '20-24'
\item '25-29'
\item '30-34'
\item '35-39'
\item '40-44'
\item '45-49'
\item '50-54'
\item '55-59'
\item '60-64'
\item '65-69'
\item '70-74'
\item '75-79'
\item '80-84'
\item '85-89'
\item '90-94'
\item '95-99'
\end{enumerate*}\end{split}
\end{equation*}
\end{sphinxuseclass}\end{sphinxVerbatimOutput}

\end{sphinxuseclass}\begin{enumerate}
\sphinxsetlistlabels{\arabic}{enumi}{enumii}{}{.}%
\setcounter{enumi}{2}
\item {} 
\sphinxAtStartPar
Use the approximation procedure for grouped data to compute the mean and standard deviation of the age in the population of your country in 2022. Argue why the age distribution can not be approximated by a normal curve. Reuse
the computations you made in the last project step.

\end{enumerate}

\sphinxAtStartPar
My country was Austria. For Austria we computed last time for the approximate mean:

\begin{sphinxuseclass}{cell}\begin{sphinxVerbatimInput}

\begin{sphinxuseclass}{cell_input}
\begin{sphinxVerbatim}[commandchars=\\\{\}]
\PYG{c+c1}{\PYGZsh{} Data for Austria}

\PYG{n}{at\PYGZus{}2022}\PYG{+w}{ }\PYG{o}{\PYGZlt{}\PYGZhy{}}\PYG{+w}{ }\PYG{n}{dat}\PYG{p}{[}\PYG{n}{dat}\PYG{o}{\PYGZdl{}}\PYG{n}{Year}\PYG{+w}{ }\PYG{o}{==}\PYG{+w}{ }\PYG{l+m}{2022}\PYG{+w}{ }\PYG{o}{\PYGZam{}}\PYG{+w}{ }\PYG{n}{dat}\PYG{o}{\PYGZdl{}}\PYG{n}{ISO2}\PYG{+w}{ }\PYG{o}{==}\PYG{+w}{ }\PYG{l+s}{\PYGZdq{}}\PYG{l+s}{AT\PYGZdq{}}\PYG{p}{,}\PYG{+w}{ }\PYG{p}{]}

\PYG{c+c1}{\PYGZsh{} Generate sequence of midpoints for the age groups}

\PYG{n}{mid}\PYG{+w}{  }\PYG{o}{\PYGZlt{}\PYGZhy{}}\PYG{+w}{ }\PYG{n+nf}{seq}\PYG{p}{(}\PYG{l+m}{2}\PYG{p}{,}\PYG{l+m}{100}\PYG{p}{,}\PYG{l+m}{5}\PYG{p}{)}

\PYG{c+c1}{\PYGZsh{} Add midpoint variable to dataframe}

\PYG{n}{at\PYGZus{}2022}\PYG{o}{\PYGZdl{}}\PYG{n}{mid}\PYG{+w}{  }\PYG{o}{\PYGZlt{}\PYGZhy{}}\PYG{+w}{ }\PYG{n}{mid}

\PYG{c+c1}{\PYGZsh{} Compute the (approximate) mean}

\PYG{n}{m}\PYG{+w}{ }\PYG{o}{\PYGZlt{}\PYGZhy{}}\PYG{+w}{ }\PYG{n+nf}{sum}\PYG{p}{(}\PYG{n}{at\PYGZus{}2022}\PYG{o}{\PYGZdl{}}\PYG{n}{mid}\PYG{o}{*}\PYG{n}{at\PYGZus{}2022}\PYG{o}{\PYGZdl{}}\PYG{n}{POP}\PYG{p}{)}\PYG{o}{/}\PYG{n+nf}{sum}\PYG{p}{(}\PYG{n}{at\PYGZus{}2022}\PYG{o}{\PYGZdl{}}\PYG{n}{POP}\PYG{p}{)}

\PYG{c+c1}{\PYGZsh{} Compute the (approximate) standard deviation}

\PYG{n}{s}\PYG{+w}{  }\PYG{o}{\PYGZlt{}\PYGZhy{}}\PYG{+w}{ }\PYG{n+nf}{sqrt}\PYG{p}{(}\PYG{n+nf}{sum}\PYG{p}{(}\PYG{p}{(}\PYG{n}{at\PYGZus{}2022}\PYG{o}{\PYGZdl{}}\PYG{n}{mid}\PYG{+w}{ }\PYG{o}{\PYGZhy{}}\PYG{+w}{ }\PYG{n}{m}\PYG{p}{)}\PYG{o}{\PYGZca{}}\PYG{l+m}{2}\PYG{o}{*}\PYG{n}{at\PYGZus{}2022}\PYG{o}{\PYGZdl{}}\PYG{n}{POP}\PYG{p}{)}\PYG{o}{/}\PYG{n+nf}{sum}\PYG{p}{(}\PYG{n}{at\PYGZus{}2022}\PYG{o}{\PYGZdl{}}\PYG{n}{POP}\PYG{p}{)}\PYG{p}{)}

\PYG{n}{m}
\PYG{n}{s}
\end{sphinxVerbatim}

\end{sphinxuseclass}\end{sphinxVerbatimInput}
\begin{sphinxVerbatimOutput}

\begin{sphinxuseclass}{cell_output}\begin{equation*}
\begin{split}43.430240348906\end{split}
\end{equation*}\begin{equation*}
\begin{split}23.3493963356936\end{split}
\end{equation*}
\end{sphinxuseclass}\end{sphinxVerbatimOutput}

\end{sphinxuseclass}
\sphinxAtStartPar
Note again what we did here. Take for instance the age group 20 \sphinxhyphen{} 24. As an approximation we treat every observation in this age group as if it was age 22 \sphinxhyphen{} the midpoint. If we multiply by the population number in this age group we get the total age in this group. This is done for every age group added up and divided by the
total population accross all age groups. Then we get an approximate average.

\sphinxAtStartPar
For the standard deviation we proceed in a similar way. We subtract the approximate mean age from each observation and because we assume that each observation in an age group can be approximated by the mean age, we subtract the mean from this and square. This has to be multiplied by the number of the population to get the aggregate deviation in the age group. Sum up over age groups and divide by the total population and take the square root. Then you have the approximate standard deviation.

\sphinxAtStartPar
Now with 43 + 23 and 43 \sphinxhyphen{} 23 we get the interval 20 \sphinxhyphen{} 63. These boundaries roughly span the age groups 20\sphinxhyphen{}24 to 60\sphinxhyphen{}64 in our data. The percentage of observations between these groups is about:

\begin{sphinxuseclass}{cell}\begin{sphinxVerbatimInput}

\begin{sphinxuseclass}{cell_input}
\begin{sphinxVerbatim}[commandchars=\\\{\}]
\PYG{n}{table\PYGZus{}at}\PYG{o}{\PYGZdl{}}\PYG{n}{CumProp}\PYG{p}{[}\PYG{n}{table\PYGZus{}at}\PYG{o}{\PYGZdl{}}\PYG{n}{Age}\PYG{+w}{ }\PYG{o}{==}\PYG{+w}{ }\PYG{l+s}{\PYGZdq{}}\PYG{l+s}{60\PYGZhy{}64\PYGZdq{}}\PYG{p}{]}\PYG{+w}{ }\PYG{o}{\PYGZhy{}}\PYG{+w}{  }\PYG{n}{table\PYGZus{}at}\PYG{o}{\PYGZdl{}}\PYG{n}{CumProp}\PYG{p}{[}\PYG{n}{table\PYGZus{}at}\PYG{o}{\PYGZdl{}}\PYG{n}{Age}\PYG{+w}{ }\PYG{o}{==}\PYG{+w}{ }\PYG{l+s}{\PYGZdq{}}\PYG{l+s}{20\PYGZhy{}24\PYGZdq{}}\PYG{p}{]}
\end{sphinxVerbatim}

\end{sphinxuseclass}\end{sphinxVerbatimInput}
\begin{sphinxVerbatimOutput}

\begin{sphinxuseclass}{cell_output}\begin{equation*}
\begin{split}0.555304565290999\end{split}
\end{equation*}
\end{sphinxuseclass}\end{sphinxVerbatimOutput}

\end{sphinxuseclass}
\sphinxAtStartPar
This is about 55 \% of the observations. If age was normally distributed this would need to be 68 \%.

\sphinxstepscope


\chapter{Exercises: Unit 4 Making predictions using regression}
\label{\detokenize{exercises_unit_4:exercises-unit-4-making-predictions-using-regression}}\label{\detokenize{exercises_unit_4::doc}}

\section{Exercises}
\label{\detokenize{exercises_unit_4:exercises}}

\subsection{Exercise 1: Points in a coordinate system}
\label{\detokenize{exercises_unit_4:exercise-1-points-in-a-coordinate-system}}\begin{enumerate}
\sphinxsetlistlabels{\arabic}{enumi}{enumii}{}{.}%
\item {} 
\sphinxAtStartPar
A = (1,2), B = (4,4), C = (5,3), D = (5,1), E = (3,0)

\item {} 
\sphinxAtStartPar
x moves up 3 y moves up 2

\item {} 
\sphinxAtStartPar
D

\end{enumerate}


\subsection{Exercise 2: What is the slope}
\label{\detokenize{exercises_unit_4:exercise-2-what-is-the-slope}}
\sphinxAtStartPar
First slope is 0, second slope is \sphinxhyphen{}1


\subsection{Exercise 3: Where on the line are you?}
\label{\detokenize{exercises_unit_4:exercise-3-where-on-the-line-are-you}}
\sphinxAtStartPar
Line through (2,1) with slope 1. Note: You can sketch this by hand. I have done this here in R just for my convenience to put the plot into the notebook.

\begin{sphinxuseclass}{cell}\begin{sphinxVerbatimInput}

\begin{sphinxuseclass}{cell_input}
\begin{sphinxVerbatim}[commandchars=\\\{\}]
\PYG{n+nf}{plot.new}\PYG{p}{(}\PYG{p}{)}

\PYG{n+nf}{grid}\PYG{p}{(}\PYG{n}{nx}\PYG{+w}{ }\PYG{o}{=}\PYG{+w}{ }\PYG{l+m}{20}\PYG{p}{,}\PYG{+w}{ }\PYG{n}{ny}\PYG{+w}{ }\PYG{o}{=}\PYG{+w}{ }\PYG{l+m}{20}\PYG{p}{,}
\PYG{+w}{    }\PYG{n}{lty}\PYG{+w}{ }\PYG{o}{=}\PYG{+w}{ }\PYG{l+m}{2}\PYG{p}{,}\PYG{+w}{ }\PYG{n}{col}\PYG{+w}{ }\PYG{o}{=}\PYG{+w}{ }\PYG{l+s}{\PYGZdq{}}\PYG{l+s}{gray\PYGZdq{}}\PYG{p}{,}\PYG{+w}{ }\PYG{n}{lwd}\PYG{+w}{ }\PYG{o}{=}\PYG{+w}{ }\PYG{l+m}{2}\PYG{p}{)}

\PYG{n+nf}{par}\PYG{p}{(}\PYG{n}{new}\PYG{+w}{ }\PYG{o}{=}\PYG{+w}{ }\PYG{k+kc}{TRUE}\PYG{p}{)}

\PYG{n+nf}{curve}\PYG{p}{(}\PYG{l+m}{\PYGZhy{}1}\PYG{+w}{ }\PYG{o}{+}\PYG{+w}{ }\PYG{n}{x}\PYG{p}{,}\PYG{+w}{ }\PYG{n}{from}\PYG{+w}{ }\PYG{o}{=}\PYG{+w}{ }\PYG{l+m}{\PYGZhy{}10}\PYG{p}{,}\PYG{+w}{ }\PYG{n}{to}\PYG{+w}{ }\PYG{o}{=}\PYG{+w}{ }\PYG{l+m}{10}\PYG{p}{,}\PYG{+w}{ }\PYG{n}{lwd}\PYG{+w}{ }\PYG{o}{=}\PYG{+w}{ }\PYG{l+m}{2}\PYG{p}{,}\PYG{+w}{ }\PYG{n}{col}\PYG{+w}{ }\PYG{o}{=}\PYG{+w}{ }\PYG{l+s}{\PYGZdq{}}\PYG{l+s}{red\PYGZdq{}}\PYG{p}{)}
\PYG{n+nf}{abline}\PYG{p}{(}\PYG{n}{v}\PYG{+w}{ }\PYG{o}{=}\PYG{+w}{ }\PYG{l+m}{0}\PYG{p}{)}
\PYG{n+nf}{abline}\PYG{p}{(}\PYG{n}{h}\PYG{+w}{ }\PYG{o}{=}\PYG{+w}{ }\PYG{l+m}{0}\PYG{p}{)}
\end{sphinxVerbatim}

\end{sphinxuseclass}\end{sphinxVerbatimInput}
\begin{sphinxVerbatimOutput}

\begin{sphinxuseclass}{cell_output}
\noindent\sphinxincludegraphics{{87d4417ccfd80862ecf830a672437f496a77c02c946de66f122f3a0871e02dbb}.png}

\end{sphinxuseclass}\end{sphinxVerbatimOutput}

\end{sphinxuseclass}
\sphinxAtStartPar
Slope \sphinxhyphen{}1

\begin{sphinxuseclass}{cell}\begin{sphinxVerbatimInput}

\begin{sphinxuseclass}{cell_input}
\begin{sphinxVerbatim}[commandchars=\\\{\}]
\PYG{n+nf}{plot.new}\PYG{p}{(}\PYG{p}{)}

\PYG{n+nf}{grid}\PYG{p}{(}\PYG{n}{nx}\PYG{+w}{ }\PYG{o}{=}\PYG{+w}{ }\PYG{l+m}{20}\PYG{p}{,}\PYG{+w}{ }\PYG{n}{ny}\PYG{+w}{ }\PYG{o}{=}\PYG{+w}{ }\PYG{l+m}{20}\PYG{p}{,}
\PYG{+w}{    }\PYG{n}{lty}\PYG{+w}{ }\PYG{o}{=}\PYG{+w}{ }\PYG{l+m}{2}\PYG{p}{,}\PYG{+w}{ }\PYG{n}{col}\PYG{+w}{ }\PYG{o}{=}\PYG{+w}{ }\PYG{l+s}{\PYGZdq{}}\PYG{l+s}{gray\PYGZdq{}}\PYG{p}{,}\PYG{+w}{ }\PYG{n}{lwd}\PYG{+w}{ }\PYG{o}{=}\PYG{+w}{ }\PYG{l+m}{2}\PYG{p}{)}

\PYG{n+nf}{par}\PYG{p}{(}\PYG{n}{new}\PYG{+w}{ }\PYG{o}{=}\PYG{+w}{ }\PYG{k+kc}{TRUE}\PYG{p}{)}

\PYG{n+nf}{curve}\PYG{p}{(}\PYG{l+m}{3}\PYG{+w}{ }\PYG{o}{\PYGZhy{}}\PYG{+w}{ }\PYG{n}{x}\PYG{p}{,}\PYG{+w}{ }\PYG{n}{from}\PYG{+w}{ }\PYG{o}{=}\PYG{+w}{ }\PYG{l+m}{\PYGZhy{}10}\PYG{p}{,}\PYG{+w}{ }\PYG{n}{to}\PYG{+w}{ }\PYG{o}{=}\PYG{+w}{ }\PYG{l+m}{10}\PYG{p}{,}\PYG{+w}{ }\PYG{n}{lwd}\PYG{+w}{ }\PYG{o}{=}\PYG{+w}{ }\PYG{l+m}{2}\PYG{p}{,}\PYG{+w}{ }\PYG{n}{col}\PYG{+w}{ }\PYG{o}{=}\PYG{+w}{ }\PYG{l+s}{\PYGZdq{}}\PYG{l+s}{red\PYGZdq{}}\PYG{p}{)}
\PYG{n+nf}{abline}\PYG{p}{(}\PYG{n}{v}\PYG{+w}{ }\PYG{o}{=}\PYG{+w}{ }\PYG{l+m}{0}\PYG{p}{)}
\PYG{n+nf}{abline}\PYG{p}{(}\PYG{n}{h}\PYG{+w}{ }\PYG{o}{=}\PYG{+w}{ }\PYG{l+m}{0}\PYG{p}{)}
\end{sphinxVerbatim}

\end{sphinxuseclass}\end{sphinxVerbatimInput}
\begin{sphinxVerbatimOutput}

\begin{sphinxuseclass}{cell_output}
\noindent\sphinxincludegraphics{{18737d25edb1d384c23c621634455bb221177c0961a87f2830df366993e852b3}.png}

\end{sphinxuseclass}\end{sphinxVerbatimOutput}

\end{sphinxuseclass}
\sphinxAtStartPar
Slope 0

\begin{sphinxuseclass}{cell}\begin{sphinxVerbatimInput}

\begin{sphinxuseclass}{cell_input}
\begin{sphinxVerbatim}[commandchars=\\\{\}]
\PYG{n+nf}{plot.new}\PYG{p}{(}\PYG{p}{)}

\PYG{n+nf}{grid}\PYG{p}{(}\PYG{n}{nx}\PYG{+w}{ }\PYG{o}{=}\PYG{+w}{ }\PYG{l+m}{20}\PYG{p}{,}\PYG{+w}{ }\PYG{n}{ny}\PYG{+w}{ }\PYG{o}{=}\PYG{+w}{ }\PYG{l+m}{20}\PYG{p}{,}
\PYG{+w}{    }\PYG{n}{lty}\PYG{+w}{ }\PYG{o}{=}\PYG{+w}{ }\PYG{l+m}{2}\PYG{p}{,}\PYG{+w}{ }\PYG{n}{col}\PYG{+w}{ }\PYG{o}{=}\PYG{+w}{ }\PYG{l+s}{\PYGZdq{}}\PYG{l+s}{gray\PYGZdq{}}\PYG{p}{,}\PYG{+w}{ }\PYG{n}{lwd}\PYG{+w}{ }\PYG{o}{=}\PYG{+w}{ }\PYG{l+m}{2}\PYG{p}{)}

\PYG{n+nf}{par}\PYG{p}{(}\PYG{n}{new}\PYG{+w}{ }\PYG{o}{=}\PYG{+w}{ }\PYG{k+kc}{TRUE}\PYG{p}{)}

\PYG{n+nf}{curve}\PYG{p}{(}\PYG{l+m}{1}\PYG{+w}{ }\PYG{o}{\PYGZhy{}}\PYG{+w}{ }\PYG{l+m}{0}\PYG{o}{*}\PYG{n}{x}\PYG{p}{,}\PYG{+w}{ }\PYG{n}{from}\PYG{+w}{ }\PYG{o}{=}\PYG{+w}{ }\PYG{l+m}{\PYGZhy{}10}\PYG{p}{,}\PYG{+w}{ }\PYG{n}{to}\PYG{+w}{ }\PYG{o}{=}\PYG{+w}{ }\PYG{l+m}{10}\PYG{p}{,}\PYG{+w}{ }\PYG{n}{lwd}\PYG{+w}{ }\PYG{o}{=}\PYG{+w}{ }\PYG{l+m}{2}\PYG{p}{,}\PYG{+w}{ }\PYG{n}{col}\PYG{+w}{ }\PYG{o}{=}\PYG{+w}{ }\PYG{l+s}{\PYGZdq{}}\PYG{l+s}{red\PYGZdq{}}\PYG{p}{)}
\PYG{n+nf}{abline}\PYG{p}{(}\PYG{n}{v}\PYG{+w}{ }\PYG{o}{=}\PYG{+w}{ }\PYG{l+m}{0}\PYG{p}{)}

\PYG{n+nf}{abline}\PYG{p}{(}\PYG{n}{h}\PYG{+w}{ }\PYG{o}{=}\PYG{+w}{ }\PYG{l+m}{0}\PYG{p}{)}
\end{sphinxVerbatim}

\end{sphinxuseclass}\end{sphinxVerbatimInput}
\begin{sphinxVerbatimOutput}

\begin{sphinxuseclass}{cell_output}
\noindent\sphinxincludegraphics{{1c61e7bd516c458decda46771c171722eb789df88bf20335157f7aae65c66848}.png}

\end{sphinxuseclass}\end{sphinxVerbatimOutput}

\end{sphinxuseclass}
\sphinxAtStartPar
with slope 1: below the line, with slope \sphinxhyphen{}1: above the line, with slope 0 above the line

\begin{sphinxuseclass}{cell}\begin{sphinxVerbatimInput}

\begin{sphinxuseclass}{cell_input}
\begin{sphinxVerbatim}[commandchars=\\\{\}]
\PYG{n+nf}{plot.new}\PYG{p}{(}\PYG{p}{)}

\PYG{n+nf}{grid}\PYG{p}{(}\PYG{n}{nx}\PYG{+w}{ }\PYG{o}{=}\PYG{+w}{ }\PYG{l+m}{20}\PYG{p}{,}\PYG{+w}{ }\PYG{n}{ny}\PYG{+w}{ }\PYG{o}{=}\PYG{+w}{ }\PYG{l+m}{20}\PYG{p}{,}
\PYG{+w}{    }\PYG{n}{lty}\PYG{+w}{ }\PYG{o}{=}\PYG{+w}{ }\PYG{l+m}{2}\PYG{p}{,}\PYG{+w}{ }\PYG{n}{col}\PYG{+w}{ }\PYG{o}{=}\PYG{+w}{ }\PYG{l+s}{\PYGZdq{}}\PYG{l+s}{gray\PYGZdq{}}\PYG{p}{,}\PYG{+w}{ }\PYG{n}{lwd}\PYG{+w}{ }\PYG{o}{=}\PYG{+w}{ }\PYG{l+m}{2}\PYG{p}{)}

\PYG{n+nf}{par}\PYG{p}{(}\PYG{n}{new}\PYG{+w}{ }\PYG{o}{=}\PYG{+w}{ }\PYG{k+kc}{TRUE}\PYG{p}{)}

\PYG{n+nf}{curve}\PYG{p}{(}\PYG{l+m}{2}\PYG{+w}{ }\PYG{o}{\PYGZhy{}}\PYG{+w}{ }\PYG{n}{x}\PYG{p}{,}\PYG{+w}{ }\PYG{n}{from}\PYG{+w}{ }\PYG{o}{=}\PYG{+w}{ }\PYG{l+m}{\PYGZhy{}10}\PYG{p}{,}\PYG{+w}{ }\PYG{n}{to}\PYG{+w}{ }\PYG{o}{=}\PYG{+w}{ }\PYG{l+m}{10}\PYG{p}{,}\PYG{+w}{ }\PYG{n}{lwd}\PYG{+w}{ }\PYG{o}{=}\PYG{+w}{ }\PYG{l+m}{2}\PYG{p}{,}\PYG{+w}{ }\PYG{n}{col}\PYG{+w}{ }\PYG{o}{=}\PYG{+w}{ }\PYG{l+s}{\PYGZdq{}}\PYG{l+s}{blue\PYGZdq{}}\PYG{p}{)}
\PYG{n+nf}{abline}\PYG{p}{(}\PYG{n}{v}\PYG{+w}{ }\PYG{o}{=}\PYG{+w}{ }\PYG{l+m}{0}\PYG{p}{)}

\PYG{n+nf}{abline}\PYG{p}{(}\PYG{n}{h}\PYG{+w}{ }\PYG{o}{=}\PYG{+w}{ }\PYG{l+m}{0}\PYG{p}{)}
\end{sphinxVerbatim}

\end{sphinxuseclass}\end{sphinxVerbatimInput}
\begin{sphinxVerbatimOutput}

\begin{sphinxuseclass}{cell_output}
\noindent\sphinxincludegraphics{{7f1329c176c217b29234e2123942a8c17b0a10eb2efda477f1bad5ec294d75de}.png}

\end{sphinxuseclass}\end{sphinxVerbatimOutput}

\end{sphinxuseclass}
\begin{sphinxuseclass}{cell}\begin{sphinxVerbatimInput}

\begin{sphinxuseclass}{cell_input}
\begin{sphinxVerbatim}[commandchars=\\\{\}]
\PYG{n+nf}{plot.new}\PYG{p}{(}\PYG{p}{)}

\PYG{n+nf}{grid}\PYG{p}{(}\PYG{n}{nx}\PYG{+w}{ }\PYG{o}{=}\PYG{+w}{ }\PYG{l+m}{20}\PYG{p}{,}\PYG{+w}{ }\PYG{n}{ny}\PYG{+w}{ }\PYG{o}{=}\PYG{+w}{ }\PYG{l+m}{20}\PYG{p}{,}
\PYG{+w}{    }\PYG{n}{lty}\PYG{+w}{ }\PYG{o}{=}\PYG{+w}{ }\PYG{l+m}{2}\PYG{p}{,}\PYG{+w}{ }\PYG{n}{col}\PYG{+w}{ }\PYG{o}{=}\PYG{+w}{ }\PYG{l+s}{\PYGZdq{}}\PYG{l+s}{gray\PYGZdq{}}\PYG{p}{,}\PYG{+w}{ }\PYG{n}{lwd}\PYG{+w}{ }\PYG{o}{=}\PYG{+w}{ }\PYG{l+m}{2}\PYG{p}{)}

\PYG{n+nf}{par}\PYG{p}{(}\PYG{n}{new}\PYG{+w}{ }\PYG{o}{=}\PYG{+w}{ }\PYG{k+kc}{TRUE}\PYG{p}{)}

\PYG{n+nf}{curve}\PYG{p}{(}\PYG{l+m}{2}\PYG{+w}{ }\PYG{o}{+}\PYG{+w}{ }\PYG{n}{x}\PYG{p}{,}\PYG{+w}{ }\PYG{n}{from}\PYG{+w}{ }\PYG{o}{=}\PYG{+w}{ }\PYG{l+m}{\PYGZhy{}10}\PYG{p}{,}\PYG{+w}{ }\PYG{n}{to}\PYG{+w}{ }\PYG{o}{=}\PYG{+w}{ }\PYG{l+m}{10}\PYG{p}{,}\PYG{+w}{ }\PYG{n}{lwd}\PYG{+w}{ }\PYG{o}{=}\PYG{+w}{ }\PYG{l+m}{2}\PYG{p}{,}\PYG{+w}{ }\PYG{n}{col}\PYG{+w}{ }\PYG{o}{=}\PYG{+w}{ }\PYG{l+s}{\PYGZdq{}}\PYG{l+s}{blue\PYGZdq{}}\PYG{p}{)}
\PYG{n+nf}{abline}\PYG{p}{(}\PYG{n}{v}\PYG{+w}{ }\PYG{o}{=}\PYG{+w}{ }\PYG{l+m}{0}\PYG{p}{)}

\PYG{n+nf}{abline}\PYG{p}{(}\PYG{n}{h}\PYG{+w}{ }\PYG{o}{=}\PYG{+w}{ }\PYG{l+m}{0}\PYG{p}{)}
\end{sphinxVerbatim}

\end{sphinxuseclass}\end{sphinxVerbatimInput}
\begin{sphinxVerbatimOutput}

\begin{sphinxuseclass}{cell_output}
\noindent\sphinxincludegraphics{{47ed219c786d4b0d1837a361d3480579320dfb75d4718540d4c15c27b65d5812}.png}

\end{sphinxuseclass}\end{sphinxVerbatimOutput}

\end{sphinxuseclass}

\subsection{Exercise 4: Plotting graphs (by hand)}
\label{\detokenize{exercises_unit_4:exercise-4-plotting-graphs-by-hand}}
\sphinxAtStartPar
Again here I allow myself to plot the graphs by R instead of by hand:

\begin{sphinxuseclass}{cell}\begin{sphinxVerbatimInput}

\begin{sphinxuseclass}{cell_input}
\begin{sphinxVerbatim}[commandchars=\\\{\}]
\PYG{n+nf}{plot.new}\PYG{p}{(}\PYG{p}{)}

\PYG{n+nf}{grid}\PYG{p}{(}\PYG{n}{nx}\PYG{+w}{ }\PYG{o}{=}\PYG{+w}{ }\PYG{l+m}{20}\PYG{p}{,}\PYG{+w}{ }\PYG{n}{ny}\PYG{+w}{ }\PYG{o}{=}\PYG{+w}{ }\PYG{l+m}{20}\PYG{p}{,}
\PYG{+w}{    }\PYG{n}{lty}\PYG{+w}{ }\PYG{o}{=}\PYG{+w}{ }\PYG{l+m}{2}\PYG{p}{,}\PYG{+w}{ }\PYG{n}{col}\PYG{+w}{ }\PYG{o}{=}\PYG{+w}{ }\PYG{l+s}{\PYGZdq{}}\PYG{l+s}{gray\PYGZdq{}}\PYG{p}{,}\PYG{+w}{ }\PYG{n}{lwd}\PYG{+w}{ }\PYG{o}{=}\PYG{+w}{ }\PYG{l+m}{2}\PYG{p}{)}

\PYG{n+nf}{par}\PYG{p}{(}\PYG{n}{new}\PYG{+w}{ }\PYG{o}{=}\PYG{+w}{ }\PYG{k+kc}{TRUE}\PYG{p}{)}

\PYG{n+nf}{curve}\PYG{p}{(}\PYG{l+m}{1}\PYG{+w}{ }\PYG{o}{+}\PYG{+w}{ }\PYG{l+m}{2}\PYG{o}{*}\PYG{n}{x}\PYG{p}{,}\PYG{+w}{ }\PYG{n}{from}\PYG{+w}{ }\PYG{o}{=}\PYG{+w}{ }\PYG{l+m}{\PYGZhy{}5}\PYG{p}{,}\PYG{+w}{ }\PYG{n}{to}\PYG{+w}{ }\PYG{o}{=}\PYG{+w}{ }\PYG{l+m}{5}\PYG{p}{,}\PYG{+w}{ }\PYG{n}{lwd}\PYG{+w}{ }\PYG{o}{=}\PYG{+w}{ }\PYG{l+m}{2}\PYG{p}{,}\PYG{+w}{ }\PYG{n}{col}\PYG{+w}{ }\PYG{o}{=}\PYG{+w}{ }\PYG{l+s}{\PYGZdq{}}\PYG{l+s}{blue\PYGZdq{}}\PYG{p}{)}
\PYG{n+nf}{abline}\PYG{p}{(}\PYG{n}{v}\PYG{+w}{ }\PYG{o}{=}\PYG{+w}{ }\PYG{l+m}{0}\PYG{p}{)}

\PYG{n+nf}{abline}\PYG{p}{(}\PYG{n}{h}\PYG{+w}{ }\PYG{o}{=}\PYG{+w}{ }\PYG{l+m}{0}\PYG{p}{)}
\end{sphinxVerbatim}

\end{sphinxuseclass}\end{sphinxVerbatimInput}
\begin{sphinxVerbatimOutput}

\begin{sphinxuseclass}{cell_output}
\noindent\sphinxincludegraphics{{168743bbe5063e10c933375811987e086f46f508a3e8540a0f03d337360c5bb9}.png}

\end{sphinxuseclass}\end{sphinxVerbatimOutput}

\end{sphinxuseclass}
\sphinxAtStartPar
The height of the line at x = 2 is 5, the slope is 2

\begin{sphinxuseclass}{cell}\begin{sphinxVerbatimInput}

\begin{sphinxuseclass}{cell_input}
\begin{sphinxVerbatim}[commandchars=\\\{\}]
\PYG{n+nf}{plot.new}\PYG{p}{(}\PYG{p}{)}

\PYG{n+nf}{grid}\PYG{p}{(}\PYG{n}{nx}\PYG{+w}{ }\PYG{o}{=}\PYG{+w}{ }\PYG{l+m}{20}\PYG{p}{,}\PYG{+w}{ }\PYG{n}{ny}\PYG{+w}{ }\PYG{o}{=}\PYG{+w}{ }\PYG{l+m}{20}\PYG{p}{,}
\PYG{+w}{    }\PYG{n}{lty}\PYG{+w}{ }\PYG{o}{=}\PYG{+w}{ }\PYG{l+m}{2}\PYG{p}{,}\PYG{+w}{ }\PYG{n}{col}\PYG{+w}{ }\PYG{o}{=}\PYG{+w}{ }\PYG{l+s}{\PYGZdq{}}\PYG{l+s}{gray\PYGZdq{}}\PYG{p}{,}\PYG{+w}{ }\PYG{n}{lwd}\PYG{+w}{ }\PYG{o}{=}\PYG{+w}{ }\PYG{l+m}{2}\PYG{p}{)}

\PYG{n+nf}{par}\PYG{p}{(}\PYG{n}{new}\PYG{+w}{ }\PYG{o}{=}\PYG{+w}{ }\PYG{k+kc}{TRUE}\PYG{p}{)}

\PYG{n+nf}{curve}\PYG{p}{(}\PYG{l+m}{2}\PYG{+w}{ }\PYG{o}{+}\PYG{+w}{ }\PYG{l+m}{0.5}\PYG{o}{*}\PYG{n}{x}\PYG{p}{,}\PYG{+w}{ }\PYG{n}{from}\PYG{+w}{ }\PYG{o}{=}\PYG{+w}{ }\PYG{l+m}{\PYGZhy{}10}\PYG{p}{,}\PYG{+w}{ }\PYG{n}{to}\PYG{+w}{ }\PYG{o}{=}\PYG{+w}{ }\PYG{l+m}{10}\PYG{p}{,}\PYG{+w}{ }\PYG{n}{lwd}\PYG{+w}{ }\PYG{o}{=}\PYG{+w}{ }\PYG{l+m}{2}\PYG{p}{,}\PYG{+w}{ }\PYG{n}{col}\PYG{+w}{ }\PYG{o}{=}\PYG{+w}{ }\PYG{l+s}{\PYGZdq{}}\PYG{l+s}{blue\PYGZdq{}}\PYG{p}{)}
\PYG{n+nf}{abline}\PYG{p}{(}\PYG{n}{v}\PYG{+w}{ }\PYG{o}{=}\PYG{+w}{ }\PYG{l+m}{0}\PYG{p}{)}

\PYG{n+nf}{abline}\PYG{p}{(}\PYG{n}{h}\PYG{+w}{ }\PYG{o}{=}\PYG{+w}{ }\PYG{l+m}{0}\PYG{p}{)}
\end{sphinxVerbatim}

\end{sphinxuseclass}\end{sphinxVerbatimInput}
\begin{sphinxVerbatimOutput}

\begin{sphinxuseclass}{cell_output}
\noindent\sphinxincludegraphics{{baef2cda84224b9be6f0d8eac8a8300f0c972f8e914ce78c55d818b2d0a514a7}.png}

\end{sphinxuseclass}\end{sphinxVerbatimOutput}

\end{sphinxuseclass}
\sphinxAtStartPar
The height of the line at x = 2 is 3 and the slope is 1/2

\begin{sphinxuseclass}{cell}\begin{sphinxVerbatimInput}

\begin{sphinxuseclass}{cell_input}
\begin{sphinxVerbatim}[commandchars=\\\{\}]
\PYG{n}{dat}\PYG{+w}{  }\PYG{o}{\PYGZlt{}\PYGZhy{}}\PYG{+w}{ }\PYG{n+nf}{data.frame}\PYG{p}{(}\PYG{n}{x}\PYG{+w}{ }\PYG{o}{=}\PYG{+w}{ }\PYG{n+nf}{c}\PYG{p}{(}\PYG{l+m}{1}\PYG{o}{:}\PYG{l+m}{4}\PYG{p}{)}\PYG{p}{,}\PYG{+w}{ }\PYG{n}{y}\PYG{+w}{ }\PYG{o}{=}\PYG{+w}{ }\PYG{n+nf}{c}\PYG{p}{(}\PYG{l+m}{1}\PYG{o}{:}\PYG{l+m}{4}\PYG{p}{)}\PYG{o}{*}\PYG{l+m}{2}\PYG{p}{)}
\PYG{n+nf}{plot}\PYG{p}{(}\PYG{n}{dat}\PYG{p}{,}\PYG{+w}{ }\PYG{n}{pch}\PYG{+w}{ }\PYG{o}{=}\PYG{+w}{ }\PYG{l+m}{19}\PYG{p}{,}\PYG{+w}{ }\PYG{n}{col}\PYG{+w}{ }\PYG{o}{=}\PYG{+w}{ }\PYG{l+m}{4}\PYG{p}{)}
\end{sphinxVerbatim}

\end{sphinxuseclass}\end{sphinxVerbatimInput}
\begin{sphinxVerbatimOutput}

\begin{sphinxuseclass}{cell_output}
\noindent\sphinxincludegraphics{{26e46741b56005f8fd27eee8fa30d290de0260fd305bc2ce9f22f5465b68d633}.png}

\end{sphinxuseclass}\end{sphinxVerbatimOutput}

\end{sphinxuseclass}
\sphinxAtStartPar
All points lie on a line. The equation of the line is \(y = 2*x\) the slope is 1/2

\begin{sphinxuseclass}{cell}\begin{sphinxVerbatimInput}

\begin{sphinxuseclass}{cell_input}
\begin{sphinxVerbatim}[commandchars=\\\{\}]
\PYG{n+nf}{plot}\PYG{p}{(}\PYG{n}{dat}\PYG{p}{,}\PYG{+w}{ }\PYG{n}{pch}\PYG{+w}{ }\PYG{o}{=}\PYG{+w}{ }\PYG{l+m}{19}\PYG{p}{,}\PYG{+w}{ }\PYG{n}{col}\PYG{+w}{ }\PYG{o}{=}\PYG{+w}{ }\PYG{l+m}{4}\PYG{p}{)}
\PYG{n+nf}{lines}\PYG{p}{(}\PYG{n}{dat}\PYG{o}{\PYGZdl{}}\PYG{n}{x}\PYG{p}{,}\PYG{n}{dat}\PYG{o}{\PYGZdl{}}\PYG{n}{y}\PYG{p}{,}\PYG{+w}{ }\PYG{n}{col}\PYG{+w}{ }\PYG{o}{=}\PYG{+w}{ }\PYG{l+s}{\PYGZdq{}}\PYG{l+s}{red\PYGZdq{}}\PYG{p}{)}
\end{sphinxVerbatim}

\end{sphinxuseclass}\end{sphinxVerbatimInput}
\begin{sphinxVerbatimOutput}

\begin{sphinxuseclass}{cell_output}
\noindent\sphinxincludegraphics{{851e68158429f244e71818a0eb8b0974c3a084bd3bcb0266fd601475f17ce4c5}.png}

\end{sphinxuseclass}\end{sphinxVerbatimOutput}

\end{sphinxuseclass}
\begin{sphinxuseclass}{cell}\begin{sphinxVerbatimInput}

\begin{sphinxuseclass}{cell_input}
\begin{sphinxVerbatim}[commandchars=\\\{\}]
\PYG{c+c1}{\PYGZsh{} Compare the parametric curve to the curve from the data}

\PYG{n+nf}{curve}\PYG{p}{(}\PYG{l+m}{2}\PYG{o}{*}\PYG{n}{x}\PYG{p}{,}\PYG{+w}{ }\PYG{n}{from}\PYG{+w}{ }\PYG{o}{=}\PYG{+w}{ }\PYG{l+m}{1}\PYG{p}{,}\PYG{+w}{ }\PYG{n}{to}\PYG{+w}{ }\PYG{o}{=}\PYG{+w}{ }\PYG{l+m}{4}\PYG{p}{)}
\end{sphinxVerbatim}

\end{sphinxuseclass}\end{sphinxVerbatimInput}
\begin{sphinxVerbatimOutput}

\begin{sphinxuseclass}{cell_output}
\noindent\sphinxincludegraphics{{7d0cfe4908043d2198a769618c664435fad4f0b364b41c9b9342b83b02ff243d}.png}

\end{sphinxuseclass}\end{sphinxVerbatimOutput}

\end{sphinxuseclass}
\begin{sphinxuseclass}{cell}\begin{sphinxVerbatimInput}

\begin{sphinxuseclass}{cell_input}
\begin{sphinxVerbatim}[commandchars=\\\{\}]
\PYG{n}{data}\PYG{+w}{  }\PYG{o}{\PYGZlt{}\PYGZhy{}}\PYG{+w}{ }\PYG{n+nf}{data.frame}\PYG{p}{(}\PYG{n}{x}\PYG{+w}{ }\PYG{o}{=}\PYG{+w}{ }\PYG{n+nf}{c}\PYG{p}{(}\PYG{l+m}{1}\PYG{o}{:}\PYG{l+m}{4}\PYG{p}{)}\PYG{p}{,}\PYG{+w}{ }\PYG{n}{y}\PYG{o}{=}\PYG{n+nf}{c}\PYG{p}{(}\PYG{l+m}{1}\PYG{o}{:}\PYG{l+m}{4}\PYG{p}{)}\PYG{p}{)}
\PYG{n+nf}{plot}\PYG{p}{(}\PYG{n}{data}\PYG{p}{,}\PYG{+w}{ }\PYG{n}{pch}\PYG{+w}{ }\PYG{o}{=}\PYG{+w}{ }\PYG{l+m}{19}\PYG{p}{)}
\PYG{n+nf}{lines}\PYG{p}{(}\PYG{n}{data}\PYG{o}{\PYGZdl{}}\PYG{n}{x}\PYG{p}{,}\PYG{+w}{ }\PYG{n}{data}\PYG{o}{\PYGZdl{}}\PYG{n}{y}\PYG{p}{,}\PYG{+w}{ }\PYG{n}{col}\PYG{+w}{ }\PYG{o}{=}\PYG{+w}{ }\PYG{l+s}{\PYGZdq{}}\PYG{l+s}{blue\PYGZdq{}}\PYG{p}{)}
\end{sphinxVerbatim}

\end{sphinxuseclass}\end{sphinxVerbatimInput}
\begin{sphinxVerbatimOutput}

\begin{sphinxuseclass}{cell_output}
\noindent\sphinxincludegraphics{{cfb39758bc4d786cf234fe0ee1630bce43e395a450e9675235ec368b7597d2c6}.png}

\end{sphinxuseclass}\end{sphinxVerbatimOutput}

\end{sphinxuseclass}
\sphinxAtStartPar
The equation of the line is \(y = x\).
\begin{enumerate}
\sphinxsetlistlabels{\arabic}{enumi}{enumii}{}{.}%
\setcounter{enumi}{3}
\item {} 
\sphinxAtStartPar
false, false, false

\end{enumerate}


\subsection{Exercise 5: Where is the point?}
\label{\detokenize{exercises_unit_4:exercise-5-where-is-the-point}}
\begin{sphinxuseclass}{cell}\begin{sphinxVerbatimInput}

\begin{sphinxuseclass}{cell_input}
\begin{sphinxVerbatim}[commandchars=\\\{\}]
\PYG{n}{aux}\PYG{+w}{  }\PYG{o}{\PYGZlt{}\PYGZhy{}}\PYG{+w}{ }\PYG{n+nf}{data.frame}\PYG{p}{(}\PYG{n}{x}\PYG{+w}{ }\PYG{o}{=}\PYG{+w}{ }\PYG{n+nf}{c}\PYG{p}{(}\PYG{l+m}{0}\PYG{p}{,}\PYG{l+m}{0.5}\PYG{p}{,}\PYG{l+m}{1.5}\PYG{p}{,}\PYG{l+m}{2.5}\PYG{p}{)}\PYG{p}{,}\PYG{+w}{ }\PYG{n}{y}\PYG{o}{=}\PYG{+w}{ }\PYG{n+nf}{c}\PYG{p}{(}\PYG{l+m}{0}\PYG{p}{,}\PYG{+w}{ }\PYG{l+m}{0.5}\PYG{p}{,}\PYG{+w}{ }\PYG{l+m}{2}\PYG{p}{,}\PYG{+w}{ }\PYG{l+m}{2.5}\PYG{p}{)}\PYG{p}{)}
\PYG{n+nf}{plot}\PYG{p}{(}\PYG{n}{aux}\PYG{p}{,}\PYG{+w}{ }\PYG{n}{pch}\PYG{+w}{ }\PYG{o}{=}\PYG{+w}{ }\PYG{l+m}{19}\PYG{p}{)}
\end{sphinxVerbatim}

\end{sphinxuseclass}\end{sphinxVerbatimInput}
\begin{sphinxVerbatimOutput}

\begin{sphinxuseclass}{cell_output}
\noindent\sphinxincludegraphics{{ebdb937b5dba9eab1c33e50c77717225cdde59f3f9f6e9711c048a0f037d5142}.png}

\end{sphinxuseclass}\end{sphinxVerbatimOutput}

\end{sphinxuseclass}
\sphinxAtStartPar
(1.5,2) does not lie on the line but above.

\sphinxAtStartPar
The blanks are 7 and 9. We can plot the points:

\begin{sphinxuseclass}{cell}\begin{sphinxVerbatimInput}

\begin{sphinxuseclass}{cell_input}
\begin{sphinxVerbatim}[commandchars=\\\{\}]
\PYG{n}{p}\PYG{+w}{  }\PYG{o}{\PYGZlt{}\PYGZhy{}}\PYG{+w}{ }\PYG{n+nf}{data.frame}\PYG{p}{(}\PYG{n}{x}\PYG{+w}{ }\PYG{o}{=}\PYG{+w}{ }\PYG{n+nf}{c}\PYG{p}{(}\PYG{l+m}{1}\PYG{o}{:}\PYG{l+m}{4}\PYG{p}{)}\PYG{p}{,}\PYG{+w}{ }\PYG{n}{y}\PYG{o}{=}\PYG{+w}{ }\PYG{n+nf}{c}\PYG{p}{(}\PYG{l+m}{3}\PYG{p}{,}\PYG{l+m}{5}\PYG{p}{,}\PYG{l+m}{7}\PYG{p}{,}\PYG{l+m}{9}\PYG{p}{)}\PYG{p}{)}
\PYG{n+nf}{plot}\PYG{p}{(}\PYG{n}{p}\PYG{p}{,}\PYG{+w}{ }\PYG{n}{pch}\PYG{+w}{ }\PYG{o}{=}\PYG{+w}{ }\PYG{l+m}{19}\PYG{p}{)}
\end{sphinxVerbatim}

\end{sphinxuseclass}\end{sphinxVerbatimInput}
\begin{sphinxVerbatimOutput}

\begin{sphinxuseclass}{cell_output}
\noindent\sphinxincludegraphics{{45611957dbf2707333756944291fdfa08eac3ccdc98e38f9dd8bb9b7258ced0d}.png}

\end{sphinxuseclass}\end{sphinxVerbatimOutput}

\end{sphinxuseclass}
\sphinxAtStartPar
They are all on a line.

\sphinxAtStartPar
(1,2) is in b and c but not in a, (2,1) is in a but not in b or c.


\subsection{Exercise 6: The mathematics of lines}
\label{\detokenize{exercises_unit_4:exercise-6-the-mathematics-of-lines}}
\sphinxAtStartPar
The equation of the line is \(y = -x + 4\). At \(x=1\) the height is 3.

\begin{sphinxuseclass}{cell}\begin{sphinxVerbatimInput}

\begin{sphinxuseclass}{cell_input}
\begin{sphinxVerbatim}[commandchars=\\\{\}]
\PYG{n+nf}{plot.new}\PYG{p}{(}\PYG{p}{)}

\PYG{n+nf}{grid}\PYG{p}{(}\PYG{n}{nx}\PYG{+w}{ }\PYG{o}{=}\PYG{+w}{ }\PYG{l+m}{16}\PYG{p}{,}\PYG{+w}{ }\PYG{n}{ny}\PYG{+w}{ }\PYG{o}{=}\PYG{+w}{ }\PYG{l+m}{16}\PYG{p}{,}
\PYG{+w}{    }\PYG{n}{lty}\PYG{+w}{ }\PYG{o}{=}\PYG{+w}{ }\PYG{l+m}{2}\PYG{p}{,}\PYG{+w}{ }\PYG{n}{col}\PYG{+w}{ }\PYG{o}{=}\PYG{+w}{ }\PYG{l+s}{\PYGZdq{}}\PYG{l+s}{gray\PYGZdq{}}\PYG{p}{,}\PYG{+w}{ }\PYG{n}{lwd}\PYG{+w}{ }\PYG{o}{=}\PYG{+w}{ }\PYG{l+m}{2}\PYG{p}{)}

\PYG{n+nf}{par}\PYG{p}{(}\PYG{n}{new}\PYG{+w}{ }\PYG{o}{=}\PYG{+w}{ }\PYG{k+kc}{TRUE}\PYG{p}{)}

\PYG{n+nf}{curve}\PYG{p}{(}\PYG{l+m}{4}\PYG{+w}{ }\PYG{o}{\PYGZhy{}}\PYG{p}{(}\PYG{l+m}{1}\PYG{o}{/}\PYG{l+m}{2}\PYG{p}{)}\PYG{o}{*}\PYG{n}{x}\PYG{p}{,}\PYG{+w}{ }\PYG{n}{from}\PYG{+w}{ }\PYG{o}{=}\PYG{+w}{ }\PYG{l+m}{\PYGZhy{}8}\PYG{p}{,}\PYG{+w}{ }\PYG{n}{to}\PYG{+w}{ }\PYG{o}{=}\PYG{+w}{ }\PYG{l+m}{8}\PYG{p}{,}\PYG{+w}{ }\PYG{n}{lwd}\PYG{+w}{ }\PYG{o}{=}\PYG{+w}{ }\PYG{l+m}{2}\PYG{p}{,}\PYG{+w}{ }\PYG{n}{col}\PYG{+w}{ }\PYG{o}{=}\PYG{+w}{ }\PYG{l+s}{\PYGZdq{}}\PYG{l+s}{blue\PYGZdq{}}\PYG{p}{)}
\PYG{n+nf}{abline}\PYG{p}{(}\PYG{n}{v}\PYG{+w}{ }\PYG{o}{=}\PYG{+w}{ }\PYG{l+m}{0}\PYG{p}{)}

\PYG{n+nf}{abline}\PYG{p}{(}\PYG{n}{h}\PYG{+w}{ }\PYG{o}{=}\PYG{+w}{ }\PYG{l+m}{0}\PYG{p}{)}
\end{sphinxVerbatim}

\end{sphinxuseclass}\end{sphinxVerbatimInput}
\begin{sphinxVerbatimOutput}

\begin{sphinxuseclass}{cell_output}
\noindent\sphinxincludegraphics{{277001ed00382096447f47f0e8c2e4e9dc4309c42a96a9bcbb652836239b53c6}.png}

\end{sphinxuseclass}\end{sphinxVerbatimOutput}

\end{sphinxuseclass}

\subsection{Exercise 7: Comparing correlations}
\label{\detokenize{exercises_unit_4:exercise-7-comparing-correlations}}
\sphinxAtStartPar
2 stronger than 3 stronger than 4 stronger than 1


\subsection{Exercise 8: Interpreting correlation}
\label{\detokenize{exercises_unit_4:exercise-8-interpreting-correlation}}
\sphinxAtStartPar
No it is as such without further evidence not a causal relation. Discuss potential confounders, reasons that make people read less and watch more TV. It coul for instance be the case that people who have trouble reading watch more TV, so causality would run in the other direction. After all the correlation between \(x\) and \(y\) is the same as between \(y\) and \(x\).


\subsection{Exercise 9: Correlation between height and weight}
\label{\detokenize{exercises_unit_4:exercise-9-correlation-between-height-and-weight}}
\sphinxAtStartPar
true, true, true, false


\subsection{Exercise 10: Explain}
\label{\detokenize{exercises_unit_4:exercise-10-explain}}
\sphinxAtStartPar
Standard units allow us to compare the strengths of linear relations, because in standard units values that are larger than the mean are positive and values that are less than the mean are negative.


\section{Exercises R}
\label{\detokenize{exercises_unit_4:exercises-r}}

\subsection{Exercise 1: Life expectancy}
\label{\detokenize{exercises_unit_4:exercise-1-life-expectancy}}
\begin{sphinxuseclass}{cell}\begin{sphinxVerbatimInput}

\begin{sphinxuseclass}{cell_input}
\begin{sphinxVerbatim}[commandchars=\\\{\}]
\PYG{n+nf}{curve}\PYG{p}{(}\PYG{l+m}{67.78}\PYG{+w}{ }\PYG{o}{+}\PYG{+w}{ }\PYG{l+m}{0.0002549}\PYG{o}{*}\PYG{n}{x}\PYG{p}{,}\PYG{+w}{ }\PYG{n}{from}\PYG{+w}{ }\PYG{o}{=}\PYG{+w}{ }\PYG{l+m}{0}\PYG{p}{,}\PYG{+w}{ }\PYG{n}{to}\PYG{+w}{ }\PYG{o}{=}\PYG{+w}{ }\PYG{l+m}{10000}\PYG{p}{,}
\PYG{+w}{     }\PYG{n}{main}\PYG{+w}{ }\PYG{o}{=}\PYG{+w}{ }\PYG{l+s}{\PYGZdq{}}\PYG{l+s}{Relation of GDP per capita and life expectancy\PYGZdq{}}\PYG{p}{,}
\PYG{+w}{     }\PYG{n}{xlab}\PYG{+w}{ }\PYG{o}{=}\PYG{+w}{ }\PYG{l+s}{\PYGZdq{}}\PYG{l+s}{GDP per capita in international Dollars at 2011 prices\PYGZdq{}}\PYG{p}{,}
\PYG{+w}{     }\PYG{n}{ylab}\PYG{+w}{ }\PYG{o}{=}\PYG{+w}{ }\PYG{l+s}{\PYGZdq{}}\PYG{l+s}{Average life expectancy at birth in years\PYGZdq{}}\PYG{+w}{    }
\PYG{+w}{     }\PYG{p}{)}
\end{sphinxVerbatim}

\end{sphinxuseclass}\end{sphinxVerbatimInput}
\begin{sphinxVerbatimOutput}

\begin{sphinxuseclass}{cell_output}
\noindent\sphinxincludegraphics{{07fd25fabb2c44133c307635e83d1bace07673a480752f66459be09a602260eb}.png}

\end{sphinxuseclass}\end{sphinxVerbatimOutput}

\end{sphinxuseclass}
\begin{sphinxuseclass}{cell}\begin{sphinxVerbatimInput}

\begin{sphinxuseclass}{cell_input}
\begin{sphinxVerbatim}[commandchars=\\\{\}]
\PYG{n+nf}{library}\PYG{p}{(}\PYG{n}{JWL}\PYG{p}{)}
\PYG{n}{lex}\PYG{+w}{  }\PYG{o}{\PYGZlt{}\PYGZhy{}}\PYG{+w}{ }\PYG{n}{life\PYGZus{}expectancy}
\end{sphinxVerbatim}

\end{sphinxuseclass}\end{sphinxVerbatimInput}

\end{sphinxuseclass}
\begin{sphinxuseclass}{cell}\begin{sphinxVerbatimInput}

\begin{sphinxuseclass}{cell_input}
\begin{sphinxVerbatim}[commandchars=\\\{\}]
\PYG{n+nf}{head}\PYG{p}{(}\PYG{n}{lex}\PYG{p}{)}
\end{sphinxVerbatim}

\end{sphinxuseclass}\end{sphinxVerbatimInput}
\begin{sphinxVerbatimOutput}

\begin{sphinxuseclass}{cell_output}\begin{equation*}
\begin{split}A data.frame: 6 × 5
\begin{tabular}{r|lllll}
  & Entity & Code & Year & LE & GDPC\\
  & <chr> & <chr> & <int> & <dbl> & <dbl>\\
\hline
	2 & Afghanistan & AFG & 1950 & 27.7275 & 1156\\
	3 & Afghanistan & AFG & 1951 & 27.9634 & 1170\\
	4 & Afghanistan & AFG & 1952 & 28.4456 & 1189\\
	5 & Afghanistan & AFG & 1953 & 28.9304 & 1240\\
	6 & Afghanistan & AFG & 1954 & 29.2258 & 1245\\
	7 & Afghanistan & AFG & 1955 & 29.9206 & 1246\\
\end{tabular}\end{split}
\end{equation*}
\end{sphinxuseclass}\end{sphinxVerbatimOutput}

\end{sphinxuseclass}
\begin{sphinxuseclass}{cell}\begin{sphinxVerbatimInput}

\begin{sphinxuseclass}{cell_input}
\begin{sphinxVerbatim}[commandchars=\\\{\}]
\PYG{n}{lex\PYGZus{}2018}\PYG{+w}{  }\PYG{o}{\PYGZlt{}\PYGZhy{}}\PYG{+w}{ }\PYG{n}{lex}\PYG{p}{[}\PYG{n}{lex}\PYG{o}{\PYGZdl{}}\PYG{n}{Year}\PYG{+w}{ }\PYG{o}{==}\PYG{+w}{ }\PYG{l+m}{2018}\PYG{p}{,}\PYG{+w}{ }\PYG{p}{]}
\end{sphinxVerbatim}

\end{sphinxuseclass}\end{sphinxVerbatimInput}

\end{sphinxuseclass}
\begin{sphinxuseclass}{cell}\begin{sphinxVerbatimInput}

\begin{sphinxuseclass}{cell_input}
\begin{sphinxVerbatim}[commandchars=\\\{\}]
\PYG{n+nf}{head}\PYG{p}{(}\PYG{n}{lex\PYGZus{}2018}\PYG{p}{)}
\end{sphinxVerbatim}

\end{sphinxuseclass}\end{sphinxVerbatimInput}
\begin{sphinxVerbatimOutput}

\begin{sphinxuseclass}{cell_output}\begin{equation*}
\begin{split}A data.frame: 6 × 5
\begin{tabular}{r|lllll}
  & Entity & Code & Year & LE & GDPC\\
  & <chr> & <chr> & <int> & <dbl> & <dbl>\\
\hline
	70 & Afghanistan & AFG & 2018 & 63.0810 &  1934.555\\
	665 & Albania     & ALB & 2018 & 79.1838 & 11104.166\\
	924 & Algeria     & DZA & 2018 & 76.0656 & 14228.025\\
	1634 & Angola      & AGO & 2018 & 62.1438 &  7771.442\\
	2277 & Argentina   & ARG & 2018 & 76.9994 & 18556.383\\
	2536 & Armenia     & ARM & 2018 & 75.0645 & 11454.425\\
\end{tabular}\end{split}
\end{equation*}
\end{sphinxuseclass}\end{sphinxVerbatimOutput}

\end{sphinxuseclass}
\begin{sphinxuseclass}{cell}\begin{sphinxVerbatimInput}

\begin{sphinxuseclass}{cell_input}
\begin{sphinxVerbatim}[commandchars=\\\{\}]
\PYG{n}{sel}\PYG{+w}{  }\PYG{o}{\PYGZlt{}\PYGZhy{}}\PYG{+w}{ }\PYG{n+nf}{c}\PYG{p}{(}\PYG{l+s}{\PYGZdq{}}\PYG{l+s}{Chad\PYGZdq{}}\PYG{p}{,}\PYG{+w}{ }\PYG{l+s}{\PYGZdq{}}\PYG{l+s}{Egypt\PYGZdq{}}\PYG{p}{,}\PYG{+w}{ }\PYG{l+s}{\PYGZdq{}}\PYG{l+s}{Kenya\PYGZdq{}}\PYG{p}{,}\PYG{+w}{ }\PYG{l+s}{\PYGZdq{}}\PYG{l+s}{Vietnam\PYGZdq{}}\PYG{p}{,}\PYG{+w}{ }\PYG{l+s}{\PYGZdq{}}\PYG{l+s}{Thailand\PYGZdq{}}\PYG{p}{,}\PYG{+w}{ }\PYG{l+s}{\PYGZdq{}}\PYG{l+s}{India\PYGZdq{}}\PYG{p}{,}\PYG{+w}{ }\PYG{l+s}{\PYGZdq{}}\PYG{l+s}{Switzerland\PYGZdq{}}\PYG{p}{,}\PYG{+w}{ }\PYG{l+s}{\PYGZdq{}}\PYG{l+s}{United States\PYGZdq{}}\PYG{p}{,}\PYG{+w}{ }\PYG{l+s}{\PYGZdq{}}\PYG{l+s}{Qatar\PYGZdq{}}\PYG{p}{)}
\end{sphinxVerbatim}

\end{sphinxuseclass}\end{sphinxVerbatimInput}

\end{sphinxuseclass}
\begin{sphinxuseclass}{cell}\begin{sphinxVerbatimInput}

\begin{sphinxuseclass}{cell_input}
\begin{sphinxVerbatim}[commandchars=\\\{\}]
\PYG{n}{countries}\PYG{+w}{  }\PYG{o}{\PYGZlt{}\PYGZhy{}}\PYG{+w}{ }\PYG{n}{lex\PYGZus{}2018}\PYG{p}{[}\PYG{n}{lex\PYGZus{}2018}\PYG{o}{\PYGZdl{}}\PYG{n}{Entity}\PYG{+w}{ }\PYG{o}{\PYGZpc{}in\PYGZpc{}}\PYG{+w}{ }\PYG{n}{sel}\PYG{p}{,}\PYG{+w}{ }\PYG{p}{]}\PYG{+w}{ }
\end{sphinxVerbatim}

\end{sphinxuseclass}\end{sphinxVerbatimInput}

\end{sphinxuseclass}
\begin{sphinxuseclass}{cell}\begin{sphinxVerbatimInput}

\begin{sphinxuseclass}{cell_input}
\begin{sphinxVerbatim}[commandchars=\\\{\}]
\PYG{n}{data}\PYG{+w}{  }\PYG{o}{\PYGZlt{}\PYGZhy{}}\PYG{+w}{ }\PYG{n}{countries}\PYG{p}{[}\PYG{+w}{ }\PYG{p}{,}\PYG{n+nf}{c}\PYG{p}{(}\PYG{l+s}{\PYGZdq{}}\PYG{l+s}{Entity\PYGZdq{}}\PYG{p}{,}\PYG{+w}{ }\PYG{l+s}{\PYGZdq{}}\PYG{l+s}{GDPC\PYGZdq{}}\PYG{p}{,}\PYG{+w}{ }\PYG{l+s}{\PYGZdq{}}\PYG{l+s}{LE\PYGZdq{}}\PYG{p}{)}\PYG{p}{]}
\end{sphinxVerbatim}

\end{sphinxuseclass}\end{sphinxVerbatimInput}

\end{sphinxuseclass}
\begin{sphinxuseclass}{cell}\begin{sphinxVerbatimInput}

\begin{sphinxuseclass}{cell_input}
\begin{sphinxVerbatim}[commandchars=\\\{\}]
\PYG{n}{data}
\end{sphinxVerbatim}

\end{sphinxuseclass}\end{sphinxVerbatimInput}
\begin{sphinxVerbatimOutput}

\begin{sphinxuseclass}{cell_output}\begin{equation*}
\begin{split}A data.frame: 9 × 3
\begin{tabular}{r|lll}
  & Entity & GDPC & LE\\
  & <chr> & <dbl> & <dbl>\\
\hline
	10008 & Chad          &   2046.363 & 52.8253\\
	15257 & Egypt         &  11957.212 & 71.3669\\
	24460 & India         &   6806.498 & 70.7095\\
	27940 & Kenya         &   3377.470 & 62.6765\\
	45731 & Qatar         & 153764.170 & 80.8982\\
	54635 & Switzerland   &  61372.730 & 83.5615\\
	55857 & Thailand      &  16648.623 & 78.6622\\
	59903 & United States &  55334.740 & 78.9896\\
	61747 & Vietnam       &   6814.142 & 73.9757\\
\end{tabular}\end{split}
\end{equation*}
\end{sphinxuseclass}\end{sphinxVerbatimOutput}

\end{sphinxuseclass}
\begin{sphinxuseclass}{cell}\begin{sphinxVerbatimInput}

\begin{sphinxuseclass}{cell_input}
\begin{sphinxVerbatim}[commandchars=\\\{\}]
\PYG{n+nf}{curve}\PYG{p}{(}\PYG{l+m}{67.78}\PYG{+w}{ }\PYG{o}{+}\PYG{+w}{ }\PYG{l+m}{0.0002549}\PYG{o}{*}\PYG{n}{x}\PYG{p}{,}\PYG{+w}{ }\PYG{n}{from}\PYG{+w}{ }\PYG{o}{=}\PYG{+w}{ }\PYG{l+m}{0}\PYG{p}{,}\PYG{+w}{ }\PYG{n}{to}\PYG{+w}{ }\PYG{o}{=}\PYG{+w}{ }\PYG{l+m}{160000}\PYG{p}{,}
\PYG{+w}{     }\PYG{n}{main}\PYG{+w}{ }\PYG{o}{=}\PYG{+w}{ }\PYG{l+s}{\PYGZdq{}}\PYG{l+s}{Relation of GDP per capita and life expectancy\PYGZdq{}}\PYG{p}{,}
\PYG{+w}{     }\PYG{n}{xlab}\PYG{+w}{ }\PYG{o}{=}\PYG{+w}{ }\PYG{l+s}{\PYGZdq{}}\PYG{l+s}{GDP per capita in international Dollars at 2011 prices\PYGZdq{}}\PYG{p}{,}
\PYG{+w}{     }\PYG{n}{ylab}\PYG{+w}{ }\PYG{o}{=}\PYG{+w}{ }\PYG{l+s}{\PYGZdq{}}\PYG{l+s}{Average life expectancy at birth in years\PYGZdq{}}\PYG{+w}{    }
\PYG{+w}{     }\PYG{p}{)}

\PYG{n+nf}{points}\PYG{p}{(}\PYG{n}{data}\PYG{o}{\PYGZdl{}}\PYG{n}{GDPC}\PYG{p}{,}\PYG{+w}{ }\PYG{n}{data}\PYG{o}{\PYGZdl{}}\PYG{n}{LE}\PYG{p}{,}\PYG{+w}{ }\PYG{n}{pch}\PYG{+w}{ }\PYG{o}{=}\PYG{+w}{ }\PYG{l+m}{19}\PYG{p}{,}\PYG{+w}{ }\PYG{n}{col}\PYG{+w}{ }\PYG{o}{=}\PYG{+w}{ }\PYG{l+s}{\PYGZdq{}}\PYG{l+s}{red\PYGZdq{}}\PYG{p}{)}
\PYG{n+nf}{text}\PYG{p}{(}\PYG{n}{x}\PYG{+w}{ }\PYG{o}{=}\PYG{+w}{ }\PYG{n}{data}\PYG{o}{\PYGZdl{}}\PYG{n}{GDPC}\PYG{+w}{ }\PYG{o}{+}\PYG{+w}{ }\PYG{l+m}{2}\PYG{p}{,}\PYG{+w}{ }\PYG{n}{y}\PYG{+w}{ }\PYG{o}{=}\PYG{+w}{ }\PYG{n}{data}\PYG{o}{\PYGZdl{}}\PYG{n}{LE}\PYG{+w}{ }\PYG{o}{+}\PYG{+w}{ }\PYG{l+m}{2}\PYG{p}{,}\PYG{+w}{ }\PYG{n}{label}\PYG{+w}{ }\PYG{o}{=}\PYG{+w}{ }\PYG{n}{data}\PYG{o}{\PYGZdl{}}\PYG{n}{Entity}\PYG{p}{)}
\end{sphinxVerbatim}

\end{sphinxuseclass}\end{sphinxVerbatimInput}
\begin{sphinxVerbatimOutput}

\begin{sphinxuseclass}{cell_output}
\noindent\sphinxincludegraphics{{cb5c4930aaf70c77de355a72abe13e498ae8eb4d89c731caad33708059865e40}.png}

\end{sphinxuseclass}\end{sphinxVerbatimOutput}

\end{sphinxuseclass}
\sphinxAtStartPar
Add my own country

\begin{sphinxuseclass}{cell}\begin{sphinxVerbatimInput}

\begin{sphinxuseclass}{cell_input}
\begin{sphinxVerbatim}[commandchars=\\\{\}]
\PYG{n+nf}{curve}\PYG{p}{(}\PYG{l+m}{67.78}\PYG{+w}{ }\PYG{o}{+}\PYG{+w}{ }\PYG{l+m}{0.0002549}\PYG{o}{*}\PYG{n}{x}\PYG{p}{,}\PYG{+w}{ }\PYG{n}{from}\PYG{+w}{ }\PYG{o}{=}\PYG{+w}{ }\PYG{l+m}{0}\PYG{p}{,}\PYG{+w}{ }\PYG{n}{to}\PYG{+w}{ }\PYG{o}{=}\PYG{+w}{ }\PYG{l+m}{160000}\PYG{p}{,}
\PYG{+w}{     }\PYG{n}{main}\PYG{+w}{ }\PYG{o}{=}\PYG{+w}{ }\PYG{l+s}{\PYGZdq{}}\PYG{l+s}{Relation of GDP per capita and life expectancy\PYGZdq{}}\PYG{p}{,}
\PYG{+w}{     }\PYG{n}{xlab}\PYG{+w}{ }\PYG{o}{=}\PYG{+w}{ }\PYG{l+s}{\PYGZdq{}}\PYG{l+s}{GDP per capita in international Dollars at 2011 prices\PYGZdq{}}\PYG{p}{,}
\PYG{+w}{     }\PYG{n}{ylab}\PYG{+w}{ }\PYG{o}{=}\PYG{+w}{ }\PYG{l+s}{\PYGZdq{}}\PYG{l+s}{Average life expectancy at birth in years\PYGZdq{}}\PYG{+w}{    }
\PYG{+w}{     }\PYG{p}{)}

\PYG{n+nf}{points}\PYG{p}{(}\PYG{n}{data}\PYG{o}{\PYGZdl{}}\PYG{n}{GDPC}\PYG{p}{,}\PYG{+w}{ }\PYG{n}{data}\PYG{o}{\PYGZdl{}}\PYG{n}{LE}\PYG{p}{,}\PYG{+w}{ }\PYG{n}{pch}\PYG{+w}{ }\PYG{o}{=}\PYG{+w}{ }\PYG{l+m}{19}\PYG{p}{,}\PYG{+w}{ }\PYG{n}{col}\PYG{+w}{ }\PYG{o}{=}\PYG{+w}{ }\PYG{l+s}{\PYGZdq{}}\PYG{l+s}{red\PYGZdq{}}\PYG{p}{)}
\PYG{n+nf}{points}\PYG{p}{(}\PYG{n}{lex\PYGZus{}2018}\PYG{o}{\PYGZdl{}}\PYG{n}{GDPC}\PYG{p}{[}\PYG{n}{lex\PYGZus{}2018}\PYG{o}{\PYGZdl{}}\PYG{n}{Entity}\PYG{+w}{ }\PYG{o}{==}\PYG{+w}{ }\PYG{l+s}{\PYGZdq{}}\PYG{l+s}{Austria\PYGZdq{}}\PYG{p}{]}\PYG{p}{,}\PYG{+w}{ }\PYG{n}{lex\PYGZus{}2018}\PYG{o}{\PYGZdl{}}\PYG{n}{LE}\PYG{p}{[}\PYG{n}{lex\PYGZus{}2018}\PYG{o}{\PYGZdl{}}\PYG{n}{Entity}\PYG{+w}{ }\PYG{o}{==}\PYG{+w}{ }\PYG{l+s}{\PYGZdq{}}\PYG{l+s}{Austria\PYGZdq{}}\PYG{p}{]}\PYG{p}{,}\PYG{+w}{ }\PYG{n}{pch}\PYG{+w}{ }\PYG{o}{=}\PYG{+w}{ }\PYG{l+m}{19}\PYG{p}{,}\PYG{+w}{ }\PYG{n}{col}\PYG{+w}{ }\PYG{o}{=}\PYG{+w}{ }\PYG{l+m}{4}\PYG{p}{)}
\PYG{n+nf}{text}\PYG{p}{(}\PYG{n}{x}\PYG{+w}{ }\PYG{o}{=}\PYG{+w}{ }\PYG{n}{data}\PYG{o}{\PYGZdl{}}\PYG{n}{GDPC}\PYG{+w}{ }\PYG{o}{+}\PYG{+w}{ }\PYG{l+m}{2}\PYG{p}{,}\PYG{+w}{ }\PYG{n}{y}\PYG{+w}{ }\PYG{o}{=}\PYG{+w}{ }\PYG{n}{data}\PYG{o}{\PYGZdl{}}\PYG{n}{LE}\PYG{+w}{ }\PYG{o}{+}\PYG{+w}{ }\PYG{l+m}{2}\PYG{p}{,}\PYG{+w}{ }\PYG{n}{label}\PYG{+w}{ }\PYG{o}{=}\PYG{+w}{ }\PYG{n}{data}\PYG{o}{\PYGZdl{}}\PYG{n}{Entity}\PYG{p}{)}
\PYG{n+nf}{text}\PYG{p}{(}\PYG{n}{x}\PYG{+w}{ }\PYG{o}{=}\PYG{+w}{ }\PYG{n}{lex\PYGZus{}2018}\PYG{o}{\PYGZdl{}}\PYG{n}{GDPC}\PYG{p}{[}\PYG{n}{lex\PYGZus{}2018}\PYG{o}{\PYGZdl{}}\PYG{n}{Entity}\PYG{+w}{ }\PYG{o}{==}\PYG{+w}{ }\PYG{l+s}{\PYGZdq{}}\PYG{l+s}{Austria\PYGZdq{}}\PYG{p}{]}\PYG{+w}{ }\PYG{o}{+}\PYG{+w}{ }\PYG{l+m}{2}\PYG{p}{,}\PYG{+w}{ }
\PYG{+w}{    }\PYG{n}{y}\PYG{+w}{ }\PYG{o}{=}\PYG{+w}{ }\PYG{n}{lex\PYGZus{}2018}\PYG{o}{\PYGZdl{}}\PYG{n}{LE}\PYG{p}{[}\PYG{n}{lex\PYGZus{}2018}\PYG{o}{\PYGZdl{}}\PYG{n}{Entity}\PYG{+w}{ }\PYG{o}{==}\PYG{+w}{ }\PYG{l+s}{\PYGZdq{}}\PYG{l+s}{Austria\PYGZdq{}}\PYG{p}{]}\PYG{+w}{ }\PYG{o}{+}\PYG{+w}{ }\PYG{l+m}{2}\PYG{p}{,}\PYG{+w}{ }\PYG{n}{label}\PYG{+w}{ }\PYG{o}{=}\PYG{+w}{ }\PYG{n}{lex\PYGZus{}2018}\PYG{o}{\PYGZdl{}}\PYG{n}{Entity}\PYG{p}{[}\PYG{n}{lex\PYGZus{}2018}\PYG{o}{\PYGZdl{}}\PYG{n}{Entity}\PYG{+w}{ }\PYG{o}{==}\PYG{+w}{ }\PYG{l+s}{\PYGZdq{}}\PYG{l+s}{Austria\PYGZdq{}}\PYG{p}{]}\PYG{p}{)}
\end{sphinxVerbatim}

\end{sphinxuseclass}\end{sphinxVerbatimInput}
\begin{sphinxVerbatimOutput}

\begin{sphinxuseclass}{cell_output}
\noindent\sphinxincludegraphics{{5b51264462984c06434c6e45d6c109fc7f7009407887b49bb0247957032cf0ff}.png}

\end{sphinxuseclass}\end{sphinxVerbatimOutput}

\end{sphinxuseclass}

\subsection{Exercise 2: Reproduce the scatterplot on primary school enrolment rate}
\label{\detokenize{exercises_unit_4:exercise-2-reproduce-the-scatterplot-on-primary-school-enrolment-rate}}
\begin{sphinxuseclass}{cell}\begin{sphinxVerbatimInput}

\begin{sphinxuseclass}{cell_input}
\begin{sphinxVerbatim}[commandchars=\\\{\}]
\PYG{n+nf}{library}\PYG{p}{(}\PYG{n}{JWL}\PYG{p}{)}
\PYG{n}{dat}\PYG{+w}{  }\PYG{o}{\PYGZlt{}\PYGZhy{}}\PYG{+w}{ }\PYG{n}{enrol\PYGZus{}attend\PYGZus{}dat}
\PYG{n+nf}{head}\PYG{p}{(}\PYG{n}{dat}\PYG{p}{)}
\end{sphinxVerbatim}

\end{sphinxuseclass}\end{sphinxVerbatimInput}
\begin{sphinxVerbatimOutput}

\begin{sphinxuseclass}{cell_output}\begin{equation*}
\begin{split}A data.frame: 6 × 6
\begin{tabular}{r|llllll}
  & Country & Code & Year & Attendance & Enrolment & Population\\
  & <chr> & <chr> & <int> & <dbl> & <dbl> & <dbl>\\
\hline
	597 & Albania    & ALB & 2008 & 94.47980 & 90.26729 &  2951690\\
	2436 & Armenia    & ARM & 2005 & 98.57435 & 80.08213 &  3047254\\
	3654 & Azerbaijan & AZE & 2006 & 79.27288 & 83.15377 &  8763353\\
	5726 & Benin      & BEN & 2006 & 63.06672 & 82.57930 &  8402635\\
	5727 & Benin      & BEN & 2012 & 70.60000 & 94.86310 & 10014087\\
	6332 & Bolivia    & BOL & 2003 & 82.40857 & 95.62232 &  9057386\\
\end{tabular}\end{split}
\end{equation*}
\end{sphinxuseclass}\end{sphinxVerbatimOutput}

\end{sphinxuseclass}
\begin{sphinxuseclass}{cell}\begin{sphinxVerbatimInput}

\begin{sphinxuseclass}{cell_input}
\begin{sphinxVerbatim}[commandchars=\\\{\}]
\PYG{n+nf}{library}\PYG{p}{(}\PYG{n}{JWL}\PYG{p}{)}
\PYG{n+nf}{plot}\PYG{p}{(}\PYG{n}{dat}\PYG{o}{\PYGZdl{}}\PYG{n}{Enrolment}\PYG{p}{,}\PYG{+w}{ }\PYG{n}{dat}\PYG{o}{\PYGZdl{}}\PYG{n}{Attendance}\PYG{p}{,}\PYG{+w}{ }
\PYG{+w}{     }\PYG{n}{main}\PYG{+w}{ }\PYG{o}{=}\PYG{+w}{ }\PYG{p}{(}\PYG{l+s}{\PYGZdq{}}\PYG{l+s}{Enrollment versus Attendance\PYGZdq{}}\PYG{p}{)}\PYG{p}{,}\PYG{+w}{ }
\PYG{+w}{     }\PYG{n}{xlab}\PYG{+w}{ }\PYG{o}{=}\PYG{+w}{ }\PYG{p}{(}\PYG{l+s}{\PYGZdq{}}\PYG{l+s}{Enrollment rate primary school\PYGZdq{}}\PYG{p}{)}\PYG{p}{,}\PYG{+w}{ }
\PYG{+w}{     }\PYG{n}{ylab}\PYG{+w}{ }\PYG{o}{=}\PYG{+w}{ }\PYG{p}{(}\PYG{l+s}{\PYGZdq{}}\PYG{l+s}{Attendance rate primary school\PYGZdq{}}\PYG{p}{)}\PYG{p}{,}\PYG{+w}{ }\PYG{n}{pch}\PYG{+w}{ }\PYG{o}{=}\PYG{+w}{ }\PYG{l+m}{16}\PYG{p}{)}

\PYG{n+nf}{abline}\PYG{p}{(}\PYG{l+m}{1}\PYG{p}{,}\PYG{l+m}{1}\PYG{p}{,}\PYG{+w}{ }\PYG{n}{col}\PYG{+w}{ }\PYG{o}{=}\PYG{+w}{ }\PYG{l+s}{\PYGZdq{}}\PYG{l+s}{blue\PYGZdq{}}\PYG{p}{)}
\end{sphinxVerbatim}

\end{sphinxuseclass}\end{sphinxVerbatimInput}
\begin{sphinxVerbatimOutput}

\begin{sphinxuseclass}{cell_output}
\noindent\sphinxincludegraphics{{9f8fb8b2e9e00b208ce2af4ebf39dee467743ccdd1642078ba21a537516ad1a4}.png}

\end{sphinxuseclass}\end{sphinxVerbatimOutput}

\end{sphinxuseclass}

\subsection{Exercise 3: Write a conversion function from any data to standard units}
\label{\detokenize{exercises_unit_4:exercise-3-write-a-conversion-function-from-any-data-to-standard-units}}
\begin{sphinxuseclass}{cell}\begin{sphinxVerbatimInput}

\begin{sphinxuseclass}{cell_input}
\begin{sphinxVerbatim}[commandchars=\\\{\}]
\PYG{n+nf}{library}\PYG{p}{(}\PYG{n}{JWL}\PYG{p}{)}

\PYG{n}{testdat}\PYG{+w}{  }\PYG{o}{\PYGZlt{}\PYGZhy{}}\PYG{+w}{ }\PYG{n}{pearson}
\end{sphinxVerbatim}

\end{sphinxuseclass}\end{sphinxVerbatimInput}

\end{sphinxuseclass}
\begin{sphinxuseclass}{cell}\begin{sphinxVerbatimInput}

\begin{sphinxuseclass}{cell_input}
\begin{sphinxVerbatim}[commandchars=\\\{\}]
\PYG{n+nf}{head}\PYG{p}{(}\PYG{n}{testdat}\PYG{p}{)}
\end{sphinxVerbatim}

\end{sphinxuseclass}\end{sphinxVerbatimInput}
\begin{sphinxVerbatimOutput}

\begin{sphinxuseclass}{cell_output}\begin{equation*}
\begin{split}A spec\_tbl\_df: 6 × 2
\begin{tabular}{r|ll}
  & Father & Son\\
  & <dbl> & <dbl>\\
\hline
	1 & 65.0 & 59.8\\
	2 & 63.3 & 63.2\\
	3 & 65.0 & 63.3\\
	4 & 65.8 & 62.8\\
	5 & 61.1 & 64.3\\
	6 & 63.0 & 64.2\\
\end{tabular}\end{split}
\end{equation*}
\end{sphinxuseclass}\end{sphinxVerbatimOutput}

\end{sphinxuseclass}
\begin{sphinxuseclass}{cell}\begin{sphinxVerbatimInput}

\begin{sphinxuseclass}{cell_input}
\begin{sphinxVerbatim}[commandchars=\\\{\}]
\PYG{n}{test}\PYG{+w}{ }\PYG{o}{\PYGZlt{}\PYGZhy{}}\PYG{+w}{ }\PYG{n+nf}{as.data.frame}\PYG{p}{(}\PYG{n+nf}{scale}\PYG{p}{(}\PYG{n}{testdat}\PYG{p}{)}\PYG{p}{)}

\PYG{n+nf}{head}\PYG{p}{(}\PYG{n}{test}\PYG{p}{)}
\end{sphinxVerbatim}

\end{sphinxuseclass}\end{sphinxVerbatimInput}
\begin{sphinxVerbatimOutput}

\begin{sphinxuseclass}{cell_output}\begin{equation*}
\begin{split}A data.frame: 6 × 2
\begin{tabular}{r|ll}
  & Father & Son\\
  & <dbl> & <dbl>\\
\hline
	1 & -0.9785130 & -3.154694\\
	2 & -1.5976343 & -1.947391\\
	3 & -0.9785130 & -1.911882\\
	4 & -0.6871618 & -2.089426\\
	5 & -2.3988501 & -1.556793\\
	6 & -1.7068910 & -1.592302\\
\end{tabular}\end{split}
\end{equation*}
\end{sphinxuseclass}\end{sphinxVerbatimOutput}

\end{sphinxuseclass}
\begin{sphinxuseclass}{cell}\begin{sphinxVerbatimInput}

\begin{sphinxuseclass}{cell_input}
\begin{sphinxVerbatim}[commandchars=\\\{\}]
\PYG{n+nf}{mean}\PYG{p}{(}\PYG{n}{test}\PYG{o}{\PYGZdl{}}\PYG{n}{Father}\PYG{p}{)}
\PYG{n+nf}{mean}\PYG{p}{(}\PYG{n}{test}\PYG{o}{\PYGZdl{}}\PYG{n}{Son}\PYG{p}{)}

\PYG{n+nf}{sd}\PYG{p}{(}\PYG{n}{test}\PYG{o}{\PYGZdl{}}\PYG{n}{Father}\PYG{p}{)}
\PYG{n+nf}{sd}\PYG{p}{(}\PYG{n}{test}\PYG{o}{\PYGZdl{}}\PYG{n}{Son}\PYG{p}{)}
\end{sphinxVerbatim}

\end{sphinxuseclass}\end{sphinxVerbatimInput}
\begin{sphinxVerbatimOutput}

\begin{sphinxuseclass}{cell_output}\begin{equation*}
\begin{split}-1.31100560022995e-15\end{split}
\end{equation*}\begin{equation*}
\begin{split}-1.33732395422732e-15\end{split}
\end{equation*}\begin{equation*}
\begin{split}1\end{split}
\end{equation*}\begin{equation*}
\begin{split}1\end{split}
\end{equation*}
\end{sphinxuseclass}\end{sphinxVerbatimOutput}

\end{sphinxuseclass}
\begin{sphinxuseclass}{cell}\begin{sphinxVerbatimInput}

\begin{sphinxuseclass}{cell_input}
\begin{sphinxVerbatim}[commandchars=\\\{\}]
\PYG{c+c1}{\PYGZsh{} An implementation using the scale function of R.}

\PYG{n}{convert\PYGZus{}su}\PYG{+w}{  }\PYG{o}{\PYGZlt{}\PYGZhy{}}\PYG{+w}{ }\PYG{n+nf}{function}\PYG{p}{(}\PYG{n}{df}\PYG{p}{)}\PYG{p}{\PYGZob{}}

\PYG{+w}{    }\PYG{n+nf}{as.data.frame}\PYG{p}{(}\PYG{n+nf}{scale}\PYG{p}{(}\PYG{n}{df}\PYG{p}{)}\PYG{p}{)}
\PYG{+w}{    }
\PYG{p}{\PYGZcb{}}
\end{sphinxVerbatim}

\end{sphinxuseclass}\end{sphinxVerbatimInput}

\end{sphinxuseclass}
\begin{sphinxuseclass}{cell}\begin{sphinxVerbatimInput}

\begin{sphinxuseclass}{cell_input}
\begin{sphinxVerbatim}[commandchars=\\\{\}]
\PYG{n+nf}{convert\PYGZus{}su}\PYG{p}{(}\PYG{n}{testdat}\PYG{p}{)}
\end{sphinxVerbatim}

\end{sphinxuseclass}\end{sphinxVerbatimInput}
\begin{sphinxVerbatimOutput}

\begin{sphinxuseclass}{cell_output}\begin{equation*}
\begin{split}A data.frame: 1078 × 2
\begin{tabular}{ll}
 Father & Son\\
 <dbl> & <dbl>\\
\hline
	 -0.97851299 & -3.1546939\\
	 -1.59763428 & -1.9473907\\
	 -0.97851299 & -1.9118818\\
	 -0.68716179 & -2.0894264\\
	 -2.39885006 & -1.5567926\\
	 -1.70689098 & -1.5923015\\
	 -0.83283739 & -1.6278104\\
	 -1.08776969 & -1.6633194\\
	 -0.57790509 & -1.4502659\\
	 -0.25013500 & -1.6633194\\
	 -3.16364695 & -1.2372124\\
	 -1.74330987 & -1.1661945\\
	 -1.45195868 & -1.0596678\\
	 -1.30628308 & -1.1661945\\
	 -1.08776969 & -1.2017034\\
	 -0.90567519 & -1.3792480\\
	 -0.46864840 & -1.3082302\\
	 -0.75999959 & -1.1306856\\
	 -0.10445940 & -1.2727213\\
	 -0.32297280 & -1.1306856\\
	  0.04121619 & -1.2727213\\
	  0.66033748 & -1.1306856\\
	 -1.88898547 & -0.7400875\\
	 -1.41553978 & -0.8111054\\
	 -1.16060748 & -0.9176321\\
	 -0.97851299 & -0.9531410\\
	 -1.08776969 & -0.9531410\\
	 -0.72358069 & -0.7755964\\
	 -0.79641849 & -1.0596678\\
	 -0.75999959 & -0.9531410\\
	 ⋮ & ⋮\\
	  1.097364275 &  1.2129029\\
	  1.716485564 &  1.2839207\\
	  1.716485564 &  1.2129029\\
	 -0.177297202 &  1.4614653\\
	  0.041216194 &  1.4614653\\
	  0.441824087 &  1.7100277\\
	  0.988107577 &  1.7810456\\
	  1.279458772 &  1.4614653\\
	  1.352296571 &  1.6745188\\
	  1.461553269 &  1.5324832\\
	  1.789323363 &  1.6035010\\
	 -0.031621605 &  2.0651169\\
	  0.332567389 &  1.8165545\\
	  1.206620973 &  1.8165545\\
	  1.570809967 &  1.8165545\\
	  2.736214746 &  1.8165545\\
	  1.971417860 &  2.4557150\\
	  0.806013081 &  3.0238577\\
	 -0.796418491 & -3.0481671\\
	  1.789323363 &  2.8818220\\
	  1.643647766 & -0.7045786\\
	 -1.634053176 & -3.5097830\\
	  2.044255659 & -0.2784716\\
	 -0.687161793 & -2.7285869\\
	  0.004797295 & -3.1546939\\
	 -0.250135001 &  0.7512870\\
	  1.315877672 & -0.1364359\\
	  1.497972168 &  0.2186532\\
	  1.097364275 &  0.2186532\\
	  0.951688678 & -0.5980519\\
\end{tabular}\end{split}
\end{equation*}
\end{sphinxuseclass}\end{sphinxVerbatimOutput}

\end{sphinxuseclass}
\begin{sphinxuseclass}{cell}\begin{sphinxVerbatimInput}

\begin{sphinxuseclass}{cell_input}
\begin{sphinxVerbatim}[commandchars=\\\{\}]
\PYG{n}{calculate\PYGZus{}correlation}\PYG{+w}{  }\PYG{o}{\PYGZlt{}\PYGZhy{}}\PYG{+w}{ }\PYG{n+nf}{function}\PYG{p}{(}\PYG{n}{dat}\PYG{p}{)}\PYG{p}{\PYGZob{}}
\PYG{+w}{    }
\PYG{+w}{    }\PYG{c+c1}{\PYGZsh{}convert to standard units}
\PYG{+w}{    }
\PYG{+w}{    }\PYG{n}{su}\PYG{+w}{  }\PYG{o}{\PYGZlt{}\PYGZhy{}}\PYG{+w}{ }\PYG{n+nf}{convert\PYGZus{}su}\PYG{p}{(}\PYG{n}{dat}\PYG{p}{)}
\PYG{+w}{    }
\PYG{+w}{    }\PYG{n+nf}{mean}\PYG{p}{(}\PYG{n}{su}\PYG{p}{[}\PYG{+w}{ }\PYG{p}{,}\PYG{l+m}{1}\PYG{p}{]}\PYG{o}{*}\PYG{n}{su}\PYG{p}{[}\PYG{p}{,}\PYG{+w}{ }\PYG{l+m}{2}\PYG{p}{]}\PYG{p}{)}
\PYG{+w}{    }
\PYG{p}{\PYGZcb{}}
\end{sphinxVerbatim}

\end{sphinxuseclass}\end{sphinxVerbatimInput}

\end{sphinxuseclass}
\begin{sphinxuseclass}{cell}\begin{sphinxVerbatimInput}

\begin{sphinxuseclass}{cell_input}
\begin{sphinxVerbatim}[commandchars=\\\{\}]
\PYG{n+nf}{calculate\PYGZus{}correlation}\PYG{p}{(}\PYG{n}{testdat}\PYG{p}{)}
\end{sphinxVerbatim}

\end{sphinxuseclass}\end{sphinxVerbatimInput}
\begin{sphinxVerbatimOutput}

\begin{sphinxuseclass}{cell_output}\begin{equation*}
\begin{split}0.500697780361573\end{split}
\end{equation*}
\end{sphinxuseclass}\end{sphinxVerbatimOutput}

\end{sphinxuseclass}
\begin{sphinxuseclass}{cell}\begin{sphinxVerbatimInput}

\begin{sphinxuseclass}{cell_input}
\begin{sphinxVerbatim}[commandchars=\\\{\}]
\PYG{c+c1}{\PYGZsh{} Do the same calculation with these data using R cor() function}

\PYG{n+nf}{cor}\PYG{p}{(}\PYG{n}{testdat}\PYG{o}{\PYGZdl{}}\PYG{n}{Father}\PYG{p}{,}\PYG{+w}{ }\PYG{n}{testdat}\PYG{o}{\PYGZdl{}}\PYG{n}{Son}\PYG{p}{)}
\end{sphinxVerbatim}

\end{sphinxuseclass}\end{sphinxVerbatimInput}
\begin{sphinxVerbatimOutput}

\begin{sphinxuseclass}{cell_output}\begin{equation*}
\begin{split}0.501162680807591\end{split}
\end{equation*}
\end{sphinxuseclass}\end{sphinxVerbatimOutput}

\end{sphinxuseclass}
\sphinxAtStartPar
This works, the differences after the third decimal point come from numerical differences.


\subsection{Advertising channels and sales}
\label{\detokenize{exercises_unit_4:advertising-channels-and-sales}}
\begin{sphinxuseclass}{cell}\begin{sphinxVerbatimInput}

\begin{sphinxuseclass}{cell_input}
\begin{sphinxVerbatim}[commandchars=\\\{\}]
\PYG{n+nf}{library}\PYG{p}{(}\PYG{n}{JWL}\PYG{p}{)}

\PYG{n}{data}\PYG{+w}{  }\PYG{o}{\PYGZlt{}\PYGZhy{}}\PYG{+w}{ }\PYG{n}{advertising}
\end{sphinxVerbatim}

\end{sphinxuseclass}\end{sphinxVerbatimInput}

\end{sphinxuseclass}
\begin{sphinxuseclass}{cell}\begin{sphinxVerbatimInput}

\begin{sphinxuseclass}{cell_input}
\begin{sphinxVerbatim}[commandchars=\\\{\}]
\PYG{n+nf}{head}\PYG{p}{(}\PYG{n}{data}\PYG{p}{)}
\end{sphinxVerbatim}

\end{sphinxuseclass}\end{sphinxVerbatimInput}
\begin{sphinxVerbatimOutput}

\begin{sphinxuseclass}{cell_output}\begin{equation*}
\begin{split}A data.frame: 6 × 5
\begin{tabular}{r|lllll}
  & ID & TV & radio & newspaper & sales\\
  & <int> & <dbl> & <dbl> & <dbl> & <dbl>\\
\hline
	1 & 1 & 230.1 & 37.8 & 69.2 & 22.1\\
	2 & 2 &  44.5 & 39.3 & 45.1 & 10.4\\
	3 & 3 &  17.2 & 45.9 & 69.3 &  9.3\\
	4 & 4 & 151.5 & 41.3 & 58.5 & 18.5\\
	5 & 5 & 180.8 & 10.8 & 58.4 & 12.9\\
	6 & 6 &   8.7 & 48.9 & 75.0 &  7.2\\
\end{tabular}\end{split}
\end{equation*}
\end{sphinxuseclass}\end{sphinxVerbatimOutput}

\end{sphinxuseclass}
\begin{sphinxuseclass}{cell}\begin{sphinxVerbatimInput}

\begin{sphinxuseclass}{cell_input}
\begin{sphinxVerbatim}[commandchars=\\\{\}]
\PYG{n+nf}{plot}\PYG{p}{(}\PYG{n}{data}\PYG{o}{\PYGZdl{}}\PYG{n}{TV}\PYG{p}{,}\PYG{+w}{ }\PYG{n}{data}\PYG{o}{\PYGZdl{}}\PYG{n}{sales}\PYG{p}{,}\PYG{+w}{ }\PYG{n}{col}\PYG{+w}{ }\PYG{o}{=}\PYG{+w}{ }\PYG{l+s}{\PYGZdq{}}\PYG{l+s}{red\PYGZdq{}}\PYG{p}{)}
\PYG{n+nf}{plot}\PYG{p}{(}\PYG{n}{data}\PYG{o}{\PYGZdl{}}\PYG{n}{radio}\PYG{p}{,}\PYG{+w}{ }\PYG{n}{data}\PYG{o}{\PYGZdl{}}\PYG{n}{sales}\PYG{p}{,}\PYG{+w}{ }\PYG{n}{col}\PYG{+w}{ }\PYG{o}{=}\PYG{+w}{ }\PYG{l+s}{\PYGZdq{}}\PYG{l+s}{blue\PYGZdq{}}\PYG{p}{)}
\PYG{n+nf}{plot}\PYG{p}{(}\PYG{n}{data}\PYG{o}{\PYGZdl{}}\PYG{n}{newspaper}\PYG{p}{,}\PYG{+w}{ }\PYG{n}{data}\PYG{o}{\PYGZdl{}}\PYG{n}{sales}\PYG{p}{,}\PYG{+w}{ }\PYG{n}{col}\PYG{+w}{ }\PYG{o}{=}\PYG{+w}{ }\PYG{l+s}{\PYGZdq{}}\PYG{l+s}{green\PYGZdq{}}\PYG{p}{)}
\end{sphinxVerbatim}

\end{sphinxuseclass}\end{sphinxVerbatimInput}
\begin{sphinxVerbatimOutput}

\begin{sphinxuseclass}{cell_output}
\noindent\sphinxincludegraphics{{65e4dbb4c615077ee11868817bc015d872d5521a3cd34630a15202ec62729add}.png}

\noindent\sphinxincludegraphics{{eecb87c924e5a4336ed53cfcfc13ee6b83dcd18aaa6c5d5fd15277f77484f305}.png}

\noindent\sphinxincludegraphics{{4609de6140ef80e3cbec337d8f30b78eaa8877efcb73c803b475479fee19d65a}.png}

\end{sphinxuseclass}\end{sphinxVerbatimOutput}

\end{sphinxuseclass}
\sphinxAtStartPar
The strongest correlation with the advertising channel is TV, followed by radio, followed by newspaper, which has almost no or only very weak correlation.

\begin{sphinxuseclass}{cell}\begin{sphinxVerbatimInput}

\begin{sphinxuseclass}{cell_input}
\begin{sphinxVerbatim}[commandchars=\\\{\}]
\PYG{n+nf}{cor}\PYG{p}{(}\PYG{n}{data}\PYG{o}{\PYGZdl{}}\PYG{n}{TV}\PYG{p}{,}\PYG{+w}{ }\PYG{n}{data}\PYG{o}{\PYGZdl{}}\PYG{n}{sales}\PYG{p}{)}
\PYG{n+nf}{cor}\PYG{p}{(}\PYG{n}{data}\PYG{o}{\PYGZdl{}}\PYG{n}{radio}\PYG{p}{,}\PYG{+w}{ }\PYG{n}{data}\PYG{o}{\PYGZdl{}}\PYG{n}{sales}\PYG{p}{)}
\PYG{n+nf}{cor}\PYG{p}{(}\PYG{n}{data}\PYG{o}{\PYGZdl{}}\PYG{n}{newspaper}\PYG{p}{,}\PYG{+w}{ }\PYG{n}{data}\PYG{o}{\PYGZdl{}}\PYG{n}{sales}\PYG{p}{)}
\end{sphinxVerbatim}

\end{sphinxuseclass}\end{sphinxVerbatimInput}
\begin{sphinxVerbatimOutput}

\begin{sphinxuseclass}{cell_output}\begin{equation*}
\begin{split}0.782224424861606\end{split}
\end{equation*}\begin{equation*}
\begin{split}0.576222574571055\end{split}
\end{equation*}\begin{equation*}
\begin{split}0.228299026376165\end{split}
\end{equation*}
\end{sphinxuseclass}\end{sphinxVerbatimOutput}

\end{sphinxuseclass}
\begin{sphinxuseclass}{cell}\begin{sphinxVerbatimInput}

\begin{sphinxuseclass}{cell_input}
\begin{sphinxVerbatim}[commandchars=\\\{\}]
\PYG{n}{mod}\PYG{+w}{  }\PYG{o}{\PYGZlt{}\PYGZhy{}}\PYG{+w}{ }\PYG{n+nf}{lm}\PYG{p}{(}\PYG{n}{sales}\PYG{+w}{ }\PYG{o}{\PYGZti{}}\PYG{+w}{ }\PYG{n}{TV}\PYG{p}{,}\PYG{+w}{ }\PYG{n}{data}\PYG{+w}{ }\PYG{o}{=}\PYG{+w}{ }\PYG{n}{data}\PYG{p}{)}
\PYG{n}{mod}
\end{sphinxVerbatim}

\end{sphinxuseclass}\end{sphinxVerbatimInput}
\begin{sphinxVerbatimOutput}

\begin{sphinxuseclass}{cell_output}
\begin{sphinxVerbatim}[commandchars=\\\{\}]
Call:
lm(formula = sales \PYGZti{} TV, data = data)

Coefficients:
(Intercept)           TV  
    7.03259      0.04754  
\end{sphinxVerbatim}

\end{sphinxuseclass}\end{sphinxVerbatimOutput}

\end{sphinxuseclass}
\begin{sphinxuseclass}{cell}\begin{sphinxVerbatimInput}

\begin{sphinxuseclass}{cell_input}
\begin{sphinxVerbatim}[commandchars=\\\{\}]
\PYG{n+nf}{plot}\PYG{p}{(}\PYG{n}{data}\PYG{o}{\PYGZdl{}}\PYG{n}{TV}\PYG{p}{,}\PYG{+w}{ }\PYG{n}{data}\PYG{o}{\PYGZdl{}}\PYG{n}{sales}\PYG{p}{,}\PYG{+w}{ }\PYG{n}{col}\PYG{+w}{ }\PYG{o}{=}\PYG{+w}{ }\PYG{l+s}{\PYGZdq{}}\PYG{l+s}{red\PYGZdq{}}\PYG{p}{)}
\PYG{n+nf}{abline}\PYG{p}{(}\PYG{n}{mod}\PYG{p}{,}\PYG{+w}{ }\PYG{n}{col}\PYG{+w}{ }\PYG{o}{=}\PYG{+w}{ }\PYG{l+s}{\PYGZdq{}}\PYG{l+s}{blue\PYGZdq{}}\PYG{p}{)}
\end{sphinxVerbatim}

\end{sphinxuseclass}\end{sphinxVerbatimInput}
\begin{sphinxVerbatimOutput}

\begin{sphinxuseclass}{cell_output}
\noindent\sphinxincludegraphics{{632f7e164c5554de72330d62a8a8aa6daec25df16370198f8214cbdf0c066739}.png}

\end{sphinxuseclass}\end{sphinxVerbatimOutput}

\end{sphinxuseclass}
\sphinxAtStartPar
Intercept: 7.03259
Slope: 0.04754

\begin{sphinxuseclass}{cell}\begin{sphinxVerbatimInput}

\begin{sphinxuseclass}{cell_input}
\begin{sphinxVerbatim}[commandchars=\\\{\}]
\PYG{n}{newdata}\PYG{+w}{  }\PYG{o}{\PYGZlt{}\PYGZhy{}}\PYG{+w}{ }\PYG{n+nf}{data.frame}\PYG{p}{(}\PYG{n}{TV}\PYG{+w}{ }\PYG{o}{=}\PYG{+w}{ }\PYG{n+nf}{c}\PYG{p}{(}\PYG{l+m}{0}\PYG{p}{)}\PYG{p}{)}

\PYG{n+nf}{predict}\PYG{p}{(}\PYG{n}{mod}\PYG{p}{,}\PYG{+w}{ }\PYG{n}{newdata}\PYG{p}{)}
\end{sphinxVerbatim}

\end{sphinxuseclass}\end{sphinxVerbatimInput}
\begin{sphinxVerbatimOutput}

\begin{sphinxuseclass}{cell_output}\begin{equation*}
\begin{split}\textbf{1:} 7.03259354912769\end{split}
\end{equation*}
\end{sphinxuseclass}\end{sphinxVerbatimOutput}

\end{sphinxuseclass}
\begin{sphinxuseclass}{cell}\begin{sphinxVerbatimInput}

\begin{sphinxuseclass}{cell_input}
\begin{sphinxVerbatim}[commandchars=\\\{\}]
\PYG{n}{newdata}\PYG{+w}{  }\PYG{o}{\PYGZlt{}\PYGZhy{}}\PYG{+w}{ }\PYG{n+nf}{data.frame}\PYG{p}{(}\PYG{n}{TV}\PYG{+w}{ }\PYG{o}{=}\PYG{+w}{ }\PYG{n+nf}{c}\PYG{p}{(}\PYG{l+m}{300}\PYG{p}{)}\PYG{p}{)}

\PYG{n+nf}{predict}\PYG{p}{(}\PYG{n}{mod}\PYG{p}{,}\PYG{+w}{ }\PYG{n}{newdata}\PYG{p}{)}
\end{sphinxVerbatim}

\end{sphinxuseclass}\end{sphinxVerbatimInput}
\begin{sphinxVerbatimOutput}

\begin{sphinxuseclass}{cell_output}\begin{equation*}
\begin{split}\textbf{1:} 21.2935856790336\end{split}
\end{equation*}
\end{sphinxuseclass}\end{sphinxVerbatimOutput}

\end{sphinxuseclass}

\section{Project people count: Looking Forward: Predicting the future population of Kenya}
\label{\detokenize{exercises_unit_4:project-people-count-looking-forward-predicting-the-future-population-of-kenya}}
\begin{sphinxuseclass}{cell}\begin{sphinxVerbatimInput}

\begin{sphinxuseclass}{cell_input}
\begin{sphinxVerbatim}[commandchars=\\\{\}]
\PYG{n+nf}{library}\PYG{p}{(}\PYG{n}{JWL}\PYG{p}{)}
\PYG{n}{dat}\PYG{+w}{  }\PYG{o}{\PYGZlt{}\PYGZhy{}}\PYG{+w}{ }\PYG{n}{population\PYGZus{}statistics\PYGZus{}by\PYGZus{}age\PYGZus{}and\PYGZus{}sex}
\end{sphinxVerbatim}

\end{sphinxuseclass}\end{sphinxVerbatimInput}

\end{sphinxuseclass}
\begin{sphinxuseclass}{cell}\begin{sphinxVerbatimInput}

\begin{sphinxuseclass}{cell_input}
\begin{sphinxVerbatim}[commandchars=\\\{\}]
\PYG{n}{kdat}\PYG{+w}{  }\PYG{o}{\PYGZlt{}\PYGZhy{}}\PYG{+w}{ }\PYG{n}{dat}\PYG{p}{[}\PYG{n}{dat}\PYG{o}{\PYGZdl{}}\PYG{n}{Year}\PYG{+w}{ }\PYG{o}{\PYGZlt{}=}\PYG{+w}{ }\PYG{l+m}{2023}\PYG{+w}{ }\PYG{o}{\PYGZam{}}\PYG{+w}{ }\PYG{n}{dat}\PYG{o}{\PYGZdl{}}\PYG{n}{ISO2}\PYG{+w}{ }\PYG{o}{==}\PYG{+w}{ }\PYG{l+s}{\PYGZdq{}}\PYG{l+s}{KE\PYGZdq{}}\PYG{p}{,}\PYG{+w}{ }\PYG{p}{]}
\PYG{n+nf}{head}\PYG{p}{(}\PYG{n}{kdat}\PYG{p}{)}
\end{sphinxVerbatim}

\end{sphinxuseclass}\end{sphinxVerbatimInput}
\begin{sphinxVerbatimOutput}

\begin{sphinxuseclass}{cell_output}\begin{equation*}
\begin{split}A tibble: 6 × 6
\begin{tabular}{r|llllll}
  & Country & ISO2 & Year & Sex & Age & POP\\
  & <fct> & <fct> & <dbl> & <fct> & <fct> & <dbl>\\
\hline
	610041 & Kenya & KE & 1950 & F & 0-4   & NA\\
	610042 & Kenya & KE & 1950 & F & 5-9   & NA\\
	610043 & Kenya & KE & 1950 & F & 10-14 & NA\\
	610044 & Kenya & KE & 1950 & F & 15-19 & NA\\
	610045 & Kenya & KE & 1950 & F & 20-24 & NA\\
	610046 & Kenya & KE & 1950 & F & 25-29 & NA\\
\end{tabular}\end{split}
\end{equation*}
\end{sphinxuseclass}\end{sphinxVerbatimOutput}

\end{sphinxuseclass}
\begin{sphinxuseclass}{cell}\begin{sphinxVerbatimInput}

\begin{sphinxuseclass}{cell_input}
\begin{sphinxVerbatim}[commandchars=\\\{\}]
\PYG{n}{kdat\PYGZus{}new}\PYG{+w}{  }\PYG{o}{\PYGZlt{}\PYGZhy{}}\PYG{+w}{ }\PYG{n+nf}{na.omit}\PYG{p}{(}\PYG{n}{kdat}\PYG{p}{)}
\PYG{n+nf}{head}\PYG{p}{(}\PYG{n}{kdat\PYGZus{}new}\PYG{p}{)}
\end{sphinxVerbatim}

\end{sphinxuseclass}\end{sphinxVerbatimInput}
\begin{sphinxVerbatimOutput}

\begin{sphinxuseclass}{cell_output}\begin{equation*}
\begin{split}A tibble: 6 × 6
\begin{tabular}{r|llllll}
  & Country & ISO2 & Year & Sex & Age & POP\\
  & <fct> & <fct> & <dbl> & <fct> & <fct> & <dbl>\\
\hline
	611201 & Kenya & KE & 1979 & F & 0-4   & 1618057\\
	611202 & Kenya & KE & 1979 & F & 5-9   & 1239085\\
	611203 & Kenya & KE & 1979 & F & 10-14 & 1045040\\
	611204 & Kenya & KE & 1979 & F & 15-19 &  858494\\
	611205 & Kenya & KE & 1979 & F & 20-24 &  680045\\
	611206 & Kenya & KE & 1979 & F & 25-29 &  542065\\
\end{tabular}\end{split}
\end{equation*}
\end{sphinxuseclass}\end{sphinxVerbatimOutput}

\end{sphinxuseclass}
\begin{sphinxuseclass}{cell}\begin{sphinxVerbatimInput}

\begin{sphinxuseclass}{cell_input}
\begin{sphinxVerbatim}[commandchars=\\\{\}]
\PYG{n}{agg}\PYG{+w}{  }\PYG{o}{\PYGZlt{}\PYGZhy{}}\PYG{+w}{ }\PYG{n+nf}{tapply}\PYG{p}{(}\PYG{n}{kdat\PYGZus{}new}\PYG{o}{\PYGZdl{}}\PYG{n}{POP}\PYG{p}{,}\PYG{+w}{ }\PYG{n}{kdat\PYGZus{}new}\PYG{o}{\PYGZdl{}}\PYG{n}{Year}\PYG{p}{,}\PYG{+w}{ }\PYG{n}{sum}\PYG{p}{)}
\PYG{n+nf}{head}\PYG{p}{(}\PYG{n}{agg}\PYG{p}{)}
\end{sphinxVerbatim}

\end{sphinxuseclass}\end{sphinxVerbatimInput}
\begin{sphinxVerbatimOutput}

\begin{sphinxuseclass}{cell_output}\begin{equation*}
\begin{split}\begin{description*}
\item[1979] 15690220
\item[1980] 16330790
\item[1981] 16986998
\item[1982] 17658853
\item[1983] 18345591
\item[1984] 19046144
\end{description*}\end{split}
\end{equation*}
\end{sphinxuseclass}\end{sphinxVerbatimOutput}

\end{sphinxuseclass}
\begin{sphinxuseclass}{cell}\begin{sphinxVerbatimInput}

\begin{sphinxuseclass}{cell_input}
\begin{sphinxVerbatim}[commandchars=\\\{\}]
\PYG{n}{pop\PYGZus{}ke}\PYG{+w}{ }\PYG{o}{\PYGZlt{}\PYGZhy{}}\PYG{+w}{ }\PYG{n+nf}{data.frame}\PYG{p}{(}\PYG{n}{Year}\PYG{+w}{ }\PYG{o}{=}\PYG{+w}{ }\PYG{n+nf}{names}\PYG{p}{(}\PYG{n}{agg}\PYG{p}{)}\PYG{p}{,}\PYG{+w}{ }\PYG{n}{POP}\PYG{+w}{ }\PYG{o}{=}\PYG{+w}{ }\PYG{n}{agg}\PYG{p}{)}
\PYG{n+nf}{head}\PYG{p}{(}\PYG{n}{pop\PYGZus{}ke}\PYG{p}{)}
\end{sphinxVerbatim}

\end{sphinxuseclass}\end{sphinxVerbatimInput}
\begin{sphinxVerbatimOutput}

\begin{sphinxuseclass}{cell_output}\begin{equation*}
\begin{split}A data.frame: 6 × 2
\begin{tabular}{r|ll}
  & Year & POP\\
  & <chr> & <dbl>\\
\hline
	1979 & 1979 & 15690220\\
	1980 & 1980 & 16330790\\
	1981 & 1981 & 16986998\\
	1982 & 1982 & 17658853\\
	1983 & 1983 & 18345591\\
	1984 & 1984 & 19046144\\
\end{tabular}\end{split}
\end{equation*}
\end{sphinxuseclass}\end{sphinxVerbatimOutput}

\end{sphinxuseclass}
\begin{sphinxuseclass}{cell}\begin{sphinxVerbatimInput}

\begin{sphinxuseclass}{cell_input}
\begin{sphinxVerbatim}[commandchars=\\\{\}]
\PYG{n+nf}{plot}\PYG{p}{(}\PYG{n}{pop\PYGZus{}ke}\PYG{o}{\PYGZdl{}}\PYG{n}{Year}\PYG{p}{,}\PYG{+w}{ }\PYG{n}{pop\PYGZus{}ke}\PYG{o}{\PYGZdl{}}\PYG{n}{POP}\PYG{p}{,}\PYG{+w}{ }\PYG{n}{type}\PYG{+w}{ }\PYG{o}{=}\PYG{+w}{ }\PYG{l+s}{\PYGZdq{}}\PYG{l+s}{o\PYGZdq{}}\PYG{p}{,}\PYG{+w}{ }\PYG{n}{lwd}\PYG{+w}{ }\PYG{o}{=}\PYG{+w}{ }\PYG{l+m}{2}\PYG{p}{,}
\PYG{+w}{     }\PYG{n}{main}\PYG{+w}{ }\PYG{o}{=}\PYG{+w}{ }\PYG{l+s}{\PYGZdq{}}\PYG{l+s}{Population of Kenya over time\PYGZdq{}}\PYG{p}{,}
\PYG{+w}{     }\PYG{n}{ylab}\PYG{+w}{ }\PYG{o}{=}\PYG{+w}{ }\PYG{l+s}{\PYGZdq{}}\PYG{l+s}{Population Kenya\PYGZdq{}}\PYG{p}{,}
\PYG{+w}{     }\PYG{n}{xlab}\PYG{+w}{ }\PYG{o}{=}\PYG{+w}{ }\PYG{l+s}{\PYGZdq{}}\PYG{l+s}{Year\PYGZdq{}}\PYG{p}{,}\PYG{+w}{ }\PYG{n}{pch}\PYG{+w}{ }\PYG{o}{=}\PYG{+w}{ }\PYG{l+m}{21}\PYG{p}{,}\PYG{+w}{ }\PYG{n}{bg}\PYG{+w}{ }\PYG{o}{=}\PYG{+w}{ }\PYG{l+s}{\PYGZdq{}}\PYG{l+s}{blue\PYGZdq{}}\PYG{p}{,}\PYG{+w}{ }\PYG{n}{col}\PYG{+w}{ }\PYG{o}{=}\PYG{+w}{ }\PYG{l+s}{\PYGZdq{}}\PYG{l+s}{red\PYGZdq{}}\PYG{p}{)}
\end{sphinxVerbatim}

\end{sphinxuseclass}\end{sphinxVerbatimInput}
\begin{sphinxVerbatimOutput}

\begin{sphinxuseclass}{cell_output}
\noindent\sphinxincludegraphics{{73070ebd2cc9d351ceb2d9e7675554bf524dde4bb8e9f7804e1efff37c8f1273}.png}

\end{sphinxuseclass}\end{sphinxVerbatimOutput}

\end{sphinxuseclass}
\begin{sphinxuseclass}{cell}\begin{sphinxVerbatimInput}

\begin{sphinxuseclass}{cell_input}
\begin{sphinxVerbatim}[commandchars=\\\{\}]
\PYG{n}{pop\PYGZus{}ke}\PYG{o}{\PYGZdl{}}\PYG{n}{POP}\PYG{p}{[}\PYG{n}{pop\PYGZus{}ke}\PYG{o}{\PYGZdl{}}\PYG{n}{Year}\PYG{+w}{ }\PYG{o}{==}\PYG{+w}{ }\PYG{l+s}{\PYGZdq{}}\PYG{l+s}{2019\PYGZdq{}}\PYG{p}{]}
\end{sphinxVerbatim}

\end{sphinxuseclass}\end{sphinxVerbatimInput}
\begin{sphinxVerbatimOutput}

\begin{sphinxuseclass}{cell_output}\begin{equation*}
\begin{split}52348340\end{split}
\end{equation*}
\end{sphinxuseclass}\end{sphinxVerbatimOutput}

\end{sphinxuseclass}
\begin{sphinxuseclass}{cell}\begin{sphinxVerbatimInput}

\begin{sphinxuseclass}{cell_input}
\begin{sphinxVerbatim}[commandchars=\\\{\}]
\PYG{n}{pop\PYGZus{}ke}\PYG{o}{\PYGZdl{}}\PYG{n}{Annual\PYGZus{}Growth}\PYG{+w}{  }\PYG{o}{\PYGZlt{}\PYGZhy{}}\PYG{+w}{ }\PYG{n+nf}{c}\PYG{p}{(}\PYG{k+kc}{NA}\PYG{p}{,}\PYG{+w}{ }\PYG{n+nf}{diff}\PYG{p}{(}\PYG{n}{pop\PYGZus{}ke}\PYG{o}{\PYGZdl{}}\PYG{n}{POP}\PYG{p}{)}\PYG{p}{)}
\PYG{n+nf}{head}\PYG{p}{(}\PYG{n}{pop\PYGZus{}ke}\PYG{p}{)}
\end{sphinxVerbatim}

\end{sphinxuseclass}\end{sphinxVerbatimInput}
\begin{sphinxVerbatimOutput}

\begin{sphinxuseclass}{cell_output}\begin{equation*}
\begin{split}A data.frame: 6 × 3
\begin{tabular}{r|lll}
  & Year & POP & Annual\_Growth\\
  & <chr> & <dbl> & <dbl>\\
\hline
	1979 & 1979 & 15690220 &     NA\\
	1980 & 1980 & 16330790 & 640570\\
	1981 & 1981 & 16986998 & 656208\\
	1982 & 1982 & 17658853 & 671855\\
	1983 & 1983 & 18345591 & 686738\\
	1984 & 1984 & 19046144 & 700553\\
\end{tabular}\end{split}
\end{equation*}
\end{sphinxuseclass}\end{sphinxVerbatimOutput}

\end{sphinxuseclass}
\begin{sphinxuseclass}{cell}\begin{sphinxVerbatimInput}

\begin{sphinxuseclass}{cell_input}
\begin{sphinxVerbatim}[commandchars=\\\{\}]
\PYG{n+nf}{mean}\PYG{p}{(}\PYG{n}{pop\PYGZus{}ke}\PYG{o}{\PYGZdl{}}\PYG{n}{Annual\PYGZus{}Growth}\PYG{p}{,}\PYG{+w}{ }\PYG{n}{na.rm}\PYG{+w}{ }\PYG{o}{=}\PYG{+w}{ }\PYG{n+nb+bp}{T}\PYG{p}{)}
\end{sphinxVerbatim}

\end{sphinxuseclass}\end{sphinxVerbatimInput}
\begin{sphinxVerbatimOutput}

\begin{sphinxuseclass}{cell_output}\begin{equation*}
\begin{split}940036.386363636\end{split}
\end{equation*}
\end{sphinxuseclass}\end{sphinxVerbatimOutput}

\end{sphinxuseclass}
\begin{sphinxuseclass}{cell}\begin{sphinxVerbatimInput}

\begin{sphinxuseclass}{cell_input}
\begin{sphinxVerbatim}[commandchars=\\\{\}]
\PYG{n}{mod}\PYG{+w}{ }\PYG{o}{\PYGZlt{}\PYGZhy{}}\PYG{+w}{ }\PYG{n+nf}{lm}\PYG{p}{(}\PYG{n}{POP}\PYG{+w}{ }\PYG{o}{\PYGZti{}}\PYG{+w}{ }\PYG{n+nf}{as.numeric}\PYG{p}{(}\PYG{n}{Year}\PYG{p}{)}\PYG{p}{,}\PYG{+w}{ }\PYG{n}{data}\PYG{+w}{ }\PYG{o}{=}\PYG{+w}{ }\PYG{n}{pop\PYGZus{}ke}\PYG{p}{)}
\PYG{n}{mod}
\end{sphinxVerbatim}

\end{sphinxuseclass}\end{sphinxVerbatimInput}
\begin{sphinxVerbatimOutput}

\begin{sphinxuseclass}{cell_output}
\begin{sphinxVerbatim}[commandchars=\\\{\}]
Call:
lm(formula = POP \PYGZti{} as.numeric(Year), data = pop\PYGZus{}ke)

Coefficients:
     (Intercept)  as.numeric(Year)  
      \PYGZhy{}1.856e+09         9.443e+05  
\end{sphinxVerbatim}

\end{sphinxuseclass}\end{sphinxVerbatimOutput}

\end{sphinxuseclass}
\begin{sphinxuseclass}{cell}\begin{sphinxVerbatimInput}

\begin{sphinxuseclass}{cell_input}
\begin{sphinxVerbatim}[commandchars=\\\{\}]
\PYG{n}{mod}\PYG{+w}{ }\PYG{o}{\PYGZlt{}\PYGZhy{}}\PYG{+w}{ }\PYG{n+nf}{lm}\PYG{p}{(}\PYG{n}{POP}\PYG{+w}{ }\PYG{o}{\PYGZti{}}\PYG{+w}{ }\PYG{n+nf}{as.numeric}\PYG{p}{(}\PYG{n}{Year}\PYG{p}{)}\PYG{p}{,}\PYG{+w}{ }\PYG{n}{data}\PYG{+w}{ }\PYG{o}{=}\PYG{+w}{ }\PYG{n}{pop\PYGZus{}ke}\PYG{p}{)}

\PYG{n+nf}{plot}\PYG{p}{(}\PYG{n}{pop\PYGZus{}ke}\PYG{o}{\PYGZdl{}}\PYG{n}{Year}\PYG{p}{,}\PYG{+w}{ }\PYG{n}{pop\PYGZus{}ke}\PYG{o}{\PYGZdl{}}\PYG{n}{POP}\PYG{p}{,}\PYG{+w}{ }\PYG{n}{type}\PYG{+w}{ }\PYG{o}{=}\PYG{+w}{ }\PYG{l+s}{\PYGZdq{}}\PYG{l+s}{o\PYGZdq{}}\PYG{p}{,}\PYG{+w}{ }\PYG{n}{lwd}\PYG{+w}{ }\PYG{o}{=}\PYG{+w}{ }\PYG{l+m}{2}\PYG{p}{,}
\PYG{+w}{     }\PYG{n}{main}\PYG{+w}{ }\PYG{o}{=}\PYG{+w}{ }\PYG{l+s}{\PYGZdq{}}\PYG{l+s}{Population of Kenya over time\PYGZdq{}}\PYG{p}{,}
\PYG{+w}{     }\PYG{n}{ylab}\PYG{+w}{ }\PYG{o}{=}\PYG{+w}{ }\PYG{l+s}{\PYGZdq{}}\PYG{l+s}{Population Kenya in Millions\PYGZdq{}}\PYG{p}{,}
\PYG{+w}{     }\PYG{n}{xlab}\PYG{+w}{ }\PYG{o}{=}\PYG{+w}{ }\PYG{l+s}{\PYGZdq{}}\PYG{l+s}{Year\PYGZdq{}}\PYG{p}{,}\PYG{+w}{ }\PYG{n}{pch}\PYG{+w}{ }\PYG{o}{=}\PYG{+w}{ }\PYG{l+m}{16}\PYG{p}{)}
\PYG{n+nf}{abline}\PYG{p}{(}\PYG{n}{mod}\PYG{p}{,}\PYG{+w}{ }\PYG{n}{col}\PYG{+w}{ }\PYG{o}{=}\PYG{+w}{ }\PYG{l+s}{\PYGZdq{}}\PYG{l+s}{blue\PYGZdq{}}\PYG{p}{)}
\end{sphinxVerbatim}

\end{sphinxuseclass}\end{sphinxVerbatimInput}
\begin{sphinxVerbatimOutput}

\begin{sphinxuseclass}{cell_output}
\noindent\sphinxincludegraphics{{f938d4897e5abc5d7c14822c2c187da2f416b0620dce23b19c2b25b141946b42}.png}

\end{sphinxuseclass}\end{sphinxVerbatimOutput}

\end{sphinxuseclass}
\sphinxAtStartPar
The predictions are pretty close and differ by about 3 Mio.

\sphinxstepscope


\chapter{Exercises: Unit 5, How sure can we be about what is going on: Estimates and Intervals}
\label{\detokenize{exercises_unit_5:exercises-unit-5-how-sure-can-we-be-about-what-is-going-on-estimates-and-intervals}}\label{\detokenize{exercises_unit_5::doc}}
\sphinxstepscope


\chapter{Exercises}
\label{\detokenize{exercises_unit_6:exercises}}\label{\detokenize{exercises_unit_6::doc}}
\sphinxstepscope


\chapter{Exercises}
\label{\detokenize{exercises_unit_7:exercises}}\label{\detokenize{exercises_unit_7::doc}}
\sphinxstepscope


\chapter{Exercises}
\label{\detokenize{exercises_unit_8:exercises}}\label{\detokenize{exercises_unit_8::doc}}






\renewcommand{\indexname}{Index}
\printindex
\end{document}